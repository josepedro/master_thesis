\begin{tikzpicture}
  \begin{axis}[
  width=0.9\textwidth,
  height=0.5\textwidth,
  % x tick label style={
  %     /pgf/number format/.cd,
  %         fixed,
  %         fixed zerofill,
  %         precision=1,
  %     /tikz/.cd
  % },
  xmin=0,
  xmax=1.8,
  ymin=0,
  ymax=1.4,
  ytick distance=0.2,
  xtick distance=0.2,
  grid=major, % Display a grid  
  %grid style={dashed,gray!90}, % Set the style
  xlabel = $ka$,
  ylabel = $l/a$,
  y tick label style={/pgf/number format/.cd,%
          scaled y ticks = false,
          set decimal separator={,},
          fixed},
      x tick label style={/pgf/number format/.cd,%
          scaled x ticks = false,
          set decimal separator={,},
          fixed}%
  ]
 \addplot[color=black, mark=o] table[x index=0,y index=1] {dados/duto_sugado/loa_049_kp.txt};
 \addplot[color=black, mark=square] table[x index=0,y index=1] {dados/duto_sugado/loa_059_kp.txt};
 \addplot[color=black, mark=triangle] table[x index=0,y index=1] {dados/duto_sugado/loa_071_kp.txt};
\addplot[color=black, mark=x] table[x index=0,y index=1] {dados/duto_sugado/loa_074_kp.txt};
\addplot[color=black, mark=diamond] table[x index=0,y index=1] {dados/duto_sugado/loa_075_kp.txt};

  \end{axis}
  \end{tikzpicture}
  \caption[Coeficiente de correção da terminação $l/a$ com escoamentos sugados]{Resultados de $l/a$ em função de $ka$ para diferentes números de Mach com escoamento sugado. com vários escoamentos sugados. Os pontos com $\bigcirc$, $\square$, $\bigtriangleup$, $\times$ e $\diamond$  apresentam os resultados para $M =$ 0,05 e $Re =$ 1378,73, $M =$ 0,07 e $Re =$ 1930,23, $M =$ 0,1 e $Re =$ 2757,42, $M =$ 0,15 e $Re =$ 2057,71 e $M =$ 0,20 e $Re =$ 5514,82 respectivamente.}

  \label{fig:loa_sugado_kp}