\begin{tikzpicture}
  \begin{axis}[
  width=0.9\textwidth,
  height=0.5\textwidth,
  y tick label style={
      /pgf/number format/.cd,
          fixed,
          fixed zerofill,
          precision=3,
      /tikz/.cd
  },
  xmin=0,
  xmax=3,
  ymin=0.0015,
  ymax=0.011,
  ytick distance=0.002,
  xtick distance=0.5,
  grid=major, % Display a grid  
  %grid style={dashed,gray!90}, % Set the style
  xlabel = $(x - \Delta)/a$,
  ylabel = $\text{rms}(u')/Mc_{s}$,
  y tick label style={/pgf/number format/.cd,%
          scaled y ticks = false,
          set decimal separator={,},
          fixed},
      x tick label style={/pgf/number format/.cd,%
          scaled x ticks = false,
          set decimal separator={,},
          fixed}%
  ]
 \addplot[color=black, thick] table[x index=0,y index=1] {dados/duto_sugado/intensidade_turbulenta_dados.txt};

  \end{axis}
  \end{tikzpicture}
  \caption[Intensidade turbulenta para $M = 0,07$]{Intensidade turbulenta calculada na camada limite para $M = 0,07$ em relação a distância da terminação para o interior interior do duto.}

  \label{fig:energia_turbulenta}