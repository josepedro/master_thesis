\begin{tikzpicture}
  \begin{axis}[
  width=0.8\textwidth,
  height=0.4\textwidth,
  % x tick label style={
  %     /pgf/number format/.cd,
  %         fixed,
  %         fixed zerofill,
  %         precision=1,
  %     /tikz/.cd
  % },
  xmin=0,
  xmax=2.4,
  ymin=0,
  ymax=1.4,
  ytick distance=0.2,
  xtick distance=0.5,
  grid=major, % Display a grid  
  %grid style={dashed,gray!90}, % Set the style
  xlabel = $ka$,
  ylabel = $|R|$,
   y tick label style={/pgf/number format/.cd,%
          scaled y ticks = false,
          set decimal separator={,},
          fixed},
      x tick label style={/pgf/number format/.cd,%
          scaled x ticks = false,
          set decimal separator={,},
          fixed}%
  ]
 \addplot[color=black, mark=o] table[x index=0,y index=1] {dados/coeficiente_reflexao_anecoica/A_abs.txt};
 \addplot[color=black, mark=square] table[x index=0,y index=1] {dados/coeficiente_reflexao_anecoica/B_abs.txt};
 \addplot[color=black, mark=triangle] table[x index=0,y index=1] {dados/coeficiente_reflexao_anecoica/C_abs.txt};
 \addplot[color=black, mark=x] table[x index=0,y index=1] {dados/coeficiente_reflexao_anecoica/D_abs.txt};

  \end{axis}
  \end{tikzpicture}
  \caption[Coeficiente de reflexão $|R|$ na condição anecóica.]{Resultados do coeficiente de reflexão $|R|$ na condição anecóica localizada nas fronteiras do 
 modelo numérico. Os pontos com $\bigcirc$, $\square$, $\bigtriangleup$ e $\times$  
 apresentam os resultados para os pontos 
 \textbf{A}, \textbf{B}, \textbf{C} e \textbf{D} respectivamente.}
  \label{fig:abs_r_anecoica}