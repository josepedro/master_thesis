% parte real impedancia 
\begin{tikzpicture}
\begin{axis}[
width=\scalexA \textwidth,
height=\scaleyA \textwidth,
xmin=0,
xmax=2.5,
ymin=0,
ymax=1,
ytick distance=0.1,
xtick distance=0.5,
grid=major, % Display a grid  
%grid style={dashed,gray!90}, % Set the style
xlabel = \small{$ka$},
ylabel = \small{$\Re(Z)$},
 y tick label style={/pgf/number format/.cd,%
          scaled y ticks = false,
          set decimal separator={,},
          fixed},
      x tick label style={/pgf/number format/.cd,%
          scaled x ticks = false,
          set decimal separator={,},
          fixed}%
]

\addplot[color=black, mark=o, only marks] table[x index=0,y index=1] {dados/coeficiente_reflexao_anecoica/A_real.txt};
% \addplot[color=black, mark=square, only marks] table[x index=0,y index=1] {dados/coeficiente_reflexao_anecoica/B_real_1.txt};
 %\addplot[color=black, mark=triangle, only marks] table[x index=0,y index=1] {dados/coeficiente_reflexao_anecoica/C_real_1.txt};
 %\addplot[color=black, mark=x, only marks] table[x index=0,y index=1] {dados/coeficiente_reflexao_anecoica/D_real_1.txt};

\end{axis}
\end{tikzpicture}
\caption[Impedância $Z_{\textbf{A}}$]{}
\label{fig:parte_real}
%\caption[Impedância $Z_{\textbf{A}}$]{Resultado da impedância calculada no ponto $\textbf{A}$, próximo da condição anecóica localizada nas fronteiras do modelo numérico. A linha contínua representa a parte real e a linha tracejada representa a parte imaginária.}