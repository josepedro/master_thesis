\begin{tikzpicture}
  \begin{axis}[
  width=0.9\textwidth,
  height=0.5\textwidth,
  % x tick label style={
  %     /pgf/number format/.cd,
  %         fixed,
  %         fixed zerofill,
  %         precision=1,
  %     /tikz/.cd
  % },
  xmin=0,
  xmax=16,
  ymin=0.55,
  ymax=1.2,
  ytick distance=0.1,
  xtick distance=2,
  grid=major, % Display a grid  
  %grid style={dashed,gray!90}, % Set the style
  xlabel = $St$,
  ylabel = $|R_{r}|$,
  y tick label style={/pgf/number format/.cd,%
          scaled y ticks = false,
          set decimal separator={,},
          fixed},
      x tick label style={/pgf/number format/.cd,%
          scaled x ticks = false,
          set decimal separator={,},
          fixed}%
  ]
 \addplot[color=black, mark=x, only marks] table[x index=0,y index=1] {dados/duto_exaustao/abs_r_007_simulation_strouhal.txt};
 \addplot[color=black, mark=o, only marks] table[x index=0,y index=1] {dados/duto_exaustao/abs_r_010_simulation_strouhal.txt};
 \addplot[color=black, mark=square, only marks] table[x index=0,y index=1] {dados/duto_exaustao/abs_r_015_simulation_strouhal.txt};
 \addplot[color=black, mark=triangle, only marks] table[x index=0,y index=1] {dados/duto_exaustao/abs_r_020_simulation_strouhal.txt};

  \end{axis}
  \end{tikzpicture}
  \caption[Coeficiente de reflexão $|R_{r}|$ com escoamento de exaustão em relação ao número de Strouhal ($St$)]{Resultado de magnitudes do coeficiente de reflexão $|R_{r}|$ em relação ao número de Strouhal ($St$) calculados no ponto $\textbf{P}$ na terminação do duto com vários escoamentos de exaustão. Os pontos com $\times$, $\bigcirc$, $\square$ e $\bigtriangleup$  apresentam os resultados para os números de Mach $M =$ 0,07, $M =$ 0,10, $M =$ 0,15 e $M =$ 0,20 respectivamente.}

  \label{fig:abs_r_exaustao_strouhal}