\begin{tikzpicture}
  \begin{axis}[
  width=0.9\textwidth,
  height=0.5\textwidth,
  x tick label style={
      /pgf/number format/.cd,
          fixed,
          fixed zerofill,
          precision=2,
      /tikz/.cd
  },
  xmin=0.05,
  xmax=0.2,
  ymin=0.6826,
  ymax=1.3,
  ytick distance=0.1,
  xtick distance=0.03,
  grid=major, % Display a grid  
  %grid style={dashed,gray!90}, % Set the style
  xlabel = $M$,
  ylabel = $|R_{r}|$,
   y tick label style={/pgf/number format/.cd,%
          scaled y ticks = false,
          set decimal separator={,},
          fixed},
      x tick label style={/pgf/number format/.cd,%
          scaled x ticks = false,
          set decimal separator={,},
          fixed}%
  ]
 \addplot[color=black, thick] table[x index=0,y index=1] {dados/duto_sugado/abs_r_strouhal_mach.txt};

  \end{axis}
  \end{tikzpicture}
  \caption[Coeficiente de reflexão $R_{r}$ com escoamento de exaustão em relação ao número de Mach ($M$) no Strouhal $St = \pi/2$]{Resultado de magnitudes do coeficiente de reflexão $R_{r}$ fixados no Strouhal $St \sim \pi/2$ em relação ao número de Mach ($M$) para escoamentos sugados. Os resultados foram calculados no ponto $\textbf{P}$ na terminação do duto.}

  \label{fig:abs_r_sugado_strouhal_mach}