\chapter{Resultados}

Em vista da teoria vigente na literatura e pelo que foi exposto no ponto de vista metodológico, obteve-se resultados nos seguintes contextos:
\begin{itemize}
  \item análise da condição anecóica: a partir dos históricos temporais de pressão e velocidade de partícula nas fronteiras do modelo numérico, foram feitas análises críticas de reflexão acústica;
  \item duto sem escoamento: a partir dos históricos temporais de pressão e velocidade de partícula na terminação do duto, foram feitas validações e análises críticas dos parâmetros caracterizadores da acústica interna do duto sem escoamento;
  \item duto com escoamento de exaustão: a partir dos históricos temporais de pressão e velocidade de partícula na terminação do duto, foram feitas validações e análises críticas dos parâmetros caracterizadores da acústica interna do duto com escoamento de exaustão para regimes subsônicos ($M$ $\leq$ 0,2);
  \item duto com escoamento sugado: a partir dos históricos temporais de pressão e velocidade de partícula na terminação do duto, foram feitas validações, análises críticas e investigação dos parâmetros caracterizadores da acústica interna do duto com escoamento succionado para regimes subsônicos ($M$ $\leq$ 0,2).
\end{itemize} 

\section{Análise da Condição Anecóica}

Com a finalidade de mensurar e analisar o comportamento da condição de contorno anecóica por meio de métricas numéricas e objetivas, foram calculados impedâncias e coeficientes de reflexão nas fronteiras do modelo numérico, ou seja, nos pontos \textbf{A}, \textbf{B}, \textbf{C} e \textbf{D} como é mostrado na Figura \ref{fig:modelo}. As Figuras \ref{fig:resultados_A}, \ref{fig:resultados_B}, \ref{fig:resultados_C} e \ref{fig:resultados_D} mostram os resultados nos respectivos pontos citados.


% \begin{figure}
% \begin{center}
% \begin{tikzpicture}
% \begin{axis}[
%     title={},
%     width=0.9\textwidth,
%     height=0.45\textwidth,
%     xlabel={Frequência},
%     ylabel={Sensibilidade},
%     x unit={\space Hz \space},
% 	y unit={\space C/g \space},
%     ytick=data,
%     xmin=0,
%     ymin=0,
%     ymax=0.55,
%     legend pos=north west,
%     grid=minor, % Display a grid	
% 	grid style={dashed,gray!90}, % Set the style
% 	]
%      %\addplot[color=black,dashed,thick,mark=*,mark options={solid},smooth] table[x index=2,y index=3] {Data/Kollias_SBMR_095_exp.txt}; \label{Kollias_exp095}
%      %\addlegendentry{Experimental (Kollias)}
%       %\addplot[color=blue,semithick] table[x index=0,y index=1] {Data/Kollias_SBMR_095_exp.txt}; \label{Kollias_sim095}
%     %\addlegendentry{Simulação (Kollias)}
%              \addplot[color=black, thick] table[x index=0,y index=1] {dados/coeficiente_reflexao_anecoica/A_real.txt};
%              \addplot[color=black,dashed,  thick] table[x index=0,y index=1] {dados/coeficiente_reflexao_anecoica/A_imag.txt};
%               \label{Comsol_095}% \addlegendentry{Simulação (Comsol)}
% \end{axis}
% \end{tikzpicture}
%  \caption{Resposta em carga do acelerômetro para $r$ igual a 0,95.}
% \end{center}
% \end{figure}

\newcommand\scalexA{0.8}
\newcommand\scaleyA{0.8}
\newcommand\scalex{1}
\newcommand\scaley{1}
\newcommand\scaleA{0.5}

\newpage

A Figura \ref{fig:resultados_A} mostra os resultados de parâmetros de caracterização de reflexão no ponto \textbf{A}, ou seja, num ponto que está a frente da terminação do duto. Pode-se observar pela Figura \ref{fig:A_impedancia} que a parte real da impedância converge para o valor de $\rho_{0} c_{0}$ de referência na literatura, que em unidades do LBM possui o valor igual 0,57735. A parte imaginária da Figura \ref{fig:A_impedancia} converge num valor bem abaixo, porém ainda distante de 0 como é o ideal de uma situação totalmente anecóica. A Figura \ref{fig:A_reflexao} confirma o fato que se vê na Figura \ref{fig:A_impedancia}, ou seja, vê-se a mesma tendência do coeficiente de reflexão na parte imaginária do gráfico de impedância, ocasionando aproxidamente 15\% de reflexão. Vale ressaltar também da Figura \ref{fig:A_reflexao} que para baixas frequências há uma alta reflexão. 

\begin{figure}[ht!]
\begin{subfigure}{\scaleA \textwidth}
  \input{figuras/impedancia_A.tex}
\end{subfigure}%
\begin{subfigure}{\scaleA \textwidth}
  \input{figuras/reflexao_A.tex}
\end{subfigure}
\caption[Resultados de reflexão no ponto \textbf{A}]{Resultados da impedância $Z_{\textbf{A}}$ (\ref{fig:A_impedancia}) e módulo do coeficiente de reflexão $|R_{\textbf{A}}|$ (\ref{fig:A_reflexao}) calculados no ponto $\textbf{A}$. Na Figura (\ref{fig:A_impedancia}) a linha contínua representa a parte real e a linha tracejada representa a parte imaginária.}
\label{fig:resultados_A}
\end{figure}

A Figura \ref{fig:resultados_B} mostra os resultados de parâmetros de caracterização de reflexão no ponto \textbf{B}, ou seja, numa região de descontinuidade na direção $z$ e $y$. Pode-se observar pela Figura \ref{fig:B_impedancia} que a parte real e imaginária da impedância diverge consideravelmente de uma condição anecóica ideal. Tal fato pode ser verificado na Figura \ref{fig:B_reflexao} que, embora o coeficiente de reflexão não convirja para 1, há a presença de reflexões principalmente em baixas frequências. Mesmo o ponto \textbf{B} tendo uma tendência distoante de uma condição anecóica ideal, considera-se tal fato como desprezível visto que há poucas regiões descontínuas dessa condição de contorno no modelo numérico. 

\begin{figure}
\begin{subfigure}{\scaleA \textwidth}
  \input{figuras/impedancia_B.tex}
\end{subfigure}%
\begin{subfigure}{\scaleA \textwidth}
  \input{figuras/reflexao_B.tex}
\end{subfigure}
\caption[Resultados de reflexão no ponto \textbf{B}]{Resultados da impedância $Z_{\textbf{B}}$ (\ref{fig:B_impedancia}) e módulo do coeficiente de reflexão $|R_{\textbf{B}}|$ (\ref{fig:B_reflexao}) calculados no ponto $\textbf{B}$. Na Figura (\ref{fig:B_impedancia}) a linha contínua representa a parte real e a linha tracejada representa a parte imaginária.}
\label{fig:resultados_B}
\end{figure}

\newpage
A Figura \ref{fig:resultados_C} mostra os resultados de parâmetros de caracterização de reflexão no ponto \textbf{C}. Pode-se observar pela Figura \ref{fig:C_impedancia} que a parte real da impedância converge, para baixas e médias frequências, para o valor de $\rho_{0} c_{0}$ de referência na literatura, que em unidades do LBM possui o valor igual 0,57735. A parte imaginária da Figura \ref{fig:C_impedancia} converge num valor bem abaixo para baixas e médias frequências, porém ainda distante de 0 como é o ideal de uma situação totalmente anecóica. A Figura \ref{fig:C_reflexao} confirma o fato que se vê na Figura \ref{fig:C_impedancia}, ou seja, vê-se a mesma tendência do coeficiente de reflexão na parte imaginária do gráfico de impedância, ocasionando aproxidamente 15\% de reflexão.

A Figura \ref{fig:resultados_D} mostra os resultados de parâmetros de caracterização de reflexão no ponto \textbf{D}, numa posição atrás da terminação duto. Essa região, quando não está corretamente ajustada no que diz respeito a condição anecóica, pode-se caracterizar como parede rígida refletora e distorcer o comportamento para um contexto de duto extendido a partir de uma parede rígida como mostra o estudo de \citeonline{selamet2001wave}. Contudo, tal região se comporta como aproximadamente anecóica pois possui características congruentes ao ponto \textbf{C} no que diz respeito a impedância (Figura \ref{fig:D_impedancia}) e ao coeficiente de reflexão (Figura \ref{fig:D_reflexao}).

\begin{figure}
\begin{subfigure}{\scaleA \textwidth}
  \begin{tikzpicture}
  \begin{axis}[
	width=\scalexA \textwidth,
  height=\scaleyA \textwidth,
	xmin=0,
  xmax=2.5,
    ymin=0,
    ymax=0.7,
    ytick distance=0.1,
    xtick distance=0.5,
    grid=major, % Display a grid	
 	%grid style={dashed,gray!90}, % Set the style
	xlabel = \small{$ka$},
	ylabel = \small{$Z_{\textbf{C}}$},
   y tick label style={/pgf/number format/.cd,%
          scaled y ticks = false,
          set decimal separator={,},
          fixed},
      x tick label style={/pgf/number format/.cd,%
          scaled x ticks = false,
          set decimal separator={,},
          fixed}%
  ]
 \addplot[color=black, thick] table[x index=0,y index=1] {dados/coeficiente_reflexao_anecoica/C_real.txt};
   \addplot[color=black,dashed,  thick] table[x index=0,y index=1] {dados/coeficiente_reflexao_anecoica/C_imag.txt};
   
  \end{axis}
  \end{tikzpicture}
  \caption[Impedância $Z_{\textbf{C}}$]{}
  \label{fig:C_impedancia}
  %\caption[Impedância $Z_{\textbf{C}}$]{Resultado da impedância calculada no ponto $\textbf{C}$, próximo da condição anecóica localizada nas fronteiras do modelo numérico. A linha contínua representa a parte real e a linha tracejada representa a parte imaginária.}

\end{subfigure}%
\begin{subfigure}{\scaleA \textwidth}
  \input{figuras/reflexao_C.tex}
\end{subfigure}
\caption[Resultados de reflexão no ponto \textbf{C}]{Resultados da impedância $Z_{\textbf{C}}$ (\ref{fig:C_impedancia}) e módulo do coeficiente de reflexão $|R_{\textbf{C}}|$ (\ref{fig:C_reflexao}) calculados no ponto $\textbf{C}$. Na Figura (\ref{fig:C_impedancia}) a linha contínua representa a parte real e a linha tracejada representa a parte imaginária.}
\label{fig:resultados_C}
\end{figure}

\begin{figure}
\begin{subfigure}{\scaleA \textwidth}
  \input{figuras/impedancia_D.tex}
\end{subfigure}%
\begin{subfigure}{\scaleA \textwidth}
  \input{figuras/reflexao_D.tex}
\end{subfigure}
\caption[Resultados de reflexão no ponto \textbf{D}]{Resultados da impedância $Z_{\textbf{D}}$ (\ref{fig:D_impedancia}) e módulo do coeficiente de reflexão $|R_{\textbf{D}}|$ (\ref{fig:D_reflexao}) calculados no ponto $\textbf{D}$. Na Figura (\ref{fig:D_impedancia}) a linha contínua representa a parte real e a linha tracejada representa a parte imaginária.}
\label{fig:resultados_D}
\end{figure}

\newpage
Em vista do que foi exposto através de uma análise de reflexão acústica nos pontos \textbf{A}, \textbf{B}, \textbf{C} e \textbf{D}, a condição de contorno na fronteira do domínio pode ser considerada como aproximadamente anecóica.

\section{Duto sem Escoamento}

A primeira etapa de validação do modelo numérico abordado nesse trabalho consiste num duto sem escoamento e com fonte acústica variando nos valores de $0 \leq ka \leq 1,8$. Nessa etapa são calculados os coeficientes de reflexão e correção da terminação (ponto \textbf{P}) e comparados com os resultados de \citeonline{levine1948radiation}. Porém, considerando que para cada razão elemento por comprimento de onda terá um custo computacional e acurácia nos resultados, é preciso realizar uma análise de convergência para mensurar qual discretização é mais adequada. Portanto foram avaliadas quatro tipos de discretizações para representar o raio do duto: 20, 15, 10 e 5. Não foi possível realizar com raio maior que 20 devido a limitação de memória. A Figura \ref{fig:resultados_sem_escoamento} apresenta os resultados com as discretizações citadas.

\begin{figure}[h!]
\begin{subfigure}{\scaleA \textwidth}
  \input{figuras/abs_r_sem_escoamento.tex}
\end{subfigure}%
\begin{subfigure}{\scaleA \textwidth}
  \begin{tikzpicture}
  \begin{axis}[
  width=\scalex \textwidth,
  height=\scaley \textwidth,
  xmin=0,
  xmax=1.8,
    ymin=0,
    ymax=2,
    ytick distance=0.5,
    grid=major, % Display a grid  
  %grid style={dashed,gray!90}, % Set the style
  xlabel = \small{$ka$},
  ylabel = \small{$l/a$},
   y tick label style={/pgf/number format/.cd,%
          scaled y ticks = false,
          set decimal separator={,},
          fixed},
      x tick label style={/pgf/number format/.cd,%
          scaled x ticks = false,
          set decimal separator={,},
          fixed}%
  ]
 \addplot[color=black, thick] table[x index=0,y index=1] {dados/duto_sem_escoamento/analytical_data_loa.txt};
 \addplot[color=black, mark=o, only marks] table[x index=0,y index=1] {dados/duto_sem_escoamento/simulation_data_loa.txt};
  \addplot[color=black, mark=square, only marks] table[x index=0,y index=1] {dados/duto_sem_escoamento/analise_convergencia_10_la.txt};
 \addplot[color=black, mark=triangle, only marks] table[x index=0,y index=1] {dados/duto_sem_escoamento/analise_convergencia_15_la.txt};
 \addplot[color=black, mark=x, only marks] table[x index=0,y index=1] {dados/duto_sem_escoamento/analise_convergencia_5_la.txt};
 

  \end{axis}
  \end{tikzpicture}
  \caption[Coeficiente de Correção da Terminação $l/a$ sem Escoamento]{}
  \label{fig:loa_sem_escoamento}
\end{subfigure}
\caption[Resultados de $|R_{r}|$ e $l/a$ sem escoamento]{Resultados de $|R_{r}|$ (\ref{fig:abs_r_sem_escoamento}) e $l/a$ (\ref{fig:loa_sem_escoamento}) para duto sem escoamento. As linhas contínuas representam os resultados do estudo de \citeonline{levine1948radiation} e os pontos $\bigcirc$, $\bigtriangleup$, $\square$ e $\times$ apresentam os resultados para 20, 15, 10 e 5 elementos de discretização para representar o raio do duto respectivamente.}
\label{fig:resultados_sem_escoamento}
\end{figure}

É possível percebecer da Figura \ref{fig:resultados_sem_escoamento} que a medida que a discretização vai aumentando mais o modelo numérico converge para o valor da literatura do modelo analítico. É possível perceber que a melhor discretização é a de 20 elementos para descrever o tamanho do raio. Vale ressaltar que a tabela \ref{table:discretizacao} apresenta esse fato com vistas no tamanho do raio ($a$), número de elementos representativos ($\gamma$) para $ka = 1,8$ e as correlações $r_r$ para coeficiente de reflexão e $r_l$ para correção da terminação.    

\begin{table}[ht!]
\centering
\caption{Tamanho do raio, números de elementos representativos para $ka = 1,8$ e correlações.}
\label{table:discretizacao}
    \begin{tabular}{|l|l|l|l|}
        \hline
        $a$ & $\gamma$ & $r_r$ & $r_l$ \\ \hline
        20  & 70 & 99,95 \%  & 96,23 \%  \\ \hline
        15  & 52 &  86,08 \% & 86,84 \% \\ \hline
        10  & 35 &  70,65 \% & 68,43 \% \\ \hline
        5   & 17 &  61,23 \% & 48,06 \%  \\
        \hline
    \end{tabular}
\end{table}

Para a discretização com melhor correlação com resultados da literatura ($a = 20$) há comportamentos de $|R_{r}|$ e $l/a$ corretamente aderentes a fenomenologia puramente acústica. O Figura \ref{fig:abs_r_sem_escoamento} apresenta um alto coeficiente de reflexão para baixas frequências e o mesmo vai decando a medida que a frequência aumenta. Esse fato é devido a inércia do gás na terminação do duto que é muito mais preponderante nas baixas frequências do que nas altas. A Figura \ref{fig:loa_sem_escoamento} reflete a mesma atuação da inércia sobre as ondas acústicas, resultando num alto coeficiente de correção da terminação para baixas frequências e a diminuição do mesmo a medida que a frequência aumenta.       

% Comentar as correlações abaixo
% correlation =

    %0.7226


%correlation =

    %0.7217


%correlation =

    %0.8608


%correlation =

    %0.8584


%correlation =

    %0.9053


%correlation =

    %0.8990


\section{Duto com Escoamento de Exaustão}

Visto que o modelo numérico abordado está validado num contexto sem escoamento com uma discretização de elementos significativa ($a = 20$), há a necessidade de validá-lo com escoamento de exaustão usando os resultados na literatura desenvolvidos por \citeonline{munt1990acoustic}. Para isso foram escolhidos os números de Mach 0,07, 0,10, 0,15 e 0,20 para atender o requisito de  análises com escoamento de exaustão para $M \leq 0,2$. Além disso faz-se necessário análises do comportamento do coeficiente de reflexão em relação ao número de Strouhal visto que a natureza de fluxo de massa rotacional eclode mudanças consideráveis nessa métrica.       

\newpage
\begin{figure}[ht!]
\begin{subfigure}{\scaleA \textwidth}
  \input{figuras/abs_r_exaustao_007.tex}
\end{subfigure}%
\begin{subfigure}{\scaleA \textwidth}
  \input{figuras/loa_exaustao_007.tex}
\end{subfigure}
\caption[Resultados de $|R_{r}|$ e $l/a$ com escoamento de exaustão ($M =$ 0,07 e $Re =$ 1930,23)]{Resultados de $|R_{r}|$ (\ref{fig:abs_r_exaustao_007}) e $l/a$ (\ref{fig:loa_exaustao_007}) com escoamento de exaustão ($M =$ 0,07 e $Re =$ 1930,23). As linhas contínuas representam os resultados do estudo de \citeonline{munt1990acoustic} e os pontos circulares representam os resultados do modelo numérico.}
\label{fig:resultados_exaustao_007}
\end{figure}

A Figura \ref{fig:resultados_exaustao_007} apresenta os resultados da magnitude do coeficiente de reflexão $|R_{r}|$ (\ref{fig:abs_r_exaustao_007}) e do coeficiente de correção da terminação $l/a$ (\ref{fig:loa_exaustao_007}), calculados no ponto $\textbf{P}$ na terminação do duto com escoamento de exaustão: número de Mach $M =$ 0,07 e número de Reynolds $Re =$ 1930,23.  As correlações entre os resultados foram de 99,83 \% para o coeficiente de reflexão $|R_{r}|$ (\ref{fig:abs_r_exaustao_007}) e 87,76 \% para o coeficiente de correção da terminação $l/a$ (\ref{fig:loa_exaustao_007}). Apesar do gráfico \ref{fig:loa_exaustao_007} apresentar boa correlação com os resultados analíticos exatos do estudo de \citeonline{munt1990acoustic}, há uma divergência para baixas frequências que pode ser explicada pela limitação do método de medição que não considera processos de convecção do fluido.  

A Figura \ref{fig:resultados_exaustao_010} apresenta os resultados da magnitude do coeficiente de reflexão $|R_{r}|$ (\ref{fig:abs_r_exaustao_010}) e do coeficiente de correção da terminação $l/a$ (\ref{fig:loa_exaustao_010}), calculados no ponto $\textbf{P}$ na terminação do duto com escoamento de exaustão: número de Mach $M =$ 0,10 e número de Reynolds $Re =$ 2757,42.  As correlações entre os resultados foram de 99,85 \% para o coeficiente de reflexão $|R_{r}|$ (\ref{fig:abs_r_exaustao_010}) e 95,02 \% para o coeficiente de correção da terminação $l/a$ (\ref{fig:loa_exaustao_010}). Apesar do gráfico \ref{fig:loa_exaustao_010} apresentar boa correlação com os resultados analíticos exatos do estudo de \citeonline{munt1990acoustic}, há uma divergência nas baixas frequências ocasioanda pela mesma razão já comentada na Figura \ref{fig:resultados_exaustao_007}.

\begin{figure}[ht!]
\begin{subfigure}{\scaleA \textwidth}
  \input{figuras/abs_r_exaustao_010.tex}
\end{subfigure}%
\begin{subfigure}{\scaleA \textwidth}
  \begin{tikzpicture}
  \begin{axis}[
  width=\scalex \textwidth,
  height=\scaley \textwidth,
  xmin=0,
  xmax=1.8,
    ymin=0,
    ymax=0.9,
    ytick distance=0.1,
    grid=major, % Display a grid  
  %grid style={dashed,gray!90}, % Set the style
  xlabel = \small{$ka$},
  ylabel = \small{$l/a$},
   y tick label style={/pgf/number format/.cd,%
          scaled y ticks = false,
          set decimal separator={,},
          fixed},
      x tick label style={/pgf/number format/.cd,%
          scaled x ticks = false,
          set decimal separator={,},
          fixed}%
  ]
 \addplot[color=black, thick] table[x index=0,y index=1] {dados/duto_exaustao/loa_010_analytical.txt};
 \addplot[color=black, mark=o, only marks] table[x index=0,y index=1] {dados/duto_exaustao/loa_010_simulation.txt};
   
  \end{axis}
  \end{tikzpicture}
  \caption[Coeficiente de Correção da Terminação $l/a$ com Escoamento de Exaustão (M $=$ 0,10)]{}
  \label{fig:loa_exaustao_010}
\end{subfigure}
\caption[Resultados de $|R_{r}|$ e $l/a$ com escoamento de exaustão ($M =$ 0,10 e $Re =$ 2757,42)]{Resultados de $|R_{r}|$ (\ref{fig:abs_r_exaustao_010}) e $l/a$ (\ref{fig:loa_exaustao_010}) com escoamento de exaustão ($M =$ 0,10 e $Re =$ 2757,42). As linhas contínuas representam os resultados do estudo de \citeonline{munt1990acoustic} e os pontos circulares representam os resultados do modelo numérico.}
\label{fig:resultados_exaustao_010}
\end{figure}

\newpage
A Figura \ref{fig:resultados_exaustao_015} apresenta os resultados da magnitude do coeficiente de reflexão $|R_{r}|$ (\ref{fig:abs_r_exaustao_015}) e do coeficiente de correção da terminação $l/a$ (\ref{fig:loa_exaustao_015}), calculados no ponto $\textbf{P}$ na terminação do duto com escoamento de exaustão: número de Mach $M =$ 0,15 e número de Reynolds $Re =$ 2057,71.  As correlações entre os resultados foram de 99,80 \% para o coeficiente de reflexão $|R_{r}|$ (\ref{fig:abs_r_exaustao_015}) e 94,28 \% para o coeficiente de correção da terminação $l/a$ (\ref{fig:loa_exaustao_015}). Apesar do gráfico \ref{fig:loa_exaustao_015} apresentar boa correlação com os resultados analíticos exatos do estudo de \citeonline{munt1990acoustic}, há uma divergência nas baixas frequências ocasioanda pela mesma razão já comentada nas Figuras \ref{fig:resultados_exaustao_007} e \ref{fig:resultados_exaustao_010}. Vale ressaltar que apesar do número de Mach dos resultados da Figura \ref{fig:resultados_exaustao_015} ser maior que dos resultados da Figura \ref{fig:resultados_exaustao_010}, o número de Reynolds é menor, nesse sentido, subtende-se que os coeficientes de reflexão e de correção da terminação são insensíveis ao número de Reynolds. Tal fato é ratificado pelo fato das correlações serem congruentes entre si: 99,85 \% e 99,80 \% para $|R_{r}|$ e 95,02 \% e 94,28 \% para $l/a$. 

\begin{figure}[ht!]
\begin{subfigure}{\scaleA \textwidth}
  \input{figuras/abs_r_exaustao_015.tex}
\end{subfigure}%
\begin{subfigure}{\scaleA \textwidth}
  \input{figuras/loa_exaustao_015.tex}
\end{subfigure}
\caption[Resultados de $|R_{r}|$ e $l/a$ com escoamento de exaustão (M $=$ 0,15 e Re $=$ 2057,71)]{Resultados de $|R_{r}|$ (\ref{fig:abs_r_exaustao_015}) e $l/a$ (\ref{fig:loa_exaustao_015}) com escoamento de exaustão ($M =$ 0,15 e $Re =$ 2057,71). As linhas contínuas representam os resultados do estudo de \citeonline{munt1990acoustic} e os pontos circulares representam os resultados do modelo numérico.}
\label{fig:resultados_exaustao_015}
\end{figure}

\newpage
A Figura \ref{fig:resultados_exaustao_020} apresenta os resultados da magnitude do coeficiente de reflexão $|R_{r}|$ (\ref{fig:abs_r_exaustao_020}) e do coeficiente de correção da terminação $l/a$ (\ref{fig:loa_exaustao_020}), calculados no ponto $\textbf{P}$ na terminação do duto com escoamento de exaustão: número de Mach $M =$ 0,20 e número de Reynolds $Re =$ 5514,82.  As correlações entre os resultados foram de 98,02 \% para o coeficiente de reflexão $|R_{r}|$ (\ref{fig:abs_r_exaustao_020}) e 79,84 \% para o coeficiente de correção da terminação $l/a$ (\ref{fig:loa_exaustao_020}). Há uma divergência no gráfico \ref{fig:loa_exaustao_020} para baixas frequências e é ocasionado pelas mesmas razões abordadas nas Figuras \ref{fig:resultados_exaustao_007}, \ref{fig:resultados_exaustao_010} e \ref{fig:resultados_exaustao_015}. Também pode-se observar na Figura \ref{fig:abs_r_exaustao_020} uma divergência na região de frequências $0,5 \leq ka \leq 1,8$, isso se deve a um erro numérico inerente ao limite de estabilidade no valor de Mach $M = 0,20$, para $M > 0,20$ o modelo é totalmente instável.  

Observa-se dos resultados das Figuras \ref{fig:resultados_exaustao_007}, \ref{fig:resultados_exaustao_010}, \ref{fig:resultados_exaustao_015} e \ref{fig:resultados_exaustao_020} que $|R_{r}|$ possui um valor de pico que excede o valor unitário. Esse fenômeno ocorre, sobretudo, pela interação do escoamento com a borda do duto, a qual transforma energia cinética rotacional em energia acústica, assim como é discutido pela literatura vigente. Além disso percebe-se também que $l/a$ para baixas frequências começa num valor menor que $0,6$ e vai aumentando quando se aumenta a frequência, ou seja, a onda acústica é refletida em uma região mais próxima da abertura. Isso acontece porque o efeito de inércia provocado pela massa de fluido na saída do duto é diminuída pela presença de escoamento assim como mostra a literatura vigente.

\begin{figure}[ht!]
\begin{subfigure}{\scaleA \textwidth}
  \input{figuras/abs_r_exaustao_020.tex}
\end{subfigure}%
\begin{subfigure}{\scaleA \textwidth}
  \input{figuras/loa_exaustao_020.tex}
\end{subfigure}
\caption[Resultados de $|R_{r}|$ e $l/a$ com escoamento de exaustão (M $=$ 0,2 e Re $=$ 5514,82)]{Resultados de $|R_{r}|$ (\ref{fig:abs_r_exaustao_020}) e $l/a$ (\ref{fig:loa_exaustao_020}) com escoamento de exaustão ($M =$ 0,20 e $Re =$ 5514,82). As linhas contínuas representam os resultados do estudo de \citeonline{munt1990acoustic} e os pontos circulares representam os resultados do modelo numérico.}
\label{fig:resultados_exaustao_020}
\end{figure}

Assim como é discutida na literatura, a fenomenologia de $|R_{r}|$ abordada num contexto de exaustão está relacionada com  a frequência de desprendimento de vórtices na saída do duto. Em outras palavras, quando o número de Strouhal ($St$) atinge o valor de $\frac{\pi}{2}$, o tempo necessário para o vórtice desprendido na saída do duto propagar a distância de um diâmetro é igual ao período do campo acústico no interior do duto, causando assim o ponto máximo do coeficiente de reflexão. Tal fato é confirmado pela Figura \ref{fig:abs_r_exaustao_strouhal} que apresenta os resultados de magnitudes do coeficiente de reflexão $|R_{r}|$ em relação ao número de Strouhal ($St$), calculados no ponto $\textbf{P}$ na terminação do duto com vários escoamentos de exaustão.

Fixando o valor de Strouhal $St = \frac{\pi}{2}$ e analisando $|R_{r}|$ em relação aos números de Mach pode-se obter o comportamento do pico máximo de $|R_{r}|$ para cada número de Mach. A Figura \ref{fig:abs_r_exaustao_strouhal_mach} apresenta esse resultado, mostrando um comportamento monotônico, ou seja, $|R_{r}|$ cresce de forma regular a medida que o número de Mach aumenta. Tal fato é importante de se considerar pois para outros tipos de escoamento essa tendência pode não ocorrer.  

\newpage
\begin{figure}[ht!]
\centering
  \begin{tikzpicture}
  \begin{axis}[
  width=0.9\textwidth,
  height=0.5\textwidth,
  % x tick label style={
  %     /pgf/number format/.cd,
  %         fixed,
  %         fixed zerofill,
  %         precision=1,
  %     /tikz/.cd
  % },
  xmin=0,
  xmax=16,
  ymin=0.55,
  ymax=1.2,
  ytick distance=0.1,
  xtick distance=2,
  grid=major, % Display a grid  
  %grid style={dashed,gray!90}, % Set the style
  xlabel = $St$,
  ylabel = $|R_{r}|$,
  y tick label style={/pgf/number format/.cd,%
          scaled y ticks = false,
          set decimal separator={,},
          fixed},
      x tick label style={/pgf/number format/.cd,%
          scaled x ticks = false,
          set decimal separator={,},
          fixed}%
  ]
 \addplot[color=black, mark=x, only marks] table[x index=0,y index=1] {dados/duto_exaustao/abs_r_007_simulation_strouhal.txt};
 \addplot[color=black, mark=o, only marks] table[x index=0,y index=1] {dados/duto_exaustao/abs_r_010_simulation_strouhal.txt};
 \addplot[color=black, mark=square, only marks] table[x index=0,y index=1] {dados/duto_exaustao/abs_r_015_simulation_strouhal.txt};
 \addplot[color=black, mark=triangle, only marks] table[x index=0,y index=1] {dados/duto_exaustao/abs_r_020_simulation_strouhal.txt};

  \end{axis}
  \end{tikzpicture}
  \caption[Coeficiente de reflexão $|R_{r}|$ com escoamento de exaustão em relação ao número de Strouhal ($St$)]{Resultado de magnitudes do coeficiente de reflexão $|R_{r}|$ em relação ao número de Strouhal ($St$) calculados no ponto $\textbf{P}$ na terminação do duto com vários escoamentos de exaustão. Os pontos com $\times$, $\bigcirc$, $\square$ e $\bigtriangleup$  apresentam os resultados para os números de Mach $M =$ 0,07, $M =$ 0,10, $M =$ 0,15 e $M =$ 0,20 respectivamente.}

  \label{fig:abs_r_exaustao_strouhal}
\end{figure}

\begin{figure}[ht!]
\centering
  \begin{tikzpicture}
  \begin{axis}[
  width=0.8\textwidth,
  height=0.4\textwidth,
  x tick label style={
      /pgf/number format/.cd,
          fixed,
          fixed zerofill,
          precision=2,
      /tikz/.cd
  },
  xmin=0.07,
  xmax=0.2,
  ymin=1.0628,
  ymax=1.1289,
  ytick distance=0.01,
  xtick distance=0.03,
  grid=major, % Display a grid  
  %grid style={dashed,gray!90}, % Set the style
  xlabel = $M$,
  ylabel = $|R_{r}|$,
  y tick label style={/pgf/number format/.cd,%
          scaled y ticks = false,
          set decimal separator={,},
          fixed},
      x tick label style={/pgf/number format/.cd,%
          scaled x ticks = false,
          set decimal separator={,},
          fixed}%
  ]
 \addplot[color=black, thick] table[x index=0,y index=1] {dados/duto_exaustao/abs_r_strouhal_mach.txt};

  \end{axis}
  \end{tikzpicture}
  \caption[Coeficiente de reflexão $R_{r}$ com escoamento de exaustão em relação ao número de Mach ($M$) no Strouhal $St = \pi/2$]{Resultado de magnitudes do coeficiente de reflexão $|R_{r}|$ fixados no Strouhal $St = \pi/2$ em relação ao número de Mach ($M$) para escoamentos de exaustão.}

  \label{fig:abs_r_exaustao_strouhal_mach}
\end{figure}

\newpage
\section{Duto com Escoamento Sugado}

Visto que o modelo numérico abordado está validado e analisado num contexto sem escoamento e com escoamento de exaustão, há a necessidade de validá-lo com escoamento de succção usando os resultados na literatura desenvolvidos por \citeonline{ingard1975} e \citeonline{davies1987}. Para isso foram escolhidos os números de Mach 0,05, 0,07, 0,10, 0,15 e 0,20 para atender o requisito de  análises com escoamento succionado para $M \leq 0,2$. Além disso faz-se necessário análises do comportamento do coeficiente de reflexão em relação ao número de Strouhal visto que a natureza de fluxo de massa rotacional eclode mudanças consideráveis nessa métrica.

\begin{figure}[ht!]
\begin{subfigure}{\scaleA \textwidth}
  \begin{tikzpicture}
  \begin{axis}[
  width=\scalex \textwidth,
  height=\scaley \textwidth,
  x tick label style={
      /pgf/number format/.cd,
          fixed,
          fixed zerofill,
          precision=2,
      /tikz/.cd
  },
  xmin=0,
  xmax=0.2,
  ymin=0.68,
  ymax=1,
  ytick distance=0.05,
  xtick distance=0.05,
  grid=major, % Display a grid  
  %grid style={dashed,gray!90}, % Set the style
  xlabel = \small{Número de Mach ($M$)},
  ylabel = \small{Coeficiente de Reflexão ($R_{M}$)},
  ]
 \addplot[color=black, thick] table[x index=0,y index=1] {dados/duto_sugado/davis_analytical.txt};
 \addplot[color=black, mark=o, only marks] table[x index=0,y index=1] {dados/duto_sugado/davis_simulation.txt};
   
  \end{axis}
  \end{tikzpicture}
  \caption[Coeficiente de reflexão $R_{M}$ com escoamento sugado]{}

  \label{fig:abs_r_sugado}
\end{subfigure}%
\begin{subfigure}{\scaleA \textwidth}
  \input{figuras/loa_sugado.tex}
\end{subfigure}
\caption[Resultados de $|R_{r}|$ e $l/a$ em relação ao Mach para baixas frequências com escoamento sugado]{Resultados de $|R_{r}|$ (\ref{fig:abs_r_sugado}) e $l/a$ (\ref{fig:loa_sugado}) em relação ao Mach para $ka < 0,25$ com escoamento sugado. As linhas contínuas representam os resultados de \citeonline{davies1987}, as linhas tracejadas representam os resutados de \citeonline{ingard1975} e os pontos circulares representam os resultados para $ka = 0,1079$ calculados pelo modelo numérico.}
\label{fig:resultados_sugado}
\end{figure}


A Figura \ref{fig:resultados_sugado} apresenta os resultados da magnitude do coeficiente de reflexão $|R_{r}|$ (\ref{fig:abs_r_sugado}) e do coeficiente de correção da terminação $l/a$ (\ref{fig:loa_sugado}) em relação ao Mach para baixas frequências com escoamento sugado. Esses resultados foram calculados no ponto $\textbf{P}$ na terminação do duto com escoamento sugado com números de Reynolds $Re \leq$ 5514,82. As correlações entre os resultados foram de 98,35 \% e 95,45 \% para o coeficiente de reflexão $|R_{r}|$ (\ref{fig:abs_r_sugado}) nos resultados de \citeonline{davies1987} e \citeonline{ingard1975} respectivamente e 62,53 \% para o coeficiente de correção da terminação $l/a$ (\ref{fig:loa_sugado}). No que diz respeito a Figura \ref{fig:abs_r_sugado}, observa-se uma correlação maior com os resultados de \citeonline{davies1987} e tal fato pode ser explicado devido a esse estudo abordar dutos com seção transversal circular em vez de retangular, como aborda o estudo de \citeonline{ingard1975}. Já na Figura \ref{fig:loa_sugado} os resultados divergem consideravelmente e isso pode ser explicado pelo método de medição não considerar processos de convecção do fluido e, além disso, vale ressaltar que o resultado da litetura é uma sugestão do autor, em outras palavras, falta estudos para ratificar ou retificar o comportamento mais acurado de $l/a$ para esse contexto.

\begin{figure}[ht!]
  \centering
  \begin{tikzpicture}
  \begin{axis}[
  width=0.9\textwidth,
  height=0.5\textwidth,
  % x tick label style={
  %     /pgf/number format/.cd,
  %         fixed,
  %         fixed zerofill,
  %         precision=1,
  %     /tikz/.cd
  % },
  xmin=0,
  xmax=1.8,
  ymin=0,
  ymax=1.4,
  ytick distance=0.2,
  xtick distance=0.2,
  grid=major, % Display a grid  
  %grid style={dashed,gray!90}, % Set the style
  xlabel = $ka$,
  ylabel = $|R_{r}|$,
   y tick label style={/pgf/number format/.cd,%
          scaled y ticks = false,
          set decimal separator={,},
          fixed},
      x tick label style={/pgf/number format/.cd,%
          scaled x ticks = false,
          set decimal separator={,},
          fixed}%
  ]
 \addplot[color=black, mark=o] table[x index=0,y index=1] {dados/duto_sugado/abs_049_kp.txt};
 \addplot[color=black, mark=square] table[x index=0,y index=1] {dados/duto_sugado/abs_059_kp.txt};
 \addplot[color=black, mark=triangle] table[x index=0,y index=1] {dados/duto_sugado/abs_071_kp.txt};
\addplot[color=black, mark=x] table[x index=0,y index=1] {dados/duto_sugado/abs_074_kp.txt};
\addplot[color=black, mark=diamond] table[x index=0,y index=1] {dados/duto_sugado/abs_075_kp.txt};

  \end{axis}
  \end{tikzpicture}
  \caption[Coeficiente de reflexão $|R_{r}|$ com escoamentos sugados]{Resultados de $|R_{r}|$ em função de $ka$ para vários números de Mach de escoamento sugado. Os pontos com $\bigcirc$, $\square$, $\bigtriangleup$, $\times$ e $\diamond$  apresentam os resultados para $M =$ 0,05 e $Re =$ 1378,73, $M =$ 0,07 e $Re =$ 1930,23, $M =$ 0,1 e $Re =$ 2757,42, $M =$ 0,15 e $Re =$ 2057,71 e $M =$ 0,20 e $Re =$ 5514,82 respectivamente.}

  \label{fig:abs_r_sugado_kp}
\end{figure}

A Figura \ref{fig:abs_r_sugado_kp} apresenta resultados de magnitudes do coeficiente de reflexão $|R_{r}|$ calculados no ponto $\textbf{P}$ na terminação do duto com vários escoamentos sugados. Pode-se perceber que há uma amplificação com $|R_{r}| > 1$ para faixa de frequência $ka < 0,7$, esse fenômeno ocorre devido a interação dos vórtices na terminação do duto com o campo acústico.

A Figura \ref{fig:loa_sugado_kp} apresenta resultados do coeficiente de correção da terminação $l/a$ calculados no ponto $\textbf{P}$ na terminação do duto com vários escoamentos sugados. É possível perceber que $l/a$ possui comportamento convergente para diferentes números de Mach, somente a curva de $M = 0,2$ para $ka < 0,2$ divergiu porém é um erro numérico pois o modelo possui instabilidades nessa região de número de Mach ou maior. Tal fato corrobora com o estudo de \citeonline{davies1987} que argumenta que $l/a$ é insensível a variação do número de Mach.  

\newpage
\begin{figure}[ht!]
  \centering
  \begin{tikzpicture}
  \begin{axis}[
  width=0.9\textwidth,
  height=0.5\textwidth,
  % x tick label style={
  %     /pgf/number format/.cd,
  %         fixed,
  %         fixed zerofill,
  %         precision=1,
  %     /tikz/.cd
  % },
  xmin=0,
  xmax=1.8,
  ymin=0,
  ymax=1.4,
  ytick distance=0.2,
  xtick distance=0.2,
  grid=major, % Display a grid  
  %grid style={dashed,gray!90}, % Set the style
  xlabel = $ka$,
  ylabel = $l/a$,
  y tick label style={/pgf/number format/.cd,%
          scaled y ticks = false,
          set decimal separator={,},
          fixed},
      x tick label style={/pgf/number format/.cd,%
          scaled x ticks = false,
          set decimal separator={,},
          fixed}%
  ]
 \addplot[color=black, mark=o] table[x index=0,y index=1] {dados/duto_sugado/loa_049_kp.txt};
 \addplot[color=black, mark=square] table[x index=0,y index=1] {dados/duto_sugado/loa_059_kp.txt};
 \addplot[color=black, mark=triangle] table[x index=0,y index=1] {dados/duto_sugado/loa_071_kp.txt};
\addplot[color=black, mark=x] table[x index=0,y index=1] {dados/duto_sugado/loa_074_kp.txt};
\addplot[color=black, mark=diamond] table[x index=0,y index=1] {dados/duto_sugado/loa_075_kp.txt};

  \end{axis}
  \end{tikzpicture}
  \caption[Coeficiente de correção da terminação $l/a$ com escoamentos sugados]{Resultados de $l/a$ em função de $ka$ para diferentes números de Mach com escoamento sugado. com vários escoamentos sugados. Os pontos com $\bigcirc$, $\square$, $\bigtriangleup$, $\times$ e $\diamond$  apresentam os resultados para $M =$ 0,05 e $Re =$ 1378,73, $M =$ 0,07 e $Re =$ 1930,23, $M =$ 0,1 e $Re =$ 2757,42, $M =$ 0,15 e $Re =$ 2057,71 e $M =$ 0,20 e $Re =$ 5514,82 respectivamente.}

  \label{fig:loa_sugado_kp}
\end{figure}

\begin{figure}[ht!]
\centering
  \begin{tikzpicture}
  \begin{axis}[
  width=0.9\textwidth,
  height=0.5\textwidth,
  % x tick label style={
  %     /pgf/number format/.cd,
  %         fixed,
  %         fixed zerofill,
  %         precision=1,
  %     /tikz/.cd
  % },
  xmin=0,
  xmax=7,
  ymin=0,
  ymax=1.4,
  ytick distance=0.2,
  xtick distance=1,
  grid=major, % Display a grid  
  %grid style={dashed,gray!90}, % Set the style
  xlabel = $St$,
  ylabel = $|R_{r}|$,
   y tick label style={/pgf/number format/.cd,%
          scaled y ticks = false,
          set decimal separator={,},
          fixed},
      x tick label style={/pgf/number format/.cd,%
          scaled x ticks = false,
          set decimal separator={,},
          fixed}%
  ]
 \addplot[color=black, mark=o] table[x index=0,y index=1] {dados/duto_sugado/abs_005_strouhal.txt};
 \addplot[color=black, mark=square] table[x index=0,y index=1] {dados/duto_sugado/abs_007_strouhal.txt};
 \addplot[color=black, mark=triangle] table[x index=0,y index=1] {dados/duto_sugado/abs_010_strouhal.txt};
\addplot[color=black, mark=x] table[x index=0,y index=1] {dados/duto_sugado/abs_015_strouhal.txt};
\addplot[color=black, mark=diamond] table[x index=0,y index=1] {dados/duto_sugado/abs_020_strouhal.txt};

  \end{axis}
  \end{tikzpicture}
  \caption[Coeficiente de reflexão $|R_{r}|$ com escoamento sugado em relação ao número de Strouhal ($St$)]{Resultado de $|R_{r}|$ em relação ao número de Strouhal ($St$) com vários escoamentos sugados. Os pontos com $\bigcirc$, $\square$, $\bigtriangleup$, $\times$ e $\diamond$  apresentam os resultados para $M =$ 0,05 e $Re =$ 1378,73, $M =$ 0,07 e $Re =$ 1930,23, $M =$ 0,1 e $Re =$ 2757,42, $M =$ 0,15 e $Re =$ 2057,71 e $M =$ 0,20 e $Re =$ 5514,82 respectivamente.}

  \label{fig:abs_r_sugado_strouhal}
\end{figure}

 A Figura \ref{fig:abs_r_sugado_strouhal} apresenta resultados de magnitudes do coeficiente de reflexão $|R_{r}|$ em relação ao número de Strouhal ($St$) calculados no ponto $\textbf{P}$ na terminação do duto com vários escoamentos sugados. Pode-se perceber que os picos de amplificação de $|R_{r}|$ se alinham em $St \sim \pi/2$.

A Figura \ref{fig:abs_r_sugado_strouhal_energy} apresenta resultados do coeficiente de reflexão de energia acústica $|R_{e}|$\simbolo{$|R_{e}|$}{Coeficiente de reflexão de energia acústica} em relação ao número de Strouhal ($St$) definida por \citeonline{da2009sound} como $|R_{e}| = |R_{r}|^{2}((1 - M)(1 + M))^{2}$. Pode-se observar que há uma amplificação de $|R_{e}|$ em todos os tipos de escoamento de sucção, especificamente para $M = 0,07$ a amplificação do coeficiente de energia acústica assume o valor maior que valor unitário. Esse fato ocorre por essa amplificação estar relacionada a frequência de desprendimentos de vórtices na saída do duto, ou seja, a energia rotacional do fluido é transformada em energia acústica, pois o tempo de desprendimento de vórtices é aproximadamente igual ao período da fonte acústica. 

\begin{figure}[ht!]
\centering
  \input{figuras/energy_abs_r_sugado_strouhal.tex}
\end{figure}

Visto que o período de desprendimento de vórtices é aproximadamente igual ao período da fonte acústica e considerando a velocidade do vórtice como $\textbf{u}_{v} = \frac{\textbf{u}}{2}$\simbolo{$\textbf{u}_{v}$}{Velocidade do vórtice} segundo \citeonline{da2009sound}, a distância a ser percorrida pelo vórtice é de aproximadamente $2a/\pi$. A Figura \ref{fig:energia_turbulenta} apresenta a energia turbulenta para $M = 0,07$ calculada na camada limite ao longo do duto, começando da terminação e indo para o interior do duto. Pode-se verificar a partir dela que o pico ocorre em $(x - \Delta)/a \sim 0,65$ que é aproximadamente $2a/\pi$.

\newpage
\begin{figure}[ht!]
\centering
  \begin{tikzpicture}
  \begin{axis}[
  width=0.9\textwidth,
  height=0.5\textwidth,
  y tick label style={
      /pgf/number format/.cd,
          fixed,
          fixed zerofill,
          precision=3,
      /tikz/.cd
  },
  xmin=0,
  xmax=7,
  ymin=0,
  ymax=0.0045,
  ytick distance=0.001,
  xtick distance=1,
  grid=major, % Display a grid  
  %grid style={dashed,gray!90}, % Set the style
  xlabel = $(x - \Delta)/a$,
  ylabel = $(u')rms/Mc_{s}$,
  y tick label style={/pgf/number format/.cd,%
          scaled y ticks = false,
          set decimal separator={,},
          fixed},
      x tick label style={/pgf/number format/.cd,%
          scaled x ticks = false,
          set decimal separator={,},
          fixed}%
  ]
 \addplot[color=black, thick] table[x index=0,y index=1] {dados/duto_sugado/energia_turbulenta.txt};

  \end{axis}
  \end{tikzpicture}
  \caption[Energia turbulenta para $M = 0,07$]{Energia turbulenta calculada na camada limite para $M = 0,07$ em relação a distância da terminação para o interior interior do duto.}

  \label{fig:energia_turbulenta}
\end{figure}

Usando o colorário de Howe é possível calcular a potência acústica gerada para um período de oscilação. Para analisar a amplificação ocorrida para o Mach succionado $M = 0,07$ foi calculado a potência sonora para dois números de Strouhal: $\pi/2$ e $6,8$, que são os pontos de alta e baixa amplificação de $|R_{e}|$ respectivamente. A tabela \ref{table:potencia} mostra os resultados do procedimento citado e é possível observar que para $St = \pi/2$ há uma amplificação e para $St = 6,8$ há atenuação de energia acústica. 

\begin{table}[ht!]
\centering
\caption{Potência acústica calculada ao longo de um período de onda para $M = 0,07$ e diferentes números de Strouhal.}
\label{table:potencia}
    \begin{tabular}{|l|l|l|l|}
        \hline
        $St$ & $<P>$ \\ \hline
        $\pi/2$ & 4.6173e-06  \\ \hline  
        6,8 & -6.6410e-09 \\ \hline
    \end{tabular}
\end{table} 


A Figura \ref{fig:max_007} apresenta a energia acústica instantânea através do colorário de Howe para $M = 0,07$ e $St = \pi/2$.  
As Figuras \ref{fig:max_007_1} e \ref{fig:max_007_3} apresentam energia acústica no instante de vale e crista da onda respectivamente. É visível o vórtice com alta energia acústica se propagando e percorrendo a distância de $2a/\pi$, causando o fenômeno da amplificação. Já a Figura \ref{fig:min_007} apresenta a energia acústica instantânea através do colorário de Howe para $M = 0,07$ e $St = 6,8$. As Figuras \ref{fig:min_007_1} e \ref{fig:min_007_3} apresentam energia acústica no instante de crista e vale da onda respectivamente. Como para essa frequência não há amplificação, não há fenômeno de desprendimento de vórtices gerando energia acústica.   

\newpage
\vfill
\begin{figure}[ht!]
\begin{subfigure}{0.9 \textwidth}
  \includegraphics[width=1.\linewidth]{figuras/max_007_1.png}
  \caption[]{}
  \label{fig:max_007_1}
\end{subfigure}\par\medskip
\begin{subfigure}{0.9 \textwidth}
  \includegraphics[width=1.\linewidth]{figuras/max_007_3.png}
  \caption[]{}
  \label{fig:max_007_3}
\end{subfigure}\par\medskip
\caption[Energia acústica para $M = 0,07$ e $St = \pi/2$.]{Energia acústica no interior do duto para $M = 0,07$ e $St = \pi/2$. A Figura \ref{fig:max_007_1} apresenta energia acústica no instante de vale da onda. A Figura \ref{fig:max_007_3} apresenta energia acústica no instante de crista da onda.}\label{fig:max_007}
\end{figure}
\vfill
\clearpage

\newpage
\vfill

 \begin{figure}[ht!]
 \begin{subfigure}{0.9 \textwidth}
   \includegraphics[width=1.\linewidth]{figuras/min_007_1.png}
   \caption[]{}
   \label{fig:min_007_1}
 \end{subfigure}\par\medskip
 \begin{subfigure}{0.9 \textwidth}
   \includegraphics[width=1.\linewidth]{figuras/min_007_3.png}
   \caption[]{}
   \label{fig:min_007_3}
 \end{subfigure}\par\medskip
 \caption[Energia acústica para $M = 0,07$ e $St = 6,8$.]{Energia acústica no interior do duto para $M = 0,07$ e $St = 6,8$. A Figura \ref{fig:min_007_1} apresenta energia acústica no instante de crista da onda. A Figura \ref{fig:min_007_3} apresenta energia acústica no instante de vale da onda.}
 \label{fig:min_007}
 \end{figure}

\vfill
\clearpage

\newpage
\vfill
A Figura \ref{fig:max_007_media} apresenta a média da energia acústica para $M = 0,07$ e $St = \pi/2$ ao longo de um período de oscilação. Pode-se observar o surgimento de energia acústica causado pelos vórtices na distância de $2a/\pi$ da terminação do duto.

\begin{figure}[ht!]
\centering
  \includegraphics[width=0.62\linewidth]{figuras/max_007_media.png}
  \caption[Média da energia acústica para $M = 0,07$ e $St = \pi/2$.]{Média da energia acústica para $M = 0,07$ e $St = \pi/2$ ao longo de um período de oscilação.}
  \label{fig:max_007_media}
\end{figure}

Já a Figura \ref{fig:min_007_media} apresenta a média da energia acústica para $M = 0,07$ e $St = 6,8$ ao longo de um período de oscilação. Pode-se observar que não há amplificação e nem absorção de energia acústica pois nessa frequência o período de oscilação da onda acústica é diferente do período de desprendimento de vórtices. 

\begin{figure}[ht!]
\centering
  \includegraphics[width=0.62\linewidth]{figuras/min_007_media.png}
  \caption[Média da energia acústica para $M = 0,07$ e $St = \pi/2$.]{Média da energia acústica para $M = 0,07$ e $St = 6,8$ ao longo de um período de oscilação.}
  \label{fig:min_007_media}
\end{figure}
\vfill
\clearpage

\newpage
\begin{figure}[ht!]
\centering
  \begin{tikzpicture}
  \begin{axis}[
  width=0.9\textwidth,
  height=0.5\textwidth,
  x tick label style={
      /pgf/number format/.cd,
          fixed,
          fixed zerofill,
          precision=2,
      /tikz/.cd
  },
  xmin=0.05,
  xmax=0.2,
  ymin=0.6826,
  ymax=1.3,
  ytick distance=0.1,
  xtick distance=0.03,
  grid=major, % Display a grid  
  %grid style={dashed,gray!90}, % Set the style
  xlabel = $M$,
  ylabel = $|R_{r}|$,
   y tick label style={/pgf/number format/.cd,%
          scaled y ticks = false,
          set decimal separator={,},
          fixed},
      x tick label style={/pgf/number format/.cd,%
          scaled x ticks = false,
          set decimal separator={,},
          fixed}%
  ]
 \addplot[color=black, thick] table[x index=0,y index=1] {dados/duto_sugado/abs_r_strouhal_mach.txt};

  \end{axis}
  \end{tikzpicture}
  \caption[Coeficiente de reflexão $R_{r}$ com escoamento de exaustão em relação ao número de Mach ($M$) no Strouhal $St = \pi/2$]{Resultado de magnitudes do coeficiente de reflexão $R_{r}$ fixados no Strouhal $St \sim \pi/2$ em relação ao número de Mach ($M$) para escoamentos sugados. Os resultados foram calculados no ponto $\textbf{P}$ na terminação do duto.}

  \label{fig:abs_r_sugado_strouhal_mach}
\end{figure}

Fixando o valor de Strouhal $St = \frac{\pi}{2}$ e analisando $|R_{r}|$ em relação aos números de Mach pode-se obter o comportamento do pico máximo de $|R_{r}|$ para cada número de Mach. A Figura \ref{fig:abs_r_sugado_strouhal_mach} apresenta esse resultado, mostrando um comportamento não monotônico, ou seja, $|R_{r}|$ se comporta de forma não regular a medida que o número de Mach aumenta. Tal fato é importante de se considerar pois difere significativamente do escoamento de exaustão. Vale ressaltar também que há um máximo em $M \sim 0,07$ e a medida que se aumenta o Mach além desse valor $|R_{r}|$ vai dimiuindo.

Tal fenômeno pode ser analisado e investigado através do colorário de Howe, calculando a potência acústica gerada para $M = 0,07$ e $M = 0,1$. A tabela \ref{table:potencia_mach} mostra esse procedimento fixando o número de Strouhal em $\pi/2$. Pode-se perceber que o fenômeno da amplificação é atenuado quando o número de Mach é maior que $0,07$.    

\begin{table}[ht!]
\centering
\caption{Potência acústica calculada ao longo de um período de onda para $St = \pi/2$ e diferentes números de Mach.}
\label{table:potencia_mach}
    \begin{tabular}{|l|l|l|l|}
        \hline
        $M$ & $<P>$ \\ \hline
        $0,07$ & 4.6173e-06  \\ \hline  
        $0,1$ & 6.1312e-07 \\ \hline
    \end{tabular}
\end{table}

A Figura \ref{fig:max_01_1} apresenta a energia acústica instantânea para $M = 0,1$ e $St = \pi/2$ no instante de vale de onda. É perceptível o surgimento do vórtice de amplificação do campo acústico, assim como ocorre na Figura \ref{fig:max_007_1}.

\begin{figure}[ht!]
\centering
  \includegraphics[width=1.\linewidth]{figuras/max_01_1.png}
  \caption[Energia acústica para $M = 0,1$ e $St = \pi/2$ no instante de vale de onda.]{Energia acústica para $M = 0,1$ e $St = \pi/2$ no instante de vale de onda.}
  \label{fig:max_01_1}
\end{figure}

A Figura \ref{fig:max_01_3} apresenta a energia acústica instantânea para $M = 0,1$ e $St = \pi/2$ no instante de crista de onda. Há um fenômeno que difere do caso da Figura \ref{fig:max_01_3}, há a apresença de desprendimento de vórtice com absorção de energia acústica. Esse vórtice se desprende da terminação e alcança a distância de $2a/\pi$.

\newpage
\begin{figure}[ht!]
\centering
  \includegraphics[width=1.\linewidth]{figuras/max_01_3.png}
  \caption[Energia acústica para $M = 0,1$ e $St = \pi/2$ no instante de vale de onda.]{Energia acústica para $M = 0,1$ e $St = \pi/2$ no instante de vale de onda.}
  \label{fig:max_01_3}
\end{figure}

 Tal fato é responsável pelo declínio e não monotonocidade do gráfico da Figura \ref{fig:abs_r_sugado_strouhal_mach} e estima-se que o vórtice aumenta de energia com o aumento da velocidade de escoamento e, consequentemente, aumentando a absorção do campo acústico.   

%\newpage
%\begin{figure}[ht!]
%\centering
  %\input{figuras/loa_sugado_strouhal_mach.tex}
%\end{figure}

%valores de perda de carga:
%0.05 = 0.49
%0.07 = 0.59
%0.1 = 0.71
%0.15 = 0.74
%0.2 = 0.75
