\chapter{Resultados}

Em vista da teoria vigente na literatura e pelo que foi exposto no ponto de vista metodológico, obteve-se resultados nos seguintes contextos:
\begin{itemize}
  \item análise da condição anecóica: a partir dos históricos temporais de pressão e velocidade de partícula nas fronteiras do modelo numérico, foram feitas análises críticas de reflexão acústica;
  \item duto sem escoamento: a partir dos históricos temporais de pressão e velocidade de partícula na terminação do duto, foram feitas validações e análises críticas dos parâmetros caracterizadores da acústica interna do duto sem escoamento;
  \item duto com escoamento de exaustão: a partir dos históricos temporais de pressão e velocidade de partícula na terminação do duto, foram feitas validações e análises críticas dos parâmetros caracterizadores da acústica interna do duto com escoamento de exaustão para regimes subsônicos ($M$ $\leq$ 0,2);
  \item duto com escoamento sugado: a partir dos históricos temporais de pressão e velocidade de partícula na terminação do duto, foram feitas validações, análises críticas e investigação dos parâmetros caracterizadores da acústica interna do duto com escoamento succionado para regimes subsônicos ($M$ $\leq$ 0,2).
\end{itemize} 

\section{Análise da Condição Anecóica}

Com a finalidade de mensurar e analisar o comportamento da condição de contorno anecóica por meio de métricas numéricas e objetivas, foram calculados impedâncias e coeficientes de reflexão nas fronteiras do modelo numérico, ou seja, nos pontos \textbf{A}, \textbf{B}, \textbf{C} e \textbf{D} como é mostrado na Figura \ref{fig:modelo}. As Figuras \ref{fig:resultados_A}, \ref{fig:resultados_B}, \ref{fig:resultados_C} e \ref{fig:resultados_D} mostram os resultados nos respectivos pontos citados.


% \begin{figure}
% \begin{center}
% \begin{tikzpicture}
% \begin{axis}[
%     title={},
%     width=0.9\textwidth,
%     height=0.45\textwidth,
%     xlabel={Frequência},
%     ylabel={Sensibilidade},
%     x unit={\space Hz \space},
% 	y unit={\space C/g \space},
%     ytick=data,
%     xmin=0,
%     ymin=0,
%     ymax=0.55,
%     legend pos=north west,
%     grid=minor, % Display a grid	
% 	grid style={dashed,gray!90}, % Set the style
% 	]
%      %\addplot[color=black,dashed,thick,mark=*,mark options={solid},smooth] table[x index=2,y index=3] {Data/Kollias_SBMR_095_exp.txt}; \label{Kollias_exp095}
%      %\addlegendentry{Experimental (Kollias)}
%       %\addplot[color=blue,semithick] table[x index=0,y index=1] {Data/Kollias_SBMR_095_exp.txt}; \label{Kollias_sim095}
%     %\addlegendentry{Simulação (Kollias)}
%              \addplot[color=black, thick] table[x index=0,y index=1] {dados/coeficiente_reflexao_anecoica/A_real.txt};
%              \addplot[color=black,dashed,  thick] table[x index=0,y index=1] {dados/coeficiente_reflexao_anecoica/A_imag.txt};
%               \label{Comsol_095}% \addlegendentry{Simulação (Comsol)}
% \end{axis}
% \end{tikzpicture}
%  \caption{Resposta em carga do acelerômetro para $r$ igual a 0,95.}
% \end{center}
% \end{figure}

\newcommand\scalex{1}
\newcommand\scaley{1}
\newcommand\scaleA{0.5}

\begin{figure}
\begin{subfigure}{\scaleA \textwidth}
  % impedancia A
\begin{tikzpicture}
\begin{axis}[
width=\scalex \textwidth,
height=\scaley \textwidth,
xmin=0,
xmax=2.5,
ymin=0,
ymax=0.55,
ytick distance=0.1,
xtick distance=0.5,
grid=major, % Display a grid  
%grid style={dashed,gray!90}, % Set the style
xlabel = \small{$ka$},
ylabel = \small{$Z_{\textbf{A}}$},
]
\addplot[color=black, thick] table[x index=0,y index=1] {dados/coeficiente_reflexao_anecoica/A_real.txt};
\addplot[color=black,dashed,  thick] table[x index=0,y index=1] {dados/coeficiente_reflexao_anecoica/A_imag.txt};

\end{axis}
\end{tikzpicture}
\caption[Impedância $Z_{\textbf{A}}$]{}
\label{fig:A_impedancia}
%\caption[Impedância $Z_{\textbf{A}}$]{Resultado da impedância calculada no ponto $\textbf{A}$, próximo da condição anecóica localizada nas fronteiras do modelo numérico. A linha contínua representa a parte real e a linha tracejada representa a parte imaginária.}
\end{subfigure}%
\begin{subfigure}{\scaleA \textwidth}
  % reflexao A
\begin{tikzpicture}
\begin{axis}[
width=\scalex \textwidth,
height=\scaley \textwidth,
xmin=0,
xmax=2.5,
ymin=0,
ymax=1,
ytick distance=0.2,
xtick distance=0.5,
grid=major, % Display a grid  
%grid style={dashed,gray!90}, % Set the style
xlabel = \small{$ka$},
ylabel = \small{$|R_{\textbf{A}}|$},
]
\addplot[color=black, thick] table[x index=0,y index=1] {dados/coeficiente_reflexao_anecoica/A_abs.txt};

\end{axis}
\end{tikzpicture}
%\caption[Coeficiente de Reflexão $R_{\textbf{A}}$]{Resultado da magnitude do coeficiente de reflexão calculado no ponto $\textbf{A}$, próximo da condição anecóica localizada nas fronteiras do modelo numérico.}
\caption[Coeficiente de Reflexão $R_{\textbf{A}}$]{}
\label{fig:A_reflexao}
\end{subfigure}
\caption[Resultados de reflexão no ponto \textbf{A}]{Resultados da impedância (\ref{fig:A_impedancia}) e coeficiente de reflexão (\ref{fig:A_reflexao}) calculados no ponto $\textbf{A}$. Na Figura (\ref{fig:A_impedancia}) a linha contínua representa a parte real e a linha tracejada representa a parte imaginária.}
\label{fig:resultados_A}
\end{figure}

\newpage
A Figura \ref{fig:resultados_A} mostra os resultados de parâmetros de caracterização de reflexão no ponto \textbf{A}, ou seja, num ponto que está a frente da terminação do duto. Pode-se observar pela Figura \ref{fig:A_impedancia} que a parte real da impedância converge para o valor de $\rho_{0} c_{0}$ de referência na literatura, que em unidades do LBM possui o valor igual 0,57735. A parte imaginária da Figura \ref{fig:A_impedancia} converge num valor bem abaixo, porém ainda distante de 0 como é o ideal de uma situação totalmente anecóica. A Figura \ref{fig:A_reflexao} confirma o fato que se vê na Figura \ref{fig:A_impedancia}, ou seja, vê-se a mesma tendência do coeficiente de reflexão na parte imaginária do gráfico de impedância, ocasionando aproxidamente 15\% de reflexão. Vale ressaltar também da Figura \ref{fig:A_reflexao} que para baixas frequências há uma alta reflexão. 

A Figura \ref{fig:resultados_B} mostra os resultados de parâmetros de caracterização de reflexão no ponto \textbf{B}, ou seja, numa região de descontinuidade na direção $z$ e $y$. Pode-se observar pela Figura \ref{fig:B_impedancia} que a parte real e imaginária da impedância diverge consideravelmente de uma condição anecóica ideal. Tal fato pode ser verificado na Figura \ref{fig:B_reflexao} que, embora o coeficiente de reflexão não convirja para 1, há a presença de reflexões principalmente em baixas frequências. Mesmo o ponto \textbf{B} tendo uma tendência distoante de uma condição anecóica ideal, considera-se tal fato como desprezível visto que há poucas regiões descontínuas dessa condição de contorno no modelo numérico.   

\begin{figure}
\begin{subfigure}{\scaleA \textwidth}
  \begin{tikzpicture}
  \begin{axis}[
	width=\scalex \textwidth,
  height=\scaley \textwidth,
	xmin=0,
  xmax=2.5,
    ymin=0,
    ymax=0.9,
    ytick distance=0.2,
    xtick distance=0.5,
    grid=major, % Display a grid	
 	%grid style={dashed,gray!90}, % Set the style
	xlabel = \small{Número de Helmholtz ($ka$)},
	ylabel = \small{Impedância ($Z_{\textbf{B}}$)},
  ]
 \addplot[color=black, thick] table[x index=0,y index=1] {dados/coeficiente_reflexao_anecoica/B_real.txt};
   \addplot[color=black,dashed,  thick] table[x index=0,y index=1] {dados/coeficiente_reflexao_anecoica/B_imag.txt};
   
  \end{axis}
  \end{tikzpicture}
  \caption[Impedância $Z_{\textbf{B}}$]{}
  \label{fig:B_impedancia}
  %\caption[Impedância $Z_{\textbf{B}}$]{Resultado da impedância calculada no ponto $\textbf{B}$, próximo da condição anecóica localizada nas fronteiras do modelo numérico. A linha contínua representa a parte real e a linha tracejada representa a parte imaginária.}
  
\end{subfigure}%
\begin{subfigure}{\scaleA \textwidth}
  \begin{tikzpicture}
  \begin{axis}[
	width=\scalex \textwidth,
  height=\scaley \textwidth,
	xmin=0,
  xmax=2.5,
    ymin=0,
    ymax=1,
    ytick distance=0.2,
    xtick distance=0.5,
    grid=major, % Display a grid	
 	%grid style={dashed,gray!90}, % Set the style
	xlabel = \small{$ka$},
	ylabel = \small{$|R_{\textbf{B}}|$},
  y tick label style={/pgf/number format/.cd,%
          scaled y ticks = false,
          set decimal separator={,},
          fixed},
      x tick label style={/pgf/number format/.cd,%
          scaled x ticks = false,
          set decimal separator={,},
          fixed}%
  ]
 \addplot[color=black, thick] table[x index=0,y index=1] {dados/coeficiente_reflexao_anecoica/B_abs.txt};
   
  \end{axis}
  \end{tikzpicture}
  \caption[Coeficiente de Reflexão $R_{\textbf{B}}$]{}
  \label{fig:B_reflexao}
  %\caption[Coeficiente de Reflexão $R_{\textbf{B}}$]{Resultado da magnitude do coeficiente de reflexão calculado no ponto $\textbf{B}$, próximo da condição anecóica localizada nas fronteiras do modelo numérico.}
\end{subfigure}
\caption[Resultados de reflexão no ponto \textbf{B}]{Resultados da impedância (\ref{fig:B_impedancia}) e coeficiente de reflexão (\ref{fig:B_reflexao}) calculados no ponto $\textbf{B}$. Na Figura (\ref{fig:B_impedancia}) a linha contínua representa a parte real e a linha tracejada representa a parte imaginária.}
\label{fig:resultados_B}
\end{figure}

\newpage
A Figura \ref{fig:resultados_C} mostra os resultados de parâmetros de caracterização de reflexão no ponto \textbf{C}. Pode-se observar pela Figura \ref{fig:C_impedancia} que a parte real da impedância converge, para baixas e médias frequências, para o valor de $\rho_{0} c_{0}$ de referência na literatura, que em unidades do LBM possui o valor igual 0,57735. A parte imaginária da Figura \ref{fig:C_impedancia} converge num valor bem abaixo para baixas e médias frequências, porém ainda distante de 0 como é o ideal de uma situação totalmente anecóica. A Figura \ref{fig:C_reflexao} confirma o fato que se vê na Figura \ref{fig:C_impedancia}, ou seja, vê-se a mesma tendência do coeficiente de reflexão na parte imaginária do gráfico de impedância, ocasionando aproxidamente 15\% de reflexão.

A Figura \ref{fig:resultados_D} mostra os resultados de parâmetros de caracterização de reflexão no ponto \textbf{D}, numa posição atrás da terminação duto. Essa região, quando não está corretamente ajustada no que diz respeito a condição anecóica, pode-se caracterizar como parede rígida refletora e distorcer o comportamento para um contexto de duto extendido a partir de uma parede rígida como mostra o estudo de \citeonline{selamet2001wave}. Contudo, tal região se comporta como aproximadamente anecóica pois possui características congruentes ao ponto \textbf{C} no que diz respeito a impedância (Figura \ref{fig:D_impedancia}) e ao coeficiente de reflexão (Figura \ref{fig:D_reflexao}).

\begin{figure}
\begin{subfigure}{\scaleA \textwidth}
  \begin{tikzpicture}
  \begin{axis}[
	width=\scalex \textwidth,
  height=\scaley \textwidth,
	xmin=0,
  xmax=2.5,
    ymin=0,
    ymax=0.7,
    ytick distance=0.1,
    xtick distance=0.5,
    grid=major, % Display a grid	
 	%grid style={dashed,gray!90}, % Set the style
	xlabel = \small{$ka$},
	ylabel = \small{$Z_{\textbf{C}}$},
   y tick label style={/pgf/number format/.cd,%
          scaled y ticks = false,
          set decimal separator={,},
          fixed},
      x tick label style={/pgf/number format/.cd,%
          scaled x ticks = false,
          set decimal separator={,},
          fixed}%
  ]
 \addplot[color=black, thick] table[x index=0,y index=1] {dados/coeficiente_reflexao_anecoica/C_real.txt};
   \addplot[color=black,dashed,  thick] table[x index=0,y index=1] {dados/coeficiente_reflexao_anecoica/C_imag.txt};
   
  \end{axis}
  \end{tikzpicture}
  \caption[Impedância $Z_{\textbf{C}}$]{}
  \label{fig:C_impedancia}
  %\caption[Impedância $Z_{\textbf{C}}$]{Resultado da impedância calculada no ponto $\textbf{C}$, próximo da condição anecóica localizada nas fronteiras do modelo numérico. A linha contínua representa a parte real e a linha tracejada representa a parte imaginária.}

\end{subfigure}%
\begin{subfigure}{\scaleA \textwidth}
  \begin{tikzpicture}
  \begin{axis}[
	width=\scalexA \textwidth,
  height=\scaleyA \textwidth,
	xmin=0,
  xmax=2.5,
    ymin=0,
    ymax=1,
    ytick distance=0.2,
    xtick distance=0.5,
    grid=major, % Display a grid	
 	%grid style={dashed,gray!90}, % Set the style
	xlabel = \small{$ka$},
	ylabel = \small{$|R_{\textbf{C}}|$},
  y tick label style={/pgf/number format/.cd,%
          scaled y ticks = false,
          set decimal separator={,},
          fixed},
      x tick label style={/pgf/number format/.cd,%
          scaled x ticks = false,
          set decimal separator={,},
          fixed}%
  ]
 \addplot[color=black, thick] table[x index=0,y index=1] {dados/coeficiente_reflexao_anecoica/C_abs.txt};
   
  \end{axis}
  \end{tikzpicture}
  \caption[Coeficiente de Reflexão $R_{\textbf{C}}$]{}
  \label{fig:C_reflexao}
  %\caption[Coeficiente de Reflexão $R_{\textbf{C}}$]{Resultado da magnitude do coeficiente de reflexão calculado no ponto $\textbf{C}$, próximo da condição anecóica localizada nas fronteiras do modelo numérico.}

\end{subfigure}
\caption[Resultados de reflexão no ponto \textbf{C}]{Resultados da impedância (\ref{fig:C_impedancia}) e coeficiente de reflexão (\ref{fig:C_reflexao}) calculados no ponto $\textbf{C}$. Na Figura (\ref{fig:C_impedancia}) a linha contínua representa a parte real e a linha tracejada representa a parte imaginária.}
\label{fig:resultados_C}
\end{figure}

\begin{figure}
\begin{subfigure}{\scaleA \textwidth}
  \begin{tikzpicture}
  \begin{axis}[
	width=\scalexA \textwidth,
  height=\scaleyA \textwidth,
	xmin=0,
  xmax=2.5,
    ymin=0,
    ymax=0.62,
    ytick distance=0.1,
    xtick distance=0.5,
    grid=major, % Display a grid	
 	%grid style={dashed,gray!90}, % Set the style
	xlabel = \small{$ka$},
	ylabel = \small{$Z_{\textbf{D}}$},
   y tick label style={/pgf/number format/.cd,%
          scaled y ticks = false,
          set decimal separator={,},
          fixed},
      x tick label style={/pgf/number format/.cd,%
          scaled x ticks = false,
          set decimal separator={,},
          fixed}%
  ]
 \addplot[color=black, thick] table[x index=0,y index=1] {dados/coeficiente_reflexao_anecoica/D_real.txt};
   \addplot[color=black,dashed,  thick] table[x index=0,y index=1] {dados/coeficiente_reflexao_anecoica/D_imag.txt};
   
  \end{axis}
  \end{tikzpicture}
\caption[Impedância $Z_{\textbf{D}}$]{}
\label{fig:D_impedancia}
%\caption[Impedância $Z_{\textbf{D}}$]{Resultado da impedância calculada no ponto $\textbf{D}$, próximo da condição anecóica localizada nas fronteiras do modelo numérico. A linha contínua representa a parte real e a linha tracejada representa a parte imaginária.}
\end{subfigure}%
\begin{subfigure}{\scaleA \textwidth}
  \begin{tikzpicture}
  \begin{axis}[
	width=\scalex \textwidth,
  height=\scaley \textwidth,
	xmin=0,
  xmax=2.5,
    ymin=0,
    ymax=1,
    ytick distance=0.2,
    xtick distance=0.5,
    grid=major, % Display a grid	
 	%grid style={dashed,gray!90}, % Set the style
	xlabel = \small{$ka$},
	ylabel = \small{$|R_{\textbf{D}}|$},
  ]
 \addplot[color=black, thick] table[x index=0,y index=1] {dados/coeficiente_reflexao_anecoica/D_abs.txt};
   
  \end{axis}
  \end{tikzpicture}
 \caption[Coeficiente de Reflexão $R_{\textbf{D}}$]{}
  \label{fig:D_reflexao}
  %\caption[Coeficiente de Reflexão $R_{\textbf{D}}$]{Resultado da magnitude do coeficiente de reflexão calculado no ponto $\textbf{D}$, próximo da condição anecóica localizada nas fronteiras do modelo numérico.}

\end{subfigure}
\caption[Resultados de reflexão no ponto \textbf{D}]{Resultados da impedância (\ref{fig:D_impedancia}) e coeficiente de reflexão (\ref{fig:D_reflexao}) calculados no ponto $\textbf{D}$. Na Figura (\ref{fig:D_impedancia}) a linha contínua representa a parte real e a linha tracejada representa a parte imaginária.}
\label{fig:resultados_D}
\end{figure}

\newpage
Em vista do que foi exposto através de uma análise de reflexão acústica nos pontos \textbf{A}, \textbf{B}, \textbf{C} e \textbf{D}, a condição de contorno na fronteira do domínio pode ser considerada como aproximadamente anecóica.

\section{Duto sem Escoamento}

\begin{figure}[ht!]
\begin{subfigure}{\scaleA \textwidth}
  \begin{tikzpicture}
  \begin{axis}[
  width=\scalex \textwidth,
  height=\scaley \textwidth,
  xmin=0,
  xmax=1.8,
  ymin=0.3,
  ymax=1,
  ytick distance=0.1,
  grid=major, % Display a grid  
  %grid style={dashed,gray!90}, % Set the style
  xlabel = \small{$ka$},
  ylabel = \small{$|R_{r}|$},
   y tick label style={/pgf/number format/.cd,%
          scaled y ticks = false,
          set decimal separator={,},
          fixed},
      x tick label style={/pgf/number format/.cd,%
          scaled x ticks = false,
          set decimal separator={,},
          fixed}%
  ]
 \addplot[color=black, thick] table[x index=0,y index=1] {dados/duto_sem_escoamento/analytical_data_abs_r.txt};
 \addplot[color=black, mark=o, only marks] table[x index=0,y index=1] {dados/duto_sem_escoamento/simulation_data_abs_r.txt};
   
  \end{axis}
  \end{tikzpicture}
  \caption[Coeficiente de Reflexão $R_{r}$ sem Escoamento]{}
  \label{fig:abs_r_sem_escoamento}
\end{subfigure}%
\begin{subfigure}{\scaleA \textwidth}
  \begin{tikzpicture}
  \begin{axis}[
  width=\scalex \textwidth,
  height=\scaley \textwidth,
  xmin=0,
  xmax=1.8,
    ymin=0,
    ymax=0.8,
    ytick distance=0.1,
    grid=major, % Display a grid  
  %grid style={dashed,gray!90}, % Set the style
  xlabel = \small{$ka$},
  ylabel = \small{$l/a$},
   y tick label style={/pgf/number format/.cd,%
          scaled y ticks = false,
          set decimal separator={,},
          fixed},
      x tick label style={/pgf/number format/.cd,%
          scaled x ticks = false,
          set decimal separator={,},
          fixed}%
  ]
 \addplot[color=black, thick] table[x index=0,y index=1] {dados/duto_sem_escoamento/analytical_data_loa.txt};
 \addplot[color=black, mark=o, only marks] table[x index=0,y index=1] {dados/duto_sem_escoamento/simulation_data_loa.txt};
   
  \end{axis}
  \end{tikzpicture}
  \caption[Coeficiente de Correção da Terminação $l/a$ sem Escoamento]{}
  \label{fig:loa_sem_escoamento}
\end{subfigure}
\caption[Resultados de $R_{r}$ e $l/a$ sem escoamento]{Resultados da magnitude do coeficiente de reflexão $R_{r}$ (\ref{fig:abs_r_sem_escoamento}) e do coeficiente de correção da terminação $l/a$ (\ref{fig:loa_sem_escoamento}) calculados no ponto $\textbf{P}$ na terminação do duto sem escoamento. As linhas contínuas representam os resultado analítico do estudo de \citeonline{levine1948radiation} e os pontos circulares representam os resultados calculados pela ferramenta computacional proposta nesse estudo. As correlações entre os resultados foram de 99,95 \% para o coeficiente de reflexão $R_{r}$ (\ref{fig:abs_r_sem_escoamento}) e 96,23 \% para o coeficiente de correção da terminação $l/a$ (\ref{fig:loa_sem_escoamento}).}
\label{fig:resultados_sem_escoamento}
\end{figure}


\newpage
\section{Duto com Escoamento de Exaustão}

\subsection{Mach = 0,07}

\begin{figure}[ht!]
\begin{subfigure}{\scaleA \textwidth}
  \begin{tikzpicture}
  \begin{axis}[
  width=\scalex \textwidth,
  height=\scaley \textwidth,
  xmin=0,
  xmax=1.8,
    ymin=0.4,
    ymax=1.2,
    ytick distance=0.1,
    grid=major, % Display a grid  
  %grid style={dashed,gray!90}, % Set the style
  xlabel = \small{Número de Helmholtz ($ka$)},
  ylabel = \small{Coeficiente de Reflexão ($R_{r}$)},
  ]
 \addplot[color=black, thick] table[x index=0,y index=1] {dados/duto_exaustao/abs_r_007_analytical.txt};
 \addplot[color=black, mark=o, only marks] table[x index=0,y index=1] {dados/duto_exaustao/abs_r_007_simulation.txt};
   
  \end{axis}
  \end{tikzpicture}
  \caption[Coeficiente de Reflexão $R_{r}$ com Escoamento de Exaustão (M $=$ 0,07)]{}
  \label{fig:abs_r_exaustao_007}
\end{subfigure}%
\begin{subfigure}{\scaleA \textwidth}
  \begin{tikzpicture}
  \begin{axis}[
   width=\scalex \textwidth,
  height=\scaley \textwidth,
  xmin=0,
  xmax=1.8,
    ymin=0,
    ymax=0.9,
    ytick distance=0.1,
    grid=major, % Display a grid  
  %grid style={dashed,gray!90}, % Set the style
  xlabel = \small{$ka$},
  ylabel = \small{$l/a$},
   y tick label style={/pgf/number format/.cd,%
          scaled y ticks = false,
          set decimal separator={,},
          fixed},
      x tick label style={/pgf/number format/.cd,%
          scaled x ticks = false,
          set decimal separator={,},
          fixed}%
  ]
 \addplot[color=black, thick] table[x index=0,y index=1] {dados/duto_exaustao/loa_007_analytical.txt};
 \addplot[color=black, mark=o, only marks] table[x index=0,y index=1] {dados/duto_exaustao/loa_007_simulation.txt};
   
  \end{axis}
  \end{tikzpicture}
  \caption[Coeficiente de Correção da Terminação $l/a$ com Escoamento de Exaustão (M $=$ 0,07)]{}
  \label{fig:loa_exaustao_007}
\end{subfigure}
\caption[Resultados de $R_{r}$ e $l/a$ com escoamento de exaustão ($M =$ 0,07 e $Re =$ 1930,23)]{Resultados da magnitude do coeficiente de reflexão $R_{r}$ (\ref{fig:abs_r_exaustao_007}) e do coeficiente de correção da terminação $l/a$ (\ref{fig:loa_exaustao_007}) calculados no ponto $\textbf{P}$ na terminação do duto com escoamento de exaustão ($M =$ 0,07 e $Re =$ 1930,23). As linhas contínuas representam os resultados analíticos do estudo de \citeonline{munt1990acoustic} e os pontos circulares representam os resultados calculados pela ferramenta computacional proposta nesse estudo. As correlações entre os resultados foram de 99,83 \% para o coeficiente de reflexão $R_{r}$ (\ref{fig:abs_r_exaustao_007}) e 87,76 \% para o coeficiente de correção da terminação $l/a$ (\ref{fig:loa_exaustao_007}).}
\label{fig:resultados_exaustao_007}
\end{figure}

\newpage
\subsection{Mach = 0,10}
\begin{figure}[ht!]
\begin{subfigure}{\scaleA \textwidth}
  \begin{tikzpicture}
  \begin{axis}[
 width=\scalex \textwidth,
  height=\scaley \textwidth,
  xmin=0,
  xmax=1.8,
    ymin=0.4,
    ymax=1.2,
    ytick distance=0.1,
    grid=major, % Display a grid  
  %grid style={dashed,gray!90}, % Set the style
  xlabel = \small{Número de Helmholtz ($ka$)},
  ylabel = \small{Coeficiente de Reflexão ($R_{r}$)},
  ]
 \addplot[color=black, thick] table[x index=0,y index=1] {dados/duto_exaustao/abs_r_010_analytical.txt};
 \addplot[color=black, mark=o, only marks] table[x index=0,y index=1] {dados/duto_exaustao/abs_r_010_simulation.txt};
   
  \end{axis}
  \end{tikzpicture}
  \caption[Coeficiente de Reflexão $R_{r}$ com Escoamento de Exaustão (M $=$ 0,10)]{}
  \label{fig:abs_r_exaustao_010}
\end{subfigure}%
\begin{subfigure}{\scaleA \textwidth}
  \begin{tikzpicture}
  \begin{axis}[
  width=\scalex \textwidth,
  height=\scaley \textwidth,
  xmin=0,
  xmax=1.8,
    ymin=0.15,
    ymax=0.6,
    ytick distance=0.05,
    grid=major, % Display a grid  
  %grid style={dashed,gray!90}, % Set the style
  xlabel = \small{$ka$},
  ylabel = \small{$l/a$},
  ]
 \addplot[color=black, thick] table[x index=0,y index=1] {dados/duto_exaustao/loa_010_analytical.txt};
 \addplot[color=black, mark=o, only marks] table[x index=0,y index=1] {dados/duto_exaustao/loa_010_simulation.txt};
   
  \end{axis}
  \end{tikzpicture}
  \caption[Coeficiente de Correção da Terminação $l/a$ com Escoamento de Exaustão (M $=$ 0,10)]{}
  \label{fig:loa_exaustao_010}
\end{subfigure}
\caption[Resultados de $R_{r}$ e $l/a$ com escoamento de exaustão ($M =$ 0,10 e $Re =$ 2757,42)]{Resultados da magnitude do coeficiente de reflexão $R_{r}$ (\ref{fig:abs_r_exaustao_010}) e do coeficiente de correção da terminação $l/a$ (\ref{fig:loa_exaustao_010}) calculados no ponto $\textbf{P}$ na terminação do duto com escoamento de exaustão ($M =$ 0,10 e $Re =$ 2757,42). As linhas contínuas representam os resultados analíticos do estudo de \citeonline{munt1990acoustic} e os pontos circulares representam os resultados calculados pela ferramenta computacional proposta nesse estudo. As correlações entre os resultados foram de 99,85 \% para o coeficiente de reflexão $R_{r}$ (\ref{fig:abs_r_exaustao_010}) e 95,02 \% para o coeficiente de correção da terminação $l/a$ (\ref{fig:loa_exaustao_010}).}
\label{fig:resultados_exaustao_010}
\end{figure}


\newpage
\subsection{Mach = 0,15}

\begin{figure}[ht!]
\begin{subfigure}{\scaleA \textwidth}
  \begin{tikzpicture}
  \begin{axis}[
  width=\scalex \textwidth,
  height=\scaley \textwidth,
  xmin=0,
  xmax=1.8,
    ymin=0.4,
    ymax=1.2,
    ytick distance=0.1,
    grid=major, % Display a grid  
  %grid style={dashed,gray!90}, % Set the style
  xlabel = \small{$ka$},
  ylabel = \small{$|R_{r}|$},
  ]
 \addplot[color=black, thick] table[x index=0,y index=1] {dados/duto_exaustao/abs_r_015_analytical.txt};
 \addplot[color=black, mark=o, only marks] table[x index=0,y index=1] {dados/duto_exaustao/abs_r_015_simulation.txt};
   
  \end{axis}
  \end{tikzpicture}
  \caption[Coeficiente de Reflexão $R_{r}$ com Escoamento de Exaustão (M $=$ 0,15)]{}
  \label{fig:abs_r_exaustao_015}
\end{subfigure}%
\begin{subfigure}{\scaleA \textwidth}
  \begin{tikzpicture}
  \begin{axis}[
  width=\scalex \textwidth,
  height=\scaley \textwidth,
  xmin=0,
  xmax=1.8,
    ymin=0.15,
    ymax=0.55,
    ytick distance=0.05,
    grid=major, % Display a grid  
  %grid style={dashed,gray!90}, % Set the style
  xlabel = \small{Número de Helmholtz ($ka$)},
  ylabel = \small{Correção da Terminação ($l/a$)},
  ]
 \addplot[color=black, thick] table[x index=0,y index=1] {dados/duto_exaustao/loa_015_analytical.txt};
 \addplot[color=black, mark=o, only marks] table[x index=0,y index=1] {dados/duto_exaustao/loa_015_simulation.txt};
   
  \end{axis}
  \end{tikzpicture}
  \caption[Coeficiente de Correção da Terminação $l/a$ com Escoamento de Exaustão (M $=$ 0,15)]{}
  \label{fig:loa_exaustao_015}
\end{subfigure}
\caption[Resultados de $R_{r}$ e $l/a$ com escoamento de exaustão (M $=$ 0,15 e Re $=$ 2057,71)]{Resultados da magnitude do coeficiente de reflexão $R_{r}$ (\ref{fig:abs_r_exaustao_015}) e do coeficiente de correção da terminação $l/a$ (\ref{fig:loa_exaustao_015}) calculados no ponto $\textbf{P}$ na terminação do duto com escoamento de exaustão (M $=$ 0,15 e Re $=$ 2057,71). As linhas contínuas representam os resultados analíticos do estudo de \citeonline{munt1990acoustic} e os pontos circulares representam os resultados calculados pela ferramenta computacional proposta nesse estudo. As correlações entre os resultados foram de 99,80 \% para o coeficiente de reflexão $R_{r}$ (\ref{fig:abs_r_exaustao_015}) e 94,28 \% para o coeficiente de correção da terminação $l/a$ (\ref{fig:loa_exaustao_015}).}
\label{fig:resultados_exaustao_015}
\end{figure}


\newpage
\subsection{Mach 0,2}
\begin{figure}[ht!]
\begin{subfigure}{\scaleA \textwidth}
  \begin{tikzpicture}
  \begin{axis}[
  width=\scalex \textwidth,
  height=\scaley \textwidth,
  xmin=0,
  xmax=1.8,
    ymin=0.4,
    ymax=1.2,
    ytick distance=0.1,
    grid=major, % Display a grid  
  %grid style={dashed,gray!90}, % Set the style
  xlabel = \small{$ka$},
  ylabel = \small{$|R_{r}|$},
  ]
 \addplot[color=black, thick] table[x index=0,y index=1] {dados/duto_exaustao/abs_r_020_analytical.txt};
 \addplot[color=black, mark=o, only marks] table[x index=0,y index=1] {dados/duto_exaustao/abs_r_020_simulation.txt};
   
  \end{axis}
  \end{tikzpicture}
  \caption[Coeficiente de Reflexão $R_{r}$ com Escoamento de Exaustão (M $=$ 0,2)]{}
  \label{fig:abs_r_exaustao_020}
\end{subfigure}%
\begin{subfigure}{\scaleA \textwidth}
  \begin{tikzpicture}
  \begin{axis}[
  width=\scalex \textwidth,
  height=\scaley \textwidth,
  xmin=0,
  xmax=1.8,
    ymin=0,
    ymax=0.9,
    ytick distance=0.1,
    grid=major, % Display a grid  
  %grid style={dashed,gray!90}, % Set the style
  xlabel = \small{$ka$},
  ylabel = \small{$l/a$},
   y tick label style={/pgf/number format/.cd,%
          scaled y ticks = false,
          set decimal separator={,},
          fixed},
      x tick label style={/pgf/number format/.cd,%
          scaled x ticks = false,
          set decimal separator={,},
          fixed}%
  ]
 \addplot[color=black, thick] table[x index=0,y index=1] {dados/duto_exaustao/loa_020_analytical.txt};
 \addplot[color=black, mark=o, only marks] table[x index=0,y index=1] {dados/duto_exaustao/loa_020_simulation.txt};
   
  \end{axis}
  \end{tikzpicture}
  \caption[Coeficiente de Correção da Terminação $l/a$ com Escoamento de Exaustão (M $=$ 0,2)]{}
  \label{fig:loa_exaustao_020}
\end{subfigure}
\caption[Resultados de $R_{r}$ e $l/a$ com escoamento de exaustão (M $=$ 0,2 e Re $=$ 5514,82)]{Resultados da magnitude do coeficiente de reflexão $R_{r}$ (\ref{fig:abs_r_exaustao_020}) e do coeficiente de correção da terminação $l/a$ (\ref{fig:loa_exaustao_020}) calculados no ponto $\textbf{P}$ na terminação do duto com escoamento de exaustão (M $=$ 0,2 e Re $=$ 5514,82). As linhas contínuas representam os resultados analíticos do estudo de \citeonline{munt1990acoustic} e os pontos circulares representam os resultados calculados pela ferramenta computacional proposta nesse estudo. As correlações entre os resultados foram de 99,80 \% para o coeficiente de reflexão $R_{r}$ (\ref{fig:abs_r_exaustao_020}) e 79,84 \% para o coeficiente de correção da terminação $l/a$ (\ref{fig:loa_exaustao_020}).}
\label{fig:resultados_exaustao_020}
\end{figure}

\newpage
\begin{figure}[ht!]
\centering
  \begin{tikzpicture}
  \begin{axis}[
  width=0.9\textwidth,
  height=0.5\textwidth,
  % x tick label style={
  %     /pgf/number format/.cd,
  %         fixed,
  %         fixed zerofill,
  %         precision=1,
  %     /tikz/.cd
  % },
  xmin=0,
  xmax=16,
  ymin=0.55,
  ymax=1.2,
  ytick distance=0.1,
  xtick distance=2,
  grid=major, % Display a grid  
  %grid style={dashed,gray!90}, % Set the style
  xlabel = $St$,
  ylabel = $|R_{r}|$,
  y tick label style={/pgf/number format/.cd,%
          scaled y ticks = false,
          set decimal separator={,},
          fixed},
      x tick label style={/pgf/number format/.cd,%
          scaled x ticks = false,
          set decimal separator={,},
          fixed}%
  ]
 \addplot[color=black, mark=x, only marks] table[x index=0,y index=1] {dados/duto_exaustao/abs_r_007_simulation_strouhal.txt};
 \addplot[color=black, mark=o, only marks] table[x index=0,y index=1] {dados/duto_exaustao/abs_r_010_simulation_strouhal.txt};
 \addplot[color=black, mark=square, only marks] table[x index=0,y index=1] {dados/duto_exaustao/abs_r_015_simulation_strouhal.txt};
 \addplot[color=black, mark=triangle, only marks] table[x index=0,y index=1] {dados/duto_exaustao/abs_r_020_simulation_strouhal.txt};

  \end{axis}
  \end{tikzpicture}
  \caption[Coeficiente de reflexão $|R_{r}|$ com escoamento de exaustão em relação ao número de Strouhal ($St$)]{Resultado de magnitudes do coeficiente de reflexão $|R_{r}|$ em relação ao número de Strouhal ($St$) calculados no ponto $\textbf{P}$ na terminação do duto com vários escoamentos de exaustão. Os pontos com $\times$, $\bigcirc$, $\square$ e $\bigtriangleup$  apresentam os resultados para os números de Mach $M =$ 0,07, $M =$ 0,10, $M =$ 0,15 e $M =$ 0,20 respectivamente.}

  \label{fig:abs_r_exaustao_strouhal}
\end{figure}

\newpage
\begin{figure}[ht!]
\centering
  \begin{tikzpicture}
  \begin{axis}[
  width=0.9\textwidth,
  height=0.5\textwidth,
  x tick label style={
      /pgf/number format/.cd,
          fixed,
          fixed zerofill,
          precision=2,
      /tikz/.cd
  },
  xmin=0.07,
  xmax=0.2,
  ymin=1.0628,
  ymax=1.1289,
  ytick distance=0.01,
  xtick distance=0.03,
  grid=major, % Display a grid  
  %grid style={dashed,gray!90}, % Set the style
  xlabel = $M$,
  ylabel = $|R_{r}|$,
  ]
 \addplot[color=black, thick] table[x index=0,y index=1] {dados/duto_exaustao/abs_r_strouhal_mach.txt};

  \end{axis}
  \end{tikzpicture}
  \caption[Coeficiente de reflexão $R_{r}$ com escoamento de exaustão em relação ao número de Mach ($M$) no Strouhal $St = \pi/2$]{Resultado de magnitudes do coeficiente de reflexão $|R_{r}|$ fixados no Strouhal $St = \pi/2$ em relação ao número de Mach ($M$) para escoamentos de exaustão. Os resultados foram calculados no ponto $\textbf{P}$ na terminação do duto.}

  \label{fig:abs_r_exaustao_strouhal_mach}
\end{figure}



\newpage
\section{Duto com Escoamento Sugado}
\begin{figure}[ht!]
\begin{subfigure}{\scaleA \textwidth}
  \begin{tikzpicture}
  \begin{axis}[
  width=\scalex \textwidth,
  height=\scaley \textwidth,
  x tick label style={
      /pgf/number format/.cd,
          fixed,
          fixed zerofill,
          precision=2,
      /tikz/.cd
  },
  xmin=0,
  xmax=0.2,
  ymin=0.68,
  ymax=1,
  ytick distance=0.05,
  xtick distance=0.05,
  grid=major, % Display a grid  
  %grid style={dashed,gray!90}, % Set the style
  xlabel = \small{$M$},
  ylabel = \small{$|R_{M}|$},
  ]
 \addplot[color=black, thick] table[x index=0,y index=1] {dados/duto_sugado/davis_analytical.txt};
 \addplot[color=black, mark=o, only marks] table[x index=0,y index=1] {dados/duto_sugado/davis_simulation.txt};
   
  \end{axis}
  \end{tikzpicture}
  \caption[Coeficiente de reflexão $R_{M}$ com escoamento sugado]{}

  \label{fig:abs_r_sugado}
\end{subfigure}%
\begin{subfigure}{\scaleA \textwidth}
  \begin{tikzpicture}
  \begin{axis}[
  width=\scalex \textwidth,
  height=\scaley \textwidth,
  x tick label style={
      /pgf/number format/.cd,
          fixed,
          fixed zerofill,
          precision=2,
      /tikz/.cd
  },
  xmin=0,
  xmax=0.2,
  ymin=0.1,
  ymax=1.0,
  ytick distance=0.1,
  xtick distance=0.05,
  grid=major, % Display a grid  
  %grid style={dashed,gray!90}, % Set the style
  xlabel = \small{$M$},
  ylabel = \small{$l_{M}$},
  ]
 \addplot[color=black, thick] table[x index=0,y index=1] {dados/duto_sugado/davis_analytical_loa.txt};
 \addplot[color=black, mark=o, only marks] table[x index=0,y index=1] {dados/duto_sugado/davis_simulation_loa.txt};
   
  \end{axis}
  \end{tikzpicture}
  \caption[Coeficiente de correção da terminação ($l_{M}$) com escoamento sugado]{}

  \label{fig:loa_sugado}
\end{subfigure}
\caption[Resultados de $R_{M}$ e $l/a$ em relação ao Mach para baixas frequências ($ka$ $<$ $0,25$) com escoamento sugado]{Resultados da magnitude do coeficiente de reflexão $R_{M}$ (\ref{fig:abs_r_sugado}) e do coeficiente de correção da terminação $l/a$ (\ref{fig:loa_sugado}) em  em relação ao Mach para baixas frequências ($ka$ $<$ $0,25$) com escoamento sugado. Esses resultados foram calculados no ponto $\textbf{P}$ na terminação do duto com escoamento sugado ($Re \leq$ 5514,82). As linhas contínuas representam os resultados analíticos do estudo de \citeonline{davies1987} e os pontos circulares representam os resultados calculados pela ferramenta computacional proposta nesse estudo. As correlações entre os resultados foram de 95,45 \% para o coeficiente de reflexão $R_{M}$ (\ref{fig:abs_r_sugado}) e 62,53 \% para o coeficiente de correção da terminação $l/a$ (\ref{fig:loa_sugado}).}
\label{fig:resultados_sugado}
\end{figure}

\newpage
\begin{figure}[ht!]
\centering
  \begin{tikzpicture}
  \begin{axis}[
  width=0.9\textwidth,
  height=0.5\textwidth,
  % x tick label style={
  %     /pgf/number format/.cd,
  %         fixed,
  %         fixed zerofill,
  %         precision=1,
  %     /tikz/.cd
  % },
  xmin=0,
  xmax=7,
  ymin=0,
  ymax=1.4,
  ytick distance=0.2,
  xtick distance=1,
  grid=major, % Display a grid  
  %grid style={dashed,gray!90}, % Set the style
  xlabel = Número de Strouhal ($St$),
  ylabel = Coeficiente de Reflexão ($R_{r}$),
  ]
 \addplot[color=black, mark=o] table[x index=0,y index=1] {dados/duto_sugado/abs_005_strouhal.txt};
 \addplot[color=black, mark=square] table[x index=0,y index=1] {dados/duto_sugado/abs_007_strouhal.txt};
 \addplot[color=black, mark=triangle] table[x index=0,y index=1] {dados/duto_sugado/abs_010_strouhal.txt};
\addplot[color=black, mark=x] table[x index=0,y index=1] {dados/duto_sugado/abs_015_strouhal.txt};
\addplot[color=black, mark=diamond] table[x index=0,y index=1] {dados/duto_sugado/abs_020_strouhal.txt};

  \end{axis}
  \end{tikzpicture}
  \caption[Coeficiente de reflexão $R_{r}$ com escoamento sugado em relação ao número de Strouhal ($St$)]{Resultado de magnitudes do coeficiente de reflexão $R_{r}$ em relação ao número de Strouhal ($St$) calculados no ponto $\textbf{P}$ na terminação do duto com vários escoamentos sugados. Calculados pela ferramenta computacional proposta nesse estudo, os pontos com $\bigcirc$ apresentam os resultados para Mach $M =$ 0,05 e número de Reynolds $Re =$ 1378,73, os pontos com $\square$ apresentam os resultados para Mach $M =$ 0,07 e número de Reynolds $Re =$ 1930,23, os pontos com $\bigtriangleup$ apresentam os resultados para Mach $M =$ 0,1 e número de Reynolds $Re =$ 2757,42, os pontos com $\times$ apresentam os resultados para Mach $M =$ 0,15 e número de Reynolds $Re =$ 2057,71 e os pontos com $\diamond$ apresentam os resultados para Mach $M =$ 0,20 e número de Reynolds $Re =$ 5514,82.}

  \label{fig:abs_r_sugado_strouhal}
\end{figure}

\newpage
\begin{figure}[ht!]
\centering
  \begin{tikzpicture}
  \begin{axis}[
  width=0.9\textwidth,
  height=0.5\textwidth,
  x tick label style={
      /pgf/number format/.cd,
          fixed,
          fixed zerofill,
          precision=2,
      /tikz/.cd
  },
  xmin=0.05,
  xmax=0.2,
  ymin=0.6826,
  ymax=1.1835,
  ytick distance=0.1,
  xtick distance=0.03,
  grid=major, % Display a grid  
  %grid style={dashed,gray!90}, % Set the style
  xlabel = $M$,
  ylabel = $|R_{r}|$,
  ]
 \addplot[color=black, thick] table[x index=0,y index=1] {dados/duto_sugado/abs_r_strouhal_mach.txt};

  \end{axis}
  \end{tikzpicture}
  \caption[Coeficiente de reflexão $R_{r}$ com escoamento de exaustão em relação ao número de Mach ($M$) no Strouhal $St = \pi/2$]{Resultado de magnitudes do coeficiente de reflexão $R_{r}$ fixados no Strouhal $St = \pi/2$ em relação ao número de Mach ($M$) para escoamentos sugados. Os resultados foram calculados no ponto $\textbf{P}$ na terminação do duto.}

  \label{fig:abs_r_sugado_strouhal_mach}
\end{figure}