\chapter{Resultados}


\section{Análise da Condição Anecóica}

% \begin{figure}
% \begin{center}
% \begin{tikzpicture}
% \begin{axis}[
%     title={},
%     width=0.9\textwidth,
%     height=0.45\textwidth,
%     xlabel={Frequência},
%     ylabel={Sensibilidade},
%     x unit={\space Hz \space},
% 	y unit={\space C/g \space},
%     ytick=data,
%     xmin=0,
%     ymin=0,
%     ymax=0.55,
%     legend pos=north west,
%     grid=minor, % Display a grid	
% 	grid style={dashed,gray!90}, % Set the style
% 	]
%      %\addplot[color=black,dashed,thick,mark=*,mark options={solid},smooth] table[x index=2,y index=3] {Data/Kollias_SBMR_095_exp.txt}; \label{Kollias_exp095}
%      %\addlegendentry{Experimental (Kollias)}
%       %\addplot[color=blue,semithick] table[x index=0,y index=1] {Data/Kollias_SBMR_095_exp.txt}; \label{Kollias_sim095}
%     %\addlegendentry{Simulação (Kollias)}
%              \addplot[color=black, thick] table[x index=0,y index=1] {dados/coeficiente_reflexao_anecoica/A_real.txt};
%              \addplot[color=black,dashed,  thick] table[x index=0,y index=1] {dados/coeficiente_reflexao_anecoica/A_imag.txt};
%               \label{Comsol_095}% \addlegendentry{Simulação (Comsol)}
% \end{axis}
% \end{tikzpicture}
%  \caption{Resposta em carga do acelerômetro para $r$ igual a 0,95.}
% \end{center}
% \end{figure}

\begin{figure}[ht!]
\centering
  \begin{tikzpicture}
  \begin{axis}[
	width=0.9\textwidth,
	height=0.5\textwidth,
	xmin=0,
    ymin=0,
    ymax=0.55,
    ytick distance=0.1,
    grid=major, % Display a grid	
 	%grid style={dashed,gray!90}, % Set the style
	xlabel = Número de Helmholtz ($ka$),
	ylabel = Impedância ($Z_{\textbf{A}}$),
  ]
 \addplot[color=black, thick] table[x index=0,y index=1] {dados/coeficiente_reflexao_anecoica/A_real.txt};
   \addplot[color=black,dashed,  thick] table[x index=0,y index=1] {dados/coeficiente_reflexao_anecoica/A_imag.txt};
   
  \end{axis}
  \end{tikzpicture}
  \caption[Impedância $Z_{\textbf{A}}$]{Resultado da impedância calculada no ponto $\textbf{A}$, próximo da condição anecóica localizada nas fronteiras do modelo numérico. A linha contínua representa a parte real e a linha tracejada representa a parte imaginária.}
  \label{fig:A_reflexao}
\end{figure}

\newpage
\begin{figure}[ht!]
\centering
  \begin{tikzpicture}
  \begin{axis}[
	width=0.9\textwidth,
	height=0.5\textwidth,
	xmin=0,
    ymin=0,
    ymax=1,
    ytick distance=0.2,
    grid=major, % Display a grid	
 	%grid style={dashed,gray!90}, % Set the style
	xlabel = Número de Helmholtz ($ka$),
	ylabel = Coeficiente de Reflexão ($R_{\textbf{A}}$),
  ]
 \addplot[color=black, thick] table[x index=0,y index=1] {dados/coeficiente_reflexao_anecoica/A_abs.txt};
   
  \end{axis}
  \end{tikzpicture}
  \caption[Coeficiente de Reflexão $R_{\textbf{A}}$]{Resultado da magnitude do coeficiente de reflexão calculado no ponto $\textbf{A}$, próximo da condição anecóica localizada nas fronteiras do modelo numérico.}
  \label{fig:A_reflexao}
\end{figure}

\newpage
\begin{figure}[ht!]
\centering
  \begin{tikzpicture}
  \begin{axis}[
	width=0.9\textwidth,
	height=0.5\textwidth,
	xmin=0,
    ymin=0,
    ymax=0.9,
    ytick distance=0.1,
    grid=major, % Display a grid	
 	%grid style={dashed,gray!90}, % Set the style
	xlabel = Número de Helmholtz ($ka$),
	ylabel = Impedância ($Z_{\textbf{B}}$),
  ]
 \addplot[color=black, thick] table[x index=0,y index=1] {dados/coeficiente_reflexao_anecoica/B_real.txt};
   \addplot[color=black,dashed,  thick] table[x index=0,y index=1] {dados/coeficiente_reflexao_anecoica/B_imag.txt};
   
  \end{axis}
  \end{tikzpicture}
  \caption[Impedância $Z_{\textbf{B}}$]{Resultado da impedância calculada no ponto $\textbf{B}$, próximo da condição anecóica localizada nas fronteiras do modelo numérico. A linha contínua representa a parte real e a linha tracejada representa a parte imaginária.}
  \label{fig:B_reflexao}
\end{figure}

\newpage
\begin{figure}[ht!]
\centering
  \begin{tikzpicture}
  \begin{axis}[
	width=0.9\textwidth,
	height=0.5\textwidth,
	xmin=0,
    ymin=0,
    ymax=1,
    ytick distance=0.2,
    grid=major, % Display a grid	
 	%grid style={dashed,gray!90}, % Set the style
	xlabel = Número de Helmholtz ($ka$),
	ylabel = Coeficiente de Reflexão ($R_{\textbf{B}}$),
  ]
 \addplot[color=black, thick] table[x index=0,y index=1] {dados/coeficiente_reflexao_anecoica/B_abs.txt};
   
  \end{axis}
  \end{tikzpicture}
  \caption[Coeficiente de Reflexão $R_{\textbf{B}}$]{Resultado da magnitude do coeficiente de reflexão calculado no ponto $\textbf{B}$, próximo da condição anecóica localizada nas fronteiras do modelo numérico.}
  \label{fig:B_reflexao}
\end{figure}

\newpage
\begin{figure}[ht!]
\centering
  \begin{tikzpicture}
  \begin{axis}[
	width=0.9\textwidth,
	height=0.5\textwidth,
	xmin=0,
    ymin=0,
    ymax=0.7,
    ytick distance=0.1,
    grid=major, % Display a grid	
 	%grid style={dashed,gray!90}, % Set the style
	xlabel = Número de Helmholtz ($ka$),
	ylabel = Impedância ($Z_{\textbf{C}}$),
  ]
 \addplot[color=black, thick] table[x index=0,y index=1] {dados/coeficiente_reflexao_anecoica/C_real.txt};
   \addplot[color=black,dashed,  thick] table[x index=0,y index=1] {dados/coeficiente_reflexao_anecoica/C_imag.txt};
   
  \end{axis}
  \end{tikzpicture}
  \caption[Impedância $Z_{\textbf{C}}$]{Resultado da impedância calculada no ponto $\textbf{C}$, próximo da condição anecóica localizada nas fronteiras do modelo numérico. A linha contínua representa a parte real e a linha tracejada representa a parte imaginária.}
  \label{fig:C_reflexao}
\end{figure}

\newpage
\begin{figure}[ht!]
\centering
  \begin{tikzpicture}
  \begin{axis}[
	width=0.9\textwidth,
	height=0.5\textwidth,
	xmin=0,
    ymin=0,
    ymax=1,
    ytick distance=0.2,
    grid=major, % Display a grid	
 	%grid style={dashed,gray!90}, % Set the style
	xlabel = Número de Helmholtz ($ka$),
	ylabel = Coeficiente de Reflexão ($R_{\textbf{C}}$),
  ]
 \addplot[color=black, thick] table[x index=0,y index=1] {dados/coeficiente_reflexao_anecoica/C_abs.txt};
   
  \end{axis}
  \end{tikzpicture}
  \caption[Coeficiente de Reflexão $R_{\textbf{C}}$]{Resultado da magnitude do coeficiente de reflexão calculado no ponto $\textbf{C}$, próximo da condição anecóica localizada nas fronteiras do modelo numérico.}
  \label{fig:C_reflexao}
\end{figure}

\newpage
\begin{figure}[ht!]
\centering
  \begin{tikzpicture}
  \begin{axis}[
	width=0.9\textwidth,
	height=0.5\textwidth,
	xmin=0,
  xmax=2.5,
    ymin=0,
    ymax=0.62,
    ytick distance=0.1,
    grid=major, % Display a grid	
 	%grid style={dashed,gray!90}, % Set the style
	xlabel = Número de Helmholtz ($ka$),
	ylabel = Impedância ($Z_{\textbf{D}}$),
  ]
 \addplot[color=black, thick] table[x index=0,y index=1] {dados/coeficiente_reflexao_anecoica/D_real.txt};
   \addplot[color=black,dashed,  thick] table[x index=0,y index=1] {dados/coeficiente_reflexao_anecoica/D_imag.txt};
   
  \end{axis}
  \end{tikzpicture}
  \caption[Impedância $Z_{\textbf{D}}$]{Resultado da impedância calculada no ponto $\textbf{D}$, próximo da condição anecóica localizada nas fronteiras do modelo numérico. A linha contínua representa a parte real e a linha tracejada representa a parte imaginária.}
  \label{fig:D_reflexao}
\end{figure}

\newpage
\begin{figure}[ht!]
\centering
  \begin{tikzpicture}
  \begin{axis}[
	width=0.9\textwidth,
	height=0.5\textwidth,
	xmin=0,
  xmax=2.5,
    ymin=0,
    ymax=1,
    ytick distance=0.2,
    grid=major, % Display a grid	
 	%grid style={dashed,gray!90}, % Set the style
	xlabel = Número de Helmholtz ($ka$),
	ylabel = Coeficiente de Reflexão ($R_{\textbf{D}}$),
  ]
 \addplot[color=black, thick] table[x index=0,y index=1] {dados/coeficiente_reflexao_anecoica/D_abs.txt};
   
  \end{axis}
  \end{tikzpicture}
  \caption[Coeficiente de Reflexão $R_{\textbf{D}}$]{Resultado da magnitude do coeficiente de reflexão calculado no ponto $\textbf{D}$, próximo da condição anecóica localizada nas fronteiras do modelo numérico.}
  \label{fig:D_reflexao}
\end{figure}


\newpage
\section{Duto sem Escoamento}

\begin{figure}[ht!]
\centering
  \begin{tikzpicture}
  \begin{axis}[
  width=0.9\textwidth,
  height=0.5\textwidth,
  xmin=0,
  xmax=1.8,
    ymin=0.4,
    ymax=1,
    ytick distance=0.1,
    grid=major, % Display a grid  
  %grid style={dashed,gray!90}, % Set the style
  xlabel = Número de Helmholtz ($ka$),
  ylabel = Coeficiente de Reflexão ($R_{r}$),
  ]
 \addplot[color=black, thick] table[x index=0,y index=1] {dados/duto_sem_escoamento/analytical_data_abs_r.txt};
 \addplot[color=black, mark=o, only marks] table[x index=0,y index=1] {dados/duto_sem_escoamento/simulation_data_abs_r.txt};
   
  \end{axis}
  \end{tikzpicture}
  \caption[Coeficiente de Reflexão $R_{r}$ sem Escoamento]{Resultado da magnitude do coeficiente de reflexão $R_{r}$ calculado no ponto $\textbf{P}$ na terminação do duto sem escoamento. A linha contínua representa o resultado analítico do estudo de \citeonline{levine1948radiation} e os pontos circulares representam os resultados calculados pela ferramenta computacional proposta nesse estudo. A correlação entre os resultados foi de 99,95 \%.}
  \label{fig:abs_r_boca}
\end{figure}

\newpage
\begin{figure}[ht!]
\centering
  \begin{tikzpicture}
  \begin{axis}[
  width=0.9\textwidth,
  height=0.5\textwidth,
  xmin=0,
  xmax=1.8,
    ymin=0.4,
    ymax=0.65,
    ytick distance=0.05,
    grid=major, % Display a grid  
  %grid style={dashed,gray!90}, % Set the style
  xlabel = Número de Helmholtz ($ka$),
  ylabel = Correção da Terminação ($l/a$),
  ]
 \addplot[color=black, thick] table[x index=0,y index=1] {dados/duto_sem_escoamento/analytical_data_loa.txt};
 \addplot[color=black, mark=o, only marks] table[x index=0,y index=1] {dados/duto_sem_escoamento/simulation_data_loa.txt};
   
  \end{axis}
  \end{tikzpicture}
  \caption[Coeficiente de Correção da Terminação $l/a$ sem Escoamento]{Resultado do coeficiente de correção da terminação $l/a$ calculado no ponto $\textbf{P}$ na terminação do duto sem escoamento. A linha contínua representa o resultado analítico do estudo de \citeonline{levine1948radiation} e os pontos circulares representam os resultados calculados pela ferramenta computacional proposta nesse estudo. A correlação entre os resultados foi de 96,23 \%.}
  \label{fig:loa_boca}
\end{figure}

\newpage
\section{Duto com Escoamento de Exaustão}

\subsection{Mach 0,2}
\begin{figure}[ht!]
\centering
  \begin{tikzpicture}
  \begin{axis}[
  width=0.9\textwidth,
  height=0.5\textwidth,
  xmin=0,
  xmax=1.8,
    ymin=0.4,
    ymax=1.2,
    ytick distance=0.1,
    grid=major, % Display a grid  
  %grid style={dashed,gray!90}, % Set the style
  xlabel = Número de Helmholtz ($ka$),
  ylabel = Coeficiente de Reflexão ($R_{r}$),
  ]
 \addplot[color=black, thick] table[x index=0,y index=1] {dados/duto_exaustao/abs_r_020_analytical.txt};
 \addplot[color=black, mark=o, only marks] table[x index=0,y index=1] {dados/duto_exaustao/abs_r_020_simulation.txt};
   
  \end{axis}
  \end{tikzpicture}
  \caption[Coeficiente de Reflexão $R_{r}$ com Escoamento de Exaustão (M $=$ 0,2)]{Resultado da magnitude do coeficiente de reflexão $R_{r}$ calculado no ponto $\textbf{P}$ na terminação do duto com escoamento de exaustão (M $=$ 0,2 e Re $=$ 5514,82). A linha contínua representa o resultado analítico do estudo de \citeonline{munt1990acoustic} e os pontos circulares representam os resultados calculados pela ferramenta computacional proposta nesse estudo. A correlação entre os resultados foi de 98,1 \%.}
  \label{fig:abs_r_boca_020}
\end{figure}

\newpage
\begin{figure}[ht!]
\centering
  \begin{tikzpicture}
  \begin{axis}[
  width=0.9\textwidth,
  height=0.5\textwidth,
  xmin=0,
  xmax=1.8,
    ymin=0.15,
    ymax=0.55,
    ytick distance=0.05,
    grid=major, % Display a grid  
  %grid style={dashed,gray!90}, % Set the style
  xlabel = Número de Helmholtz ($ka$),
  ylabel = Correção da Terminação ($l/a$),
  ]
 \addplot[color=black, thick] table[x index=0,y index=1] {dados/duto_exaustao/loa_020_analytical.txt};
 \addplot[color=black, mark=o, only marks] table[x index=0,y index=1] {dados/duto_exaustao/loa_020_simulation.txt};
   
  \end{axis}
  \end{tikzpicture}
  \caption[Coeficiente de Correção da Terminação $l/a$ com Escoamento de Exaustão (M $=$ 0,2)]{Resultado do coeficiente de correção da terminação $l/a$ calculado no ponto $\textbf{P}$ na terminação do duto com escoamento de (M $=$ 0,2 e Re $=$ 5514,82). A linha contínua representa o resultado analítico do estudo de \citeonline{munt1990acoustic} e os pontos circulares representam os resultados calculados pela ferramenta computacional proposta nesse estudo. A correlação entre os resultados foi de 79,84 \%.}
  \label{fig:loa_boca_020}
\end{figure}

\newpage
\subsection{Mach = 0,15}

\begin{figure}[ht!]
\centering
  \begin{tikzpicture}
  \begin{axis}[
  width=0.9\textwidth,
  height=0.5\textwidth,
  xmin=0,
  xmax=1.8,
    ymin=0.4,
    ymax=1.2,
    ytick distance=0.1,
    grid=major, % Display a grid  
  %grid style={dashed,gray!90}, % Set the style
  xlabel = Número de Helmholtz ($ka$),
  ylabel = Coeficiente de Reflexão ($R_{r}$),
  ]
 \addplot[color=black, thick] table[x index=0,y index=1] {dados/duto_exaustao/abs_r_015_analytical.txt};
 \addplot[color=black, mark=o, only marks] table[x index=0,y index=1] {dados/duto_exaustao/abs_r_015_simulation.txt};
   
  \end{axis}
  \end{tikzpicture}
  \caption[Coeficiente de Reflexão $R_{r}$ com Escoamento de Exaustão (M $=$ 0,15)]{Resultado da magnitude do coeficiente de reflexão $R_{r}$ calculado no ponto $\textbf{P}$ na terminação do duto com escoamento de exaustão (M $=$ 0,15 e Re $=$ 2057,71). A linha contínua representa o resultado analítico do estudo de \citeonline{munt1990acoustic} e os pontos circulares representam os resultados calculados pela ferramenta computacional proposta nesse estudo. A correlação entre os resultados foi de 99,80 \%.}
  \label{fig:abs_r_boca_015}
\end{figure}

\newpage
\begin{figure}[ht!]
\centering
  \begin{tikzpicture}
  \begin{axis}[
  width=0.9\textwidth,
  height=0.5\textwidth,
  xmin=0,
  xmax=1.8,
    ymin=0.15,
    ymax=0.55,
    ytick distance=0.05,
    grid=major, % Display a grid  
  %grid style={dashed,gray!90}, % Set the style
  xlabel = Número de Helmholtz ($ka$),
  ylabel = Correção da Terminação ($l/a$),
  ]
 \addplot[color=black, thick] table[x index=0,y index=1] {dados/duto_exaustao/loa_015_analytical.txt};
 \addplot[color=black, mark=o, only marks] table[x index=0,y index=1] {dados/duto_exaustao/loa_015_simulation.txt};
   
  \end{axis}
  \end{tikzpicture}
  \caption[Coeficiente de Correção da Terminação $l/a$ com Escoamento de Exaustão (M $=$ 0,15)]{Resultado do coeficiente de correção da terminação $l/a$ calculado no ponto $\textbf{P}$ na terminação do duto com escoamento de (M $=$ 0,15 e Re $=$ 2057,71). A linha contínua representa o resultado analítico do estudo de \citeonline{munt1990acoustic} e os pontos circulares representam os resultados calculados pela ferramenta computacional proposta nesse estudo. A correlação entre os resultados foi de 94,28 \%.}
  \label{fig:loa_boca_015}
\end{figure}


\newpage
\section{Duto com Escoamento Sugado}

\begin{figure}[ht!]
\centering
  \begin{tikzpicture}
  \begin{axis}[
  width=0.9\textwidth,
  height=0.5\textwidth,
  x tick label style={
      /pgf/number format/.cd,
          fixed,
          fixed zerofill,
          precision=2,
      /tikz/.cd
  },
  xmin=0,
  xmax=0.2,
  ymin=0.68,
  ymax=1,
  ytick distance=0.05,
  xtick distance=0.05,
  grid=major, % Display a grid  
  %grid style={dashed,gray!90}, % Set the style
  xlabel = Número de Mach ($M$),
  ylabel = Coeficiente de Reflexão ($R_{M}$),
  ]
 \addplot[color=black, thick] table[x index=0,y index=1] {dados/duto_sugado/davis_analytical.txt};
 \addplot[color=black, mark=o, only marks] table[x index=0,y index=1] {dados/duto_sugado/davis_simulation.txt};
   
  \end{axis}
  \end{tikzpicture}
  \caption[Coeficiente de reflexão $R_{M}$ com escoamento sugado]{Resultado do coeficiente de reflexão $R_{M}$ em relação ao Mach para baixas frequências ($ka$ $<$ $0,25$) com escoamento sugado. A linha contínua apresenta o resultado do estudo de \citeonline{davies1987} e os pontos circulares representam os resultados calculados pela ferramenta computacional proposta nesse estudo. A correlação entre os resultados foi de 95,45 \%.}

  \label{fig:abs_r_boca_sugado}
\end{figure}

\newpage
\begin{figure}[ht!]
\centering
  \begin{tikzpicture}
  \begin{axis}[
  width=0.9\textwidth,
  height=0.5\textwidth,
  x tick label style={
      /pgf/number format/.cd,
          fixed,
          fixed zerofill,
          precision=2,
      /tikz/.cd
  },
  xmin=0,
  xmax=0.2,
  ymin=0.1,
  ymax=1.0,
  ytick distance=0.1,
  xtick distance=0.05,
  grid=major, % Display a grid  
  %grid style={dashed,gray!90}, % Set the style
  xlabel = Número de Mach ($M$),
  ylabel = Correção da Terminação ($l_{M}$),
  ]
 \addplot[color=black, thick] table[x index=0,y index=1] {dados/duto_sugado/davis_analytical_loa.txt};
 \addplot[color=black, mark=o, only marks] table[x index=0,y index=1] {dados/duto_sugado/davis_simulation_loa.txt};
   
  \end{axis}
  \end{tikzpicture}
  \caption[Coeficiente de correção da terminação ($l_{M}$) com escoamento sugado]{Resultado do coeficiente de correção da terminação $l_{M}$ em relação ao Mach para baixas frequências ($ka$ $<$ $0,25$) com escoamento sugado. A linha contínua apresenta o resultado do estudo de \citeonline{davies1987} e os pontos circulares representam os resultados calculados pela ferramenta computacional proposta nesse estudo. A correlação entre os resultados foi de 62,53 \%.}

  \label{fig:abs_r_boca_sugado_loa}
\end{figure}