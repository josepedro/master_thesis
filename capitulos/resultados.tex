\chapter{Resultados}


\section{Análise da Condição Anecóica}

% \begin{figure}
% \begin{center}
% \begin{tikzpicture}
% \begin{axis}[
%     title={},
%     width=0.9\textwidth,
%     height=0.45\textwidth,
%     xlabel={Frequência},
%     ylabel={Sensibilidade},
%     x unit={\space Hz \space},
% 	y unit={\space C/g \space},
%     ytick=data,
%     xmin=0,
%     ymin=0,
%     ymax=0.55,
%     legend pos=north west,
%     grid=minor, % Display a grid	
% 	grid style={dashed,gray!90}, % Set the style
% 	]
%      %\addplot[color=black,dashed,thick,mark=*,mark options={solid},smooth] table[x index=2,y index=3] {Data/Kollias_SBMR_095_exp.txt}; \label{Kollias_exp095}
%      %\addlegendentry{Experimental (Kollias)}
%       %\addplot[color=blue,semithick] table[x index=0,y index=1] {Data/Kollias_SBMR_095_exp.txt}; \label{Kollias_sim095}
%     %\addlegendentry{Simulação (Kollias)}
%              \addplot[color=black, thick] table[x index=0,y index=1] {dados/coeficiente_reflexao_anecoica/A_real.txt};
%              \addplot[color=black,dashed,  thick] table[x index=0,y index=1] {dados/coeficiente_reflexao_anecoica/A_imag.txt};
%               \label{Comsol_095}% \addlegendentry{Simulação (Comsol)}
% \end{axis}
% \end{tikzpicture}
%  \caption{Resposta em carga do acelerômetro para $r$ igual a 0,95.}
% \end{center}
% \end{figure}


\begin{figure}[ht!]
\centering
  \begin{tikzpicture}
  \begin{axis}[
	width=0.9\textwidth,
	height=0.5\textwidth,
	xmin=0,
    ymin=0,
    ymax=0.55,
    ytick distance=0.1,
    grid=major, % Display a grid	
 	%grid style={dashed,gray!90}, % Set the style
	xlabel = Número de Helmholtz ($ka$),
	ylabel = Impedância ($Z_{\textbf{A}}$),
  ]
 \addplot[color=black, thick] table[x index=0,y index=1] {dados/coeficiente_reflexao_anecoica/A_real.txt};
   \addplot[color=black,dashed,  thick] table[x index=0,y index=1] {dados/coeficiente_reflexao_anecoica/A_imag.txt};
   
  \end{axis}
  \end{tikzpicture}
  \caption[Impedância $Z_{\textbf{A}}$]{Resultado da impedância calculada no ponto $\textbf{A}$, próximo da condição anecóica localizada nas fronteiras do modelo numérico. A linha contínua representa a parte real e a linha tracejada representa a parte imaginária.}
  \label{fig:A_reflexao}
\end{figure}

\begin{figure}[ht!]
\centering
  \begin{tikzpicture}
  \begin{axis}[
	width=0.9\textwidth,
	height=0.5\textwidth,
	xmin=0,
    ymin=0,
    ymax=1,
    ytick distance=0.2,
    grid=major, % Display a grid	
 	%grid style={dashed,gray!90}, % Set the style
	xlabel = Número de Helmholtz ($ka$),
	ylabel = Coeficiente de Reflexão ($R_{\textbf{A}}$),
  ]
 \addplot[color=black, thick] table[x index=0,y index=1] {dados/coeficiente_reflexao_anecoica/A_abs.txt};
   
  \end{axis}
  \end{tikzpicture}
  \caption[Coeficiente de Reflexão $R_{\textbf{A}}$]{Resultado da magnitude do coeficiente de reflexão calculado no ponto $\textbf{A}$, próximo da condição anecóica localizada nas fronteiras do modelo numérico.}
  \label{fig:A_reflexao}
\end{figure}


\begin{figure}[ht!]
\centering
  \begin{tikzpicture}
  \begin{axis}[
	width=0.9\textwidth,
	height=0.5\textwidth,
	xmin=0,
    ymin=0,
    ymax=0.9,
    ytick distance=0.1,
    grid=major, % Display a grid	
 	%grid style={dashed,gray!90}, % Set the style
	xlabel = Número de Helmholtz ($ka$),
	ylabel = Impedância ($Z_{\textbf{B}}$),
  ]
 \addplot[color=black, thick] table[x index=0,y index=1] {dados/coeficiente_reflexao_anecoica/B_real.txt};
   \addplot[color=black,dashed,  thick] table[x index=0,y index=1] {dados/coeficiente_reflexao_anecoica/B_imag.txt};
   
  \end{axis}
  \end{tikzpicture}
  \caption[Impedância $Z_{\textbf{B}}$]{Resultado da impedância calculada no ponto $\textbf{B}$, próximo da condição anecóica localizada nas fronteiras do modelo numérico. A linha contínua representa a parte real e a linha tracejada representa a parte imaginária.}
  \label{fig:B_reflexao}
\end{figure}

\begin{figure}[ht!]
\centering
  \begin{tikzpicture}
  \begin{axis}[
	width=0.9\textwidth,
	height=0.5\textwidth,
	xmin=0,
    ymin=0,
    ymax=1,
    ytick distance=0.2,
    grid=major, % Display a grid	
 	%grid style={dashed,gray!90}, % Set the style
	xlabel = Número de Helmholtz ($ka$),
	ylabel = Coeficiente de Reflexão ($R_{\textbf{B}}$),
  ]
 \addplot[color=black, thick] table[x index=0,y index=1] {dados/coeficiente_reflexao_anecoica/B_abs.txt};
   
  \end{axis}
  \end{tikzpicture}
  \caption[Coeficiente de Reflexão $R_{\textbf{B}}$]{Resultado da magnitude do coeficiente de reflexão calculado no ponto $\textbf{B}$, próximo da condição anecóica localizada nas fronteiras do modelo numérico.}
  \label{fig:B_reflexao}
\end{figure}

\begin{figure}[ht!]
\centering
  \begin{tikzpicture}
  \begin{axis}[
	width=0.9\textwidth,
	height=0.5\textwidth,
	xmin=0,
    ymin=0,
    ymax=0.7,
    ytick distance=0.1,
    grid=major, % Display a grid	
 	%grid style={dashed,gray!90}, % Set the style
	xlabel = Número de Helmholtz ($ka$),
	ylabel = Impedância ($Z_{\textbf{C}}$),
  ]
 \addplot[color=black, thick] table[x index=0,y index=1] {dados/coeficiente_reflexao_anecoica/C_real.txt};
   \addplot[color=black,dashed,  thick] table[x index=0,y index=1] {dados/coeficiente_reflexao_anecoica/C_imag.txt};
   
  \end{axis}
  \end{tikzpicture}
  \caption[Impedância $Z_{\textbf{C}}$]{Resultado da impedância calculada no ponto $\textbf{C}$, próximo da condição anecóica localizada nas fronteiras do modelo numérico. A linha contínua representa a parte real e a linha tracejada representa a parte imaginária.}
  \label{fig:C_reflexao}
\end{figure}

\begin{figure}[ht!]
\centering
  \begin{tikzpicture}
  \begin{axis}[
	width=0.9\textwidth,
	height=0.5\textwidth,
	xmin=0,
    ymin=0,
    ymax=1,
    ytick distance=0.2,
    grid=major, % Display a grid	
 	%grid style={dashed,gray!90}, % Set the style
	xlabel = Número de Helmholtz ($ka$),
	ylabel = Coeficiente de Reflexão ($R_{\textbf{C}}$),
  ]
 \addplot[color=black, thick] table[x index=0,y index=1] {dados/coeficiente_reflexao_anecoica/C_abs.txt};
   
  \end{axis}
  \end{tikzpicture}
  \caption[Coeficiente de Reflexão $R_{\textbf{C}}$]{Resultado da magnitude do coeficiente de reflexão calculado no ponto $\textbf{C}$, próximo da condição anecóica localizada nas fronteiras do modelo numérico.}
  \label{fig:C_reflexao}
\end{figure}


\begin{figure}[ht!]
\centering
  \begin{tikzpicture}
  \begin{axis}[
	width=0.9\textwidth,
	height=0.5\textwidth,
	xmin=0,
    ymin=0,
    ymax=0.62,
    ytick distance=0.1,
    grid=major, % Display a grid	
 	%grid style={dashed,gray!90}, % Set the style
	xlabel = Número de Helmholtz ($ka$),
	ylabel = Impedância ($Z_{\textbf{D}}$),
  ]
 \addplot[color=black, thick] table[x index=0,y index=1] {dados/coeficiente_reflexao_anecoica/D_real.txt};
   \addplot[color=black,dashed,  thick] table[x index=0,y index=1] {dados/coeficiente_reflexao_anecoica/D_imag.txt};
   
  \end{axis}
  \end{tikzpicture}
  \caption[Impedância $Z_{\textbf{D}}$]{Resultado da impedância calculada no ponto $\textbf{D}$, próximo da condição anecóica localizada nas fronteiras do modelo numérico. A linha contínua representa a parte real e a linha tracejada representa a parte imaginária.}
  \label{fig:D_reflexao}
\end{figure}

\begin{figure}[ht!]
\centering
  \begin{tikzpicture}
  \begin{axis}[
	width=0.9\textwidth,
	height=0.5\textwidth,
	xmin=0,
    ymin=0,
    ymax=1,
    ytick distance=0.2,
    grid=major, % Display a grid	
 	%grid style={dashed,gray!90}, % Set the style
	xlabel = Número de Helmholtz ($ka$),
	ylabel = Coeficiente de Reflexão ($R_{\textbf{D}}$),
  ]
 \addplot[color=black, thick] table[x index=0,y index=1] {dados/coeficiente_reflexao_anecoica/D_abs.txt};
   
  \end{axis}
  \end{tikzpicture}
  \caption[Coeficiente de Reflexão $R_{\textbf{D}}$]{Resultado da magnitude do coeficiente de reflexão calculado no ponto $\textbf{D}$, próximo da condição anecóica localizada nas fronteiras do modelo numérico.}
  \label{fig:D_reflexao}
\end{figure}

\section{Duto sem Escoamento}