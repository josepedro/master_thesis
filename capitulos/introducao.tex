\chapter{Introdução}
\label{chapter:introdcao}

\section{Contexto}

% Falar sobre sistemas de exaustão com exemplos
Sistemas de exaustão hoje em dia possuem uma forte colaboração na composição de sons e ruídos. Escapamentos, sistemas de ventilação, buzinas e motores aeronáuticos são exemplos desses sistemas que estão altamente presentes no dia-dia. Cada vez mais a sociedade vem desenvolvendo consciência crítica dos danos que os ruídos desses tipos de sistemas podem acarretar a saúde da população. Tal fato é tão preponderante que, como é apresentado por \citeonline{munjal}, desde os anos da década de 1920 há registros de esforços para entender e caracterizar esses tipos sistemas afim de colaborar com a manutenção e desenvolvimento de ambientes saudáveis no contexto acústico.

% Falar sobre dutos
Há vários elementos estruturais que podem compor sistemas de exaustão, mas os dutos se caracterizam como fundamentais e bastante presentes. Sua forma cilíndrica permite que vários fenômenos físicos possam ocorrer e interagir entre si, principalmente os fenômenos acústicos e de fluxo de massa (escoamentos). De acordo com \citeonline{munjal}, o corpo de estudos e conhecimentos da acústica interna de dutos está bem estabelecido, mas verifica-se na literatura vários questionamentos sobre o funcionamento do mesmo na presença de escoamentos (fenômenos aeroacústicos). Em vista disso, determinar a caracterização da acústica interna de dutos é de extrema importância visto as várias tecnologias relacionadas a sistemas de exaustão sem um amparo técnico bem estabelecido da literatura.

% Falar sobre coeficiente de reflexão e correção comprimento
Em geral, pode-se utilizar dois parâmetros para caracterizar o fenômeno da acústica interna de dutos:

\begin{itemize}
    \item a magnitude do coeficiente de reflexão $\|R\|$, razão entre as componentes refletida e incidente da onda no duto, a qual é dada por
    \begin{equation}
        R_{r} =\frac{Z_{r} - Z_{0}}{Z_{r} + Z_{0}},
        \label{eq:R}
    \end{equation}
    sendo $Z_{r}$ a impedância de radiação e $Z_{0}$ a impedância característica do meio;
    
    \item coeficiente de correção da terminação normalizado pelo raio do duto $l/a$ em que $a$ é o raio do duto. Tal parâmetro representa o comprimento acústico efetivo do duto. Em outras palavras, o fator $l$ é a quantidade adicional medida a partir da abertura do duto a qual deve propagar a onda incidente antes de ser refletida para o interior do duto com fase invertida. Tal coeficiente de correção da terminação $l$ é dado por
    \begin{equation}
        l = \frac{1}{k} \arctan\!\left(\frac{Z_{r}}{Z_{0} \, \mathrm{i}}\right)
        \label{eq:l}
    \end{equation}
    sendo $k$ o número de onda.
\end{itemize}

Com o uso desses dois parâmetros pode-se projetar sistemas de exaustão com um comportamento vibroacústico adequado a diversas situações que exigem atenuação de ruídos e vibrações em certas frequências, além de poder prever com mais acurácia já que grande parte de estudos consideram a acústica interna de dutos sem escoamentos.

\section{Problema}

\section{Objetivos}