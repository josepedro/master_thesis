\chapter{Introdução}
\label{chapter:introdcao}

\section{Contexto}

% Falar sobre sistemas de exaustão com exemplos
Sistemas de fluxo de massa (exaustão e sucção) podem se tornar uma considerável fonte de ruído. Escapamentos, sistemas de ventilação, buzinas, motores aeronáuticos e aspiradores de pó são exemplos desses sistemas que estão altamente presentes no dia-a-dia. Cada vez mais, a sociedade vem desenvolvendo consciência crítica dos danos que os ruídos desses tipos de sistemas podem acarretar à saúde da população. Tal fato é tão preponderante que, como é apresentado por \citeonline{munjal}, desde os anos da década de 1920 há registros de esforços para entender e caracterizar esses tipos sistemas, afim de colaborar com a manutenção e desenvolvimento de ambientes saudáveis no contexto acústico.

% Falar sobre dutos
Há vários elementos estruturais que podem compor sistemas de exaustão, mas os dutos circulares se caracterizam como fundamentais e bastante presentes. De acordo também com \citeonline{munjal}, o corpo de estudos e conhecimentos da acústica interna de dutos está bem estabelecido, mas verifica-se na literatura vários questionamentos sobre o funcionamento da dinâmica acústica de um duto na presença de escoamentos. Em vista disso, caracterizar a acústica interna de dutos é de extrema importância visto as várias tecnologias relacionadas a sistemas de exaustão sem um amparo técnico bem estabelecido na literatura do ponto de vista da aeroacústica.

% Falar sobre coeficiente de reflexão e correção comprimento
Em geral, quando o campo acústico interno é constituído por ondas planas (modos normais), o campo de pressão interno pode ser caracterizado pela condição de contorno na sáida do duto. Neste caso, pode-se utilizar os seguintes parâmetros para análise:

\begin{itemize}
    \item a magnitude do coeficiente de reflexão $\|R_{r}\|$\simbolo{$\|R_{r}\|$}{Magnitude do coeficiente de reflexão}, razão entre as componentes refletida e incidente da onda no duto;
    

    \item coeficiente de correção da terminação do duto $l$\simbolo{$l$}{Coeficiente de correção da terminação}, normalizado pelo raio $a$ do mesmo\simbolo{$a$}{Raio do duto}. Tal parâmetro representa o comprimento adicional para o cálculo do comprimento efetivo do duto. Em outras palavras, o fator $l$ é a quantidade adicional medida a partir da abertura do duto a qual se deve propagar a onda incidente antes de ser refletida para o interior do duto com fase invertida.
\end{itemize}

\newpage
Com o uso desses dois parâmetros, pode-se prever de maneira mais precisa o campo acústico interno de dutos e, consequentemente, delinear de maneira mais acertiva as estratégias para a redução de ruído.

\section{Problema}

Com relação aos parâmetros acima discutidos, a solução exata para o problema de um duto circular não flangeado na ausência de escoamento foi proposta por \citeonline{levine1948radiation}. A solução assume que a espessura das paredes do duto são infinitamente finas e o fluido é invíscido. A partir destas simplificações, as expressões exatas para $R_{r}$ e $l$ são obtidas utilizando-se a técnica de Wiener-Hopf. Vale ressaltar também que o mesmo modelo prevê a diretividade do som irradiado pelo duto assim como é feito pelos parâmetros abordados.

Apesar da utilidade do modelo de Levine e Schwinger, em boa parte das aplicações práticas, dutos circulares transportam escoamentos médios. Para tais circunstâncias, \citeonline{munt1990acoustic} propôs um modelo analítico exato, também baseado na técnica de Wiener-Hopf, em que se considera a presença de um escoamento subsônico no interior do duto. Considera-se nesse modelo as premissas de que o escoamento é uniforme, invíscido e que a camada cisalhante do jato é infinitamente fina. Além disso, o modelo considera a condição de Kutta na borda do duto como condição de contorno de velocidade de partícula nessa região.

É importante ressaltar que modelos exatos para os parâmetros de radiação de dutos se limitam a condições de contorno simples envolvendo um duto sem flange ou com flange infinito. No entanto, observa-se na prática situações com geometrias bastante distintas daquelas previstas pelos modelos analíticos disponíveis. Além disso, a presença de escoamentos, o que é comum nestes sistemas, muda consideravelmente o comportamento acústico do coeficiente de reflexão.

No que diz respeito a escoamentos, há de se considerar que escoamentos de exaustão e sucção possuem fenomenologias distintas. A influência de escoamentos de exaustão no campo acústico interno de dutos com geometrias simples possuem um mapeamento na literatura consolidado, porém isso não se aplica a escoamentos de sucção. Escoamentos de sucção possuem peculiaridades que ainda devem ser consideradas e investigadas nos cálculos de $R_{r}$ e $l$ como, por exemplo, o surgimento da ``\textit{vena} contracta'' na terminação do duto e sua influência no campo acústico interno.    
\newpage
Por conta da complexidade analítica em abordar o problema da radiação de dutos em condições geométricas reais, faz-se necessário a utilização de técnicas numéricas como alternativa na investigação desses fenômenos.


% FALTA ARRUMAR OBJETIVOS E ORGANIZACAO DO TRABALHO
\section{Objetivos}

Considerando a problemática discutida acima, o objetivo principal desse trabalho é desenvolver uma ferramenta computacional para análise do coeficiente de reflexão para modos normais em dutos na presença de escoamentos de baixo número de Mach (M $\leq$ 0,2).

Tem-se como objetivos específicos:
\begin{itemize}
    \item implementar um esquema computacional tridimensional para a avaliação do coeficiente de reflexão em dutos a partir do método de \textit{lattice} Boltzmann;
    \item construir condições de contorno necessárias, afim de representar o problema da reflexão de onda em dutos na presença de baixos números de Mach;
    \item implementar, validar e analisar o comportamento acústico interno de dutos não flangeados com e sem escoamento de exaustão e com ondas planas;
    \item implementar e analisar o comportamento acústico interno de dutos não flangeados com escoamento sugado e com ondas planas.
\end{itemize}

\section{Organização do Trabalho}

Esse trabalho está organizado em capítulos a partir da seguinte estrutura: 

\begin{itemize}
	\item Capítulo 2 apresenta a revisão bibliográfica envolvendo os métodos analíticos exatos e aproximados para a radiação de modos normais em dutos. Uma revisão acerca dos métodos computacionais na abordagem do problema de radiação também é apresentada neste capítulo;
	\item Capítulo 3 apresenta o método de lattice Boltzmann utilizado neste trabalho e descreve o esquema numérico desenvolvido para as simulações;
	\item Capítulo 4 apresenta os resultados da implementação computacional, validações do modelo e análises com diferentes condições de escoamento;
	\item Capítulo 5 apresenta as conclusões e evoluções futuras do trabalho. Segue no final referências bibliográficas, apêndices e anexos.
\end{itemize}