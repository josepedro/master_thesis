\chapter{Introdução}
\label{chapter:introdcao}

\section{Contexto}

% Falar sobre sistemas de exaustão com exemplos
Sistemas de fluxo de massa (exaustão e sucção) podem se tornar uma considerável fonte de ruído. Escapamentos, sistemas de ventilação, buzinas, motores aeronáuticos e aspiradores de pó são exemplos desses sistemas que estão altamente presentes no dia-a-dia. Cada vez mais a sociedade vem desenvolvendo consciência crítica dos danos que os ruídos desses tipos de sistemas podem acarretar a saúde da população. Tal fato é tão preponderante que, como é apresentado por \citeonline{munjal}, desde os anos da década de 1920 há registros de esforços para entender e caracterizar esses tipos sistemas, afim de colaborar com a manutenção e desenvolvimento de ambientes saudáveis no contexto acústico.

% Falar sobre dutos
Há vários elementos estruturais que podem compor sistemas de exaustão, mas os dutos circulares se caracterizam como fundamentais e bastante presentes. De acordo também com \citeonline{munjal}, o corpo de estudos e conhecimentos da acústica interna de dutos está bem estabelecido, mas verifica-se na literatura vários questionamentos sobre o funcionamento da dinâmica acústica de um duto na presença de escoamentos (fenômenos aeroacústicos). Em vista disso, caracterizar a acústica interna de dutos é de extrema importância visto as várias tecnologias relacionadas a sistemas de exaustão sem um amparo técnico bem estabelecido da literatura no ponto de vista da aeroacústica.

% Falar sobre coeficiente de reflexão e correção comprimento
Em geral, pode-se utilizar dois parâmetros para caracterizar o campo acústico interno de dutos com paredes rígidas em baixas frequências:

\begin{itemize}
    \item a magnitude do coeficiente de reflexão $\|R\|$\simbolo{$\|R\|$}{Magnitude do coeficiente de reflexão}, razão entre as componentes refletida e incidente da onda no duto;
    

    \item coeficiente de correção da terminação do duto $l$\simbolo{$l$}{Coeficiente de correção da terminação}, normalizado pelo raio $a$ do mesmo\simbolo{$a$}{Raio do duto}. Tal parâmetro representa o comprimento adicional para o cálculo do comprimento efetivo do duto. Em outras palavras, o fator $l$ é a quantidade adicional medida a partir da abertura do duto a qual se deve propagar a onda incidente antes de ser refletida para o interior do duto com fase invertida.
\end{itemize}

Com o uso desses dois parâmetros, pode-se prever de maneira mais precisa o campo acústico interno de dutos e, consequentemente, delinear de maneira mais acertiva as estratégias para a redução de ruído.

\section{Problema}

Com relação aos parâmetros acima discutidos, a solução exata para o problema de um duto circular não flangeado na ausência de escoamento foi proposta por \citeonline{levine1948radiation}. A solução assume que a espessura das paredes do duto são infinitamente finas e o fluido é inviscido. A partir destas simplificações, as expressões exatas para $\|R\|$ e $l$ são obtidas utilizando-se a técnica de Wiener-Hopf.

Apesar da utilidade do modelo de Levine e Schwinger, em boa parte das aplicações práticas, dutos circulares transportam escoamentos médios. Para tais circunstâncias, \citeonline{munt1990acoustic} propôs um modelo analítico exato, também baseado na técnica de Wiener-Hopf, em que se considera a presença de um escoamento subsônico no interior do duto. Considera-se nesse modelo as premissas de que o escoamento é uniforme, invíscido e que a camada cisalhante do jato é infinitamente fina. Além disso, o modelo considera a condição de Kutta na borda do duto como condição de contorno de velocidade de partícula nessa região.

É importante ressaltar que modelos exatos para os parâmetros de radiação de dutos se limitam a condições de contorno simples. No entanto, observa-se na prática situações diversas em que há presença de escoamentos de exaustão e sucção com diversas geometrias. Para estes casos, não existem modelos que considerem a influência do escoamento nas propriedades de radiação. Tal fato é bastante crítico pois o comportamento acústico de um sistema na presença de escoamentos internos muda consideralmente.

Por conta da complexidade analítica em abordar o problema da radiação de dutos em condições geométricas reais, faz-se necessário a utilização de técnicas numéricas como alternativa na investigação desses fenômenos. $Softwares$ como \citeonline{comsol} e \citeonline{powerflow} possuem a viabilidade de realizar cálculos de fluido dinâmica computacional de sistemas complexos como carros e aviões. Essa capacidade técnica é oriunda em maior parte pelas tecnologias de processamento paralelo multinúcleo de processadores e implementações de seus respectivos $softwares$ protolocos como Open MPI \citeonline{openmpi}. Essa evolução tecnológica é fundamental para esse presente trabalho e vem sendo essencial também para o surgimento de outras ferramentas, que dão suporte a exploração e descoberta de novos fenômenos físicos, antes muitas vezes inviáveis de estudar por alto custo de bancadas experimentais ou alta complexidade na consolidação de um modelo matemático representativo. 


% FALTA ARRUMAR OBJETIVOS E ORGANIZACAO DO TRABALHO
\section{Objetivos}

Considerando a problemática discutida acima, o objetivo principal desse trabalho é desenvolver uma ferramenta computacional para análise do comportamento acústico interno de dutos na presença de escoamentos de baixo número de Mach (M $<$ 0,2).

Tem-se como objetivos específicos:
\begin{itemize}
    \item implementar e validar o método numérico e condições de contorno no ponto de vista acústico;
    \item implementar, validar e analisar o comportamento acústico interno de dutos não flangeados sem escoamento e com ondas planas;
    \item implementar, validar e analisar o comportamento acústico interno de dutos não flangeados com escoamento de exaustão e com ondas planas;
    \item implementar, validar e analisar o comportamento acústico interno de dutos não flangeados com escoamento sugado e com ondas planas.
\end{itemize}

\section{Organização do Trabalho}

Esse trabalho está organizado em capítulos. O capítulo 2 apresenta a revisão bibliográfica do problema de acústica de dutos e a aplicação do método de \textit{lattice} Boltzmann nesse contexto. O capítulo 3 apresenta a metodologia do trabalho, apresentação do método numérico de \textit{lattice} Boltzmann, o \textit{software} desenvolvido como ferramenta computacional e o esquemático do modelo numérico. O capítulo 4 apresenta os resultados da implementação computacional, validações do modelo e análises com diferentes condições de escoamento. O capítulo 5 apresenta as conclusões e evoluções futuras do trabalho. Segue no final referências bibliográficas, apêndices e anexos.