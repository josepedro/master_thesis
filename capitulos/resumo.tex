Analisar e estudar acústica de dutos na presença de fluxo de massa vem se tornado um desafio visto que a interação entre esses dois fenômenos afeta significativamente o comportamento acústico. Estudos investigativos matemáticos ou experimentais podem ser inviáveis nesse sentido devido a complexidade matemática ou altos custos de bancada e, para contornar esse problema, faz-se o uso de métodos numéricos e tecnologias computacionais. Nesse trabalho é desenvolvido e validado uma ferramenta computacional para análise do coeficiente de reflexão para modos normais em dutos na presença de escoamentos de baixo número de Mach ($M \leq 0,2$). Para tal foi utilizado o método de \textit{lattice} Boltzmann e suas condições de contorno implementados no software livre Palabos, bem como constiuído um modelo numérico tridimensional de um duto não flangeado. Foram abordados as condições sem escoamento, com escoamento de exaustão e com escoamento succionado para validar a ferramenta computacional, resultando em boas concordâncias com a literatura vigente, além de propiciar mais informações sobre regimes de succção no que diz respeito interação de vórtices com o campo acústico.