\chapter{Metodologia}

Assim como foi discutido no capítulo anterior, o método de \textit{lattice} Boltzmann possui bastante utilidade quando se trata de problemas aeroacústicos, pequenas flutuações de pressão e fenômentos de turbulência. Isso se deve pelo fato do método ter surgido de uma outra abordagem de fenômenos mecânicos aplicados a fluidos - uma abordagem microscópica de interações entre moléculas.

Essa abordagem se chama Dinâmica Molecular (DM)\abreviatura{DM}{Dinâmica Molecular} e é baseada nas formulações Newtonianas de choque e propagação de partículas, em outras palavras, as posições no espaço e as velocidades podem ser obtidas a partir da aplicação da segunda lei de Newton para cada partícula. Segundo essa ideia, outras propriedades do fluido como densidade, pressão e temperatura podem ser facilmente recuperadas através do cáculo da média correspondente a um conjunto de partículas. Porém o principal problema dessa abordagem é que há uma grande quantidade de equações para se resolver num pequeno volume de fluido, pois, considerando que, de acordo com o número de Avogrado, nesse mesmo volume há na ordem de $10^{23}$ moléculas para calular os movimentos cinéticos. Tal fato se torna inviável para implementação mesmo com máquinas potentes como \textit{clusters}
