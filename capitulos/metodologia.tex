\chapter{Metodologia}

Assim como foi discutido no capítulo anterior, o método de \textit{lattice} Boltzmann possui bastante utilidade quando se trata de problemas aeroacústicos, pequenas flutuações de pressão e fenômentos de turbulência. Isso se deve pelo fato do método ter surgido de uma outra abordagem de fenômenos mecânicos aplicados a fluidos - uma abordagem microscópica de interações entre moléculas.

Essa abordagem se chama Dinâmica Molecular (DM)\abreviatura{DM}{Dinâmica Molecular} e é baseada nas formulações Newtonianas de choque e propagação de partículas, em outras palavras, as posições no espaço e as velocidades podem ser obtidas a partir da aplicação da segunda lei de Newton para cada partícula. Segundo essa ideia, outras propriedades do fluido como densidade, pressão e temperatura podem ser facilmente recuperadas através do cáculo da média correspondente a um conjunto de partículas. Porém o principal problema dessa abordagem é que há uma grande quantidade de equações para se resolver num pequeno volume de fluido, pois, considerando que, de acordo com o número de Avogrado, nesse mesmo volume há na ordem de $10^{23}$ moléculas para calular os movimentos cinéticos. Tal fato se torna inviável para implementação mesmo com computadores potentes como \textit{clusters} de alto desempenho.

Uma solução para contornar o problema da grande quantidade de equações do movimento é abordar o fenômeno físico estatisticamente, ou seja, formular a evolução do movimento do fluido no tempo em termos de uma equação de transporte: uma função de distribuição de partículas. Uma equação de transporte bastante apropriada é a equação de Boltzmann que, ao ser discretizada, pode ser resolvida de forma numérica originando assim o método de \textit{lattice} Boltzmann ou \textit{lattice} \textit{Boltzmann} \textit{Method} (LBM)\abreviatura{LBM}{\textit{lattice} \textit{Boltzmann} \textit{Method}}.

Historicamente o método de \textit{lattice} Boltzmann se originou a partir de um modelo de DM chamada \textit{Lattice} \textit{Gas} \textit{Automata} (LGA)\abreviatura{LGA}{\textit{Lattice} \textit{Gas} \textit{Automata}}. Esse modelo surgiu nos anos 80 com o estudo de \citeonline{frisch1986} mostrando a recuperação das equações de Navier-Stokes para pequenos números de Knudsen. Como esse modelo funciona somente para choque de partículas singulares, houve a necessidade de um modelo mais sofisticado e completo, então nos anos 90 e 2000 os trabalhos de \citeonline{sterling1996} e \citeonline{wolf2004} consolidaram o LBM sanando essa limitação com um processo de choques de conjunto de partículas.

O LBM possui muitas vantagens em relação a técnicas tradicionais de fluido dinâmica computacional aplicadas a aeroacústica: resolve o campo acústico e o campo fluido dinâmico ao mesmo tempo em cada incremento de tempo, extração direta do campo de pressão e fácil implementação paralela elevando assim a performance frente a outros métodos. Em vista do exposto, esse capítulo tratará o LBM como metodologia com detalhes de uso, implementação e aplicação para o desenvolvimendo do modelo de um duto com condições de contorno diversas.

\section{O Método de Lattice Boltzmann}


