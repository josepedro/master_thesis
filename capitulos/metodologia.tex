\chapter{Metodologia}

Tal como foi discutido no capítulo anterior, para a investigação do coeficiente de reflexão na presença de escoamento, faz-se ncessária a utilização de esquemas numéricos que integrem na mesma estrutura a parte fluido dinâmica e acústica. Neste sentido, o método de lattice Boltzmann mostra-se adequado, sobretudo quando são considerados baixos númoeros de Mach ($M$ $<$ 0,2) e baixos números de Reynolds ($Re$ $<$ 5515). Nesse sentido, há traballhos que validam, aplicam e desenvolvem metodologias de \textit{lattice} Boltzmann no campo de estudo da aeroacústica.

Um desses estudos é o de \citeonline{crouse2006fundamental}, que mostraram a eficácia do método de \textit{lattice} Boltzmann em recuperar as equações de Navier-Stokes para baixas compressibilidades ($M$ $<$ 0,3). Há de se ressaltar que validaram também o modelo numérico de um ressonador de Helmholtz com um modelo experimental do mesmo, demonstrando assim a viabilidade da aplicação para problemas de acústica.

No que se trata de desenvolvimento de ferramentas auxiliares para tratar problemas acústicos, \citeonline{kam2006non} desenvolveram uma condição de contorno absorvente, baseada na técnica de camadas perfeitamente casadas (``\textit{perfectly} \textit{matched} \textit{layers}''). Essencialmente, a técnica se baseia na criação de uma camada com viscosidade crescente exponencialmente na direção exterior do domínio computacional.

\citeonline{marie2009} analisou e comparou esquemas de alta ordem das equações de Navier-Stokes linearizadas com o método de \textit{lattice} Boltzmann. O objeto de estudo para comparação foi análises de dispersão e dissipação de ondas acústicas em regime isotérmico. Conclui-se com esse trabalho que para um erro de dispersão pré-definino, o método de \textit{lattice} Boltzmann se comportou como mais rápido.

No que diz respeito a aplicação do método de \textit{lattice} Boltzmann num problema de aeroacústica, \citeonline{lew2010noise} desenvolveram um modelo numérico em 3D para predição de ruído em um jato turbulento subsônico. Como validação os resultados foram comparados com resultados experimentais e cálculos numéricos feitos a base de \textit{Large} \textit{Eddy} \textit{Simulation} (LES)\abreviatura{LES}{\textit{Large} \textit{Eddy} \textit{Simulation}}. Esse estudo demonstrou as principais vantagens de se trabalhar com o método de \textit{lattice} Boltzmann como por exemplo o baixo custo computacional e a facilidade em inserir \textit{nozzles} com formas complexas no domínio computacional.

Também na área de aeroacústica computacional, o trabalho de \citeonline{shi2013lattice} propõe um modelo em \textit{lattice} Boltzmann para obter dados de diretividade da radiação sonora num duto circular submetido a escoamento subsônico. Os resultados de diretividade foram comparados com os modelos de \citeonline{levine1948radiation} e \citeonline{gabard2006}, mostrando uma boa convergência principalmente nas baixas frequências.

Já no sentido de tratamento de fenômenos da acústica básica, \citeonline{viggen2013acoustic} investigou os efeitos da adição de termos fontes no método de \textit{lattice} Boltzmann, mapeando eles nos parâmetros macroscópicos através da ferramenta matemática de expansão de Chapman-Enskog. Como resultado conseguiu reproduzir fenômenos de diretivade de monopolos, dipolos e quadrupolos.

\citeonline{da2015assessment} abordaram também o uso do método de \textit{lattice} Boltzmann acoplado com \textit{Large} \textit{Eddy} \textit{Simulation} (LES) na investigação do ruído gerado na interação do escoamento de um jato com uma placa plana. Os dados de níveis de pressão sonora em campo distante foram obtidos usando a superfície de Ffowcs-Williams e Hawkings (FW-H)\abreviatura{FW-H}{Superfície de Ffowcs-Williams e Hawkings} e os mesmos possuem uma boa convergência com dados experimentais.

Investigar a acústica interna de dutos circulares com escoamentos é um processo que deve ter suporte de ferramentas bem específicas, como por exemplo o método numérico de \textit{lattice} Boltzmann. Esse capítulo portanto abordará esse método e as condições de uso implementadas, validadas e verificadas num \textit{software} de código aberto chamado \citeonline{palabos}. Abordar o uso de um \textit{software} de código aberto possibilita a verificação transparente dos processos de cálculo bem como adaptações com novas implementações dentro do projeto, focando a melhor aderência da ferramenta computacional para resolução do problema.

O presente Capítulo apresenta o método de lattice Boltzmann utilizado neste trabalho e descreve a construção de um modelo tridimensional de duto não-flangeado utilizando a plataforma de código aberto Palabos. Detalhes sobre a elaboração do modelo são discutidos detalhadamente nas seções subsequentes.

\section{O Método de Lattice Boltzmann}

O método de \textit{lattice} Boltzmann possui bastante utilidade quando se trata de problemas aeroacústicos, pequenas flutuações de pressão e fenômentos de turbulência. Isso se deve ao fato do método ter surgido de uma outra abordagem de fenômenos mecânicos aplicados a fluidos - uma abordagem microscópica de interações entre moléculas.

Essa abordagem se chama Dinâmica Molecular (DM)\abreviatura{DM}{Dinâmica Molecular} e é baseada na premissa de que cada partícula pode ter suas posições no espaço e velocidades obtidas a partir da aplicação da segunda lei de Newton. Segundo essa ideia, outras propriedades do fluido como densidade, pressão e temperatura podem ser facilmente recuperadas através do cáculo da média correspondente a um conjunto de partículas. Porém o principal problema dessa abordagem é que há uma grande quantidade de equações para se resolver num pequeno volume de fluido, pois, considerando que, de acordo com o número de Avogrado, nesse mesmo volume há na ordem de $10^{23}$ moléculas para calular os movimentos cinéticos. Tal fato se torna inviável para implementação mesmo com computadores potentes como \textit{clusters} de alto desempenho.

Uma solução para contornar o problema da grande quantidade de equações do movimento é abordar o fenômeno físico pelo ponto de vista de distribuição de moléculas, a qual se convêm chamar de partícula. Nesse caso, cada partícula é descrita a partir de uma função de distribuição, a qual indica a probabilidade de se encontrar numa dada região espacial e em um determinado instante de tempo, um conjunto de moléculas que compartilham a mesma velocidade e direção de propagação. A equação de transporte que rege a propagação das partículas e a difusão da quantidade de movimento das mesmas a partir de suas colisões é a Equação de Boltzmann que, ao ser discretizada, pode ser resolvida numericamente originando assim o método de \textit{lattice} Boltzmann ou \textit{lattice} \textit{Boltzmann} \textit{Method} (LBM)\abreviatura{LBM}{\textit{Lattice} \textit{Boltzmann} \textit{Method}}. 

Historicamente o método de \textit{lattice} Boltzmann se originou no final da década de 40 para o começo da década de 50 com os trabalhos de \citeonline{grad1949kinetic} e \citeonline{bgk}. Esses trabalhos mostram que as equações de Boltzmann podem recuperar as equações macroscópicas de Navier-Stokes através da expansão de Chapman–Enskog. Nos anos 90 o trabalho de \citeonline{he1997theory} mostrou que a forma discreta da equação de Boltzmann também recupera as eaquações de Navier Stokes para baixas compressibilidades (baixos números de Mach). Isto fornece uma ligação formal entre as equações macroscópicas de lattice Boltzmann e as equações de Navier-Stokes para baixas compressibilidades, além de possibilitar a implementação computacional desse método.

O LBM possui muitas vantagens em relação a técnicas tradicionais de fluido dinâmica computacional aplicadas a aeroacústica: resolve o campo acústico e o campo fluido dinâmico numa mesma iteração em cada incremento de tempo, extração direta do campo de pressão e fácil implementação paralela elevando assim a performance frente a outros métodos.

\subsection{Modelo BGK}

O LBM, baseado em operações de colisão e propagação de funções de distribuição de partículas com massa, é a equação de Boltzman discretizada no tempo e espaço. Cada conjunto de funções de distribuição localizadas num ponto no espaço $\textbf{x}$ e tempo $t$ pode ser chamada de célula e, segundo o trabalho de \citeonline{he}, a equação de Boltzman, que formula o comportamento de cada célula, pode ser escrita na expressão 
\begin{equation}
	f_{i}(\textbf{x} + c_{i}\Delta t, t + \Delta t) = f_{i}(\textbf{x}, t) + \Omega_{i}(f(\textbf{x}, t)),
    \label{eq:f_i}
\end{equation}
sendo que $i$ é um número inteiro que delimita direções no espaço de propagação de partículas, $f_{i}$ é a função de distribuição em uma dada direção $i$, $c_{i}$ são velocidades de propagação na direção $i$ e $\Delta t$ é o incremento de tempo. 
\simbolo{$f_{i}$}{Função de distribuição LBM na direção $i$}
\simbolo{$i$}{Direção de propagação LBM}
\simbolo{$c_{i}$}{Velocidades de propagação na direção $i$}
\simbolo{$\textbf{x}$}{Localização espacial de uma célula LBM}
\simbolo{$t$}{Localização temporal de uma célula LBM}
\simbolo{$\Delta t$}{Incremento discreto de tempo}

A equação \ref{eq:f_i} é dividida nas duas operações básicas: propagação e colisão. O lado esquerdo dessa equação representa a operação de propagação, na qual os valores das funções de distribuição de cada célula são movidos para cada direção de propagação para uma próxima célula no espaço em cada iteração no tempo. Feita a operação de propagação, é realizada a operação de colisão, representada pelo lado direito da equação, na qual o termo $\Omega_{i}$\simbolo{$\Omega_{i}$}{Operador de colisão LBM} representa o operador de colisão.

Uma das formas de calcular o operador de colisão $\Omega_{i}$ é usar a formulação proposta no estudo de \citeonline{bgk}. A aplicação dessa formulação consolida o modelo BGK (Bhatnagar–Gross–Krook)\abreviatura{BGK}{Bhatnagar–Gross–Krook} ou modelo de tempo de relaxação único: \textit{single}-\textit{relaxation}-\textit{time} (SRT)\abreviatura{SRT}{\textit{single}-\textit{relaxation}-\textit{time}}. Nesse sentido, o operador de colisão é definido por
\begin{equation}
	\Omega_{i} = -\frac{1}{\tau}(f_{i} - f_{i}^{M}),
    \label{eq:omega_i}
\end{equation}
tal que $\tau$\simbolo{$\tau$}{Período de colisão LBM} é o período de colisão, período médio de colisão entre partículas, e $f_{i}^{M}$\simbolo{$f_{i}^{M}$}{Função de distribuição de Maxwell ou de equilíbrio} é a função de distribuição de Maxwell ou função de distribuição de equilíbrio.

A função de distribuição de Maxwell $f_{i}^{M}$ pode ser calculada aplicando o princípio de máxima entropia de acordo com as retrições das leis de conservação de massa e quantidade de movimento, assim como é proposto por \citeonline{wolf}. Dessa forma a função de distribuição de Maxwell é definida por
\begin{equation}
	f_{i}^{M} = \rho \varepsilon _{i}\bigg( 1 + \frac{\textbf{u}.c_{i}}{c_{s}^{2}} + \frac{\textbf{u}.c_{i}^{2} - c_{s}^{2}\textbf{u}}{2c_{s}^{4}}\bigg),
    \label{eq:f_i_M}
\end{equation}
sendo que $\rho$ é a densidade local do fluido, $\varepsilon_{i}$ são pesos de velocidades para cada direção de propagação $i$, $\textbf{u}$ é a velocidade local do fluido, $c_{i}$ é um vetor de velocidades de propagação da célula para cada direção $i$ e $c_{s}$ é a velocidade do som.
\simbolo{$\rho$}{Densidade local do fluido}
\simbolo{$\varepsilon_{i}$}{Pesos de velocidades para cada direção de propagação $i$}
\simbolo{$\textbf{u}$}{Velocidade local do fluido}
\simbolo{$c_{s}$}{Velocidade do som}

Os parâmetros macroscópicos de densidade local do fluido $\rho$ e a velocidade local do fluido $\textbf{u}$ podem ser obtidos a partir dos momentos da função de distribuição $f_{i}$ das seguintes maneiras

\begin{equation}
  \rho = \sum{f_{i}} \text{   e }
    \label{eq:rho}
\end{equation}
\begin{equation}
  \rho \textbf{u} = \sum{f_{i} c_{i}}.
    \label{eq:u}
\end{equation}

A partir da equação de estado isoentrópica linear, a pressão local do fluido $p$ pode ser obtida na forma
\begin{equation}
  p = \rho c^{2}_{s}.
    \label{eq:p}
\end{equation}

A viscosidade cinemática $\nu$ é um parâmetro que é função do período de colisão $\tau$ e pode ser obtida com a equação
\begin{equation}
	\nu = c^{2}_{s} \bigg(\tau - \frac{1}{2}\bigg).
    \label{eq:nu}
\end{equation}
\simbolo{$p$}{Pressão local do fluido}
\simbolo{$\nu$}{Viscosidade cinemática do fluido}

Quando as equações \ref{eq:rho}, \ref{eq:u}, \ref{eq:p} e \ref{eq:nu} são usadas para recuperar os atributos macroscópicos do fluido a unidade de medida não é uma unidade física. Segudo o trabalho de \citeonline{da2016prediction}, para se ter esses atributos em unidade física é preciso aplicar regras de conversão. Essas regras de conversão se baseiam em duas constantes que são definidas a partir de unidades físicas: velocidade característica definida por   
\begin{equation}
  \zeta = c^{*}/c_{s},
    \label{eq:conversao_1}
\end{equation}
em que $c^{*}$ é a velocidade física do som, e discretização $\Delta x$ definida pelo tamanho de uma célula dado em metros.

Com os parâmetros $c^{*}$ e $\Delta x$ pode-se realizar as seguintes conversões para unidades físicas, notadas com o superíndice $*$:

\begin{equation}
  \textbf{$u^{*}$} = \zeta \textbf{$u$}\text{, }
  \label{eq:conversao_2}
\end{equation}

\begin{equation}
  \textbf{$x^{*}$} = \Delta x\textbf{$x$}\text{, }
  \label{eq:conversao_3}
\end{equation}
  
\begin{equation}
  t^{*} = \frac{\Delta x}{\zeta}t \text{, }
  \label{eq:conversao_4}
\end{equation}

\begin{equation}
  \nu^{*} = \zeta \Delta x \nu\text{, }
  \label{eq:conversao_5}
\end{equation}

\begin{equation}
  \rho^{*} = \frac{\zeta}{\Delta x} \rho \text{, }
  \label{eq:conversao_6}
\end{equation}

\begin{equation}
  p^{*} = p \zeta^{2}  \rho^{*}_{0} \text{ e }
  \label{eq:conversao_7}
\end{equation}

\begin{equation}
  f^{*} = f\frac{\zeta}{\Delta x},
  \label{eq:conversao_8}
\end{equation}

tal que $f^{*}$ e $f$ são unidades de frequências física e do LBM respectivamente\simbolo{$f^{*}$}{Frequência física}\simbolo{$f$}{Frequência em LBM}.

Há várias geometrias de célula do tipo BGK, o grupo do tipo $D_{n}Q_{b}$ ($n$ dimensões e $b$ direções de propagação ou velocidades) é um dos mais usados e foi proposto por \citeonline{qian1992lattice}. A tabela \ref{table:modelos} mostra os parâmetros para cada um dos modelos do tipo $D_{n}Q_{b}$, seus diferentes vetores de velocidades de propagação ($c_{i}$), seus respectivos pesos $\varepsilon_{i}$ e as suas constantes de velocidade do som ($c_{s}$). Esses valores são obtidos para cada geometria, de forma que se mantenham a conservação da massa e da quantidade de movimento. Portanto com esses parâmetros já se torna possível calcular a função de Maxwell ($f_{i}^{M}$) para cada operação de colisão em cada iteração de tempo.

Para esse trabalho usou-se o modelo D3Q19 e a Figura \ref{fig:d3q19} ilustra um esquemático desse tipo de célula e é possível visualizar espacialmente as direções de propagação. Vale ressaltar que para cada direção há o cáculo da função de Maxwell ($f_{i}^{M}$) e, por conseguinte, a operação de propagação das funções de distribuição para a célula adjacente no sentido de cada direção.

\begin{table}[ht!]
\centering
\caption{Modelos $D_{n}Q_{b}$}
\label{table:modelos}
\begin{tabular}{|c|c|c|c|}
\hline
Modelo & $c_{i}$ & $\varepsilon_{i}$ & $c_{s}^{2}$ \\ \hline
%-----------------------------------------------------------------------------
D1Q3   & $0$,                        & $2/3$,                   & $1/3$ \\
   	   & $\pm 1$                     & $1/6$                    &     \\ \hline
%-----------------------------------------------------------------------------
   	   & $0$,                        & $6/12$,				    &  \\
D1Q5   & $\pm 1$,                    & $2/12$,			        & $1$ \\  
       & $\pm 2$                     & $1/12$			        &    \\ \hline
%-----------------------------------------------------------------------------
D2Q7   & $(0,0)$,                  & $1/2$,                    & $1/4$ \\ 
	   & $(\pm 1/2, \pm \sqrt{3}/2)$ & $1/12$                  &    \\ \hline
%-----------------------------------------------------------------------------
       & $(0,0)$,                     & $4/9$,                    &    \\
D2Q9   & $(\pm 1,0)$, $(0,\pm 1)$,    & $1/9$,                    & $1/3$   \\
   	   & $(\pm 1,\pm 1)$              & $1/36$                    &    \\ \hline
%-----------------------------------------------------------------------------
	   & $(0,0,0)$,                           & $2/9$,            &   \\ 
D3Q15  & $(\pm 1,0,0)$, $(0,\pm 1,0)$, $(0,0,\pm 1)$, & $1/9$,   & $1/3$ \\ 
	   & $(\pm 1, \pm 1,\pm 1)$               & $1/72$           &   \\ \hline
%-----------------------------------------------------------------------------
	   & $(0,0,0)$,                           & $1/3$,                        &    \\ 
D3Q19  & $(\pm 1,0,0)$, $(0,\pm 1,0)$, $(0,0,\pm 1)$, & $1/18$,               & $1/3$ \\ 
	   & $(\pm 1,\pm 1,0)$, $(\pm 1,0,\pm 1)$, $(0,\pm 1,\pm 1)$ & $1/36$,    &    \\ \hline 
\end{tabular}
\end{table}

\begin{figure}[ht!]
\centering
  \includegraphics[width=.75\linewidth]{figuras/d3q19.pdf}
  \caption[Esquemático do D3Q19]{Esquemático do modelo D3Q19. Ilustração adaptada do estudo de \citeonline{premnath2013investigation}.}
  \label{fig:d3q19}
\end{figure}

\newpage
\subsection{Múltiplos Tempos de Relaxação}

A equação \ref{eq:omega_i} retrata um operador de colisão com tempo de relaxação único para todas as direções de propagação $i$. Essa abordagem é funcional, porém limitada à estabilidade em baixos números de Reynolds como mostra o estudo de \citeonline{lallemand2000theory}. Para esses tipos de problemas a abordagem de múltiplos tempos de relaxação (MRT)\abreviatura{MRT}{\textit{multiple}-\textit{relaxation}-\textit{time}}, pode ser usada assim como é mostrado nos estudos de \citeonline{viggen2014lattice}.

Seguindo a formulação proposta por \citeonline{d1994generalized}, a formulação de MRT se baseia na troca do parâmetro de único tempo de relaxação $\tau$ por uma matriz \textbf{$\Lambda$} de vários tempos de relaxação. Todavia a matriz \textbf{$\Lambda$} é construída de acordo com uma matriz \textbf{$M$} que projeta as funções de distribuição $f_{i}$ e $f_{i}^{M}$ no espaço dos momentos. De acordo com \citeonline{lallemand2000theory}, a possibilidade desse método ser mais estável é oriunda da capacidade de operar a colisão das células com um tempo de relaxação apropriado para cada um dos vários momentos, projetados a partir das funções de distribuição $f_{i}$ e $f_{i}^{M}$. Em vista do exposto o operador de colisão da equação \ref{eq:omega_i} se transforma em
\begin{equation}
	\Omega_{i} = -\textbf{$\Lambda$}(f_{i} - f_{i}^{M}).
    \label{eq:MRT_1}
\end{equation}
Porém a operação de colisão é realizada no espaço dos momentos. Logo é preciso projetar $f_{i}$ e $f_{i}^{M}$ no espaço dos momentos impondo
\begin{equation}
	m_{i} = \textbf{$M$}f_{i} \text{ e } m_{i}^{M} = \textbf{$M$}f_{i}^{M},
    \label{eq:MRT_2}
\end{equation}
tal que, segundo \citeonline{d1994generalized} e para o caso do modelo D3Q19, a matriz \textbf{$M$} assume a forma

% \begin{equation}
%   \tiny {\begin{bmatrix}
% a & b & t \\
% \end{bmatrix}  }
% \end{equation}

\setcounter{MaxMatrixCols}{19}
\begin{equation}
\textbf{$M$} = 
\tiny{\begin{bmatrix}
A &   A &   A &   A &   A &   A &   A &   A &   A &   A &   A &   A &   A &   A &   A &   A &   A &   A &   A \\ 
 B & C & C & C & C & C & C &   D &   D &   D &   D &   D &   D &   D &   D &   D &   D &   D &   D \\ 
  E &  F &  F &  F &  F &  F &  F &   A &   A &   A &   A &   A &   A &   A &   A &   A &   A &   A &   A \\ 
   G &   A &  J &   G &   G &   G &   G &   G &   G &   G &   G &  J &   A &  J &   A &  J &   A &  J &   A \\ 
   G &  F &   K &   G &   G &   G &   G &   G &   G &   G &   G &  J &   A &  J &   A &  J &   A &  J &   A \\ 
   G &   G &   G &   A &  J &   G &   G &   A &  J &  J &   A &   G &   G &   G &   G &   A &  J &  J &   A \\ 
   G &   G &   G &  F &   K &   G &   G &   A &  J &  J &   A &   G &   G &   G &   G &   A &  J &  J &   A \\ 
   G &   G &   G &   G &   G &   A &  J &   A &  J &   A &  J &  J &   A &   A &  J &   G &   G &   G &   G \\ 
   G &   G &   G &   G &   G &  F &   K &   A &  J &   A &  J &  J &   A &   A &  J &   G &   G &   G &   G \\ 
   G &   I &   I &  J &  J &  J &  J &  H &  H &  H &  H &   A &   A &   A &   A &   A &   A &   A &   A \\ 
   G &  F &  F &   I &   I &   I &   I &  H &  H &  H &  H &   A &   A &   A &   A &   A &   A &   A &   A \\ 
   G &   G &   G &   A &   A &  J &  J &   G &   G &   G &   G &  J &  J &  J &  J &   A &   A &   A &   A \\ 
   G &   G &   G &  H &  H &   I &   I &   G &   G &   G &   G &  J &  J &  J &  J &   A &   A &   A &   A \\ 
   G &   G &   G &   G &   G &   G &   G &   G &   G &   G &   G &   G &   G &   G &   G &  J &  J &   A &   A \\ 
   G &   G &   G &   G &   G &   G &   G &   A &   A &  J &  J &   G &   G &   G &   G &   G &   G &   G &   G \\ 
   G &   G &   G &   G &   G &   G &   G &   G &   G &   G &   G &   A &   A &  J &  J &   G &   G &   G &   G \\ 
   G &   G &   G &   G &   G &   G &   G &   G &   G &   G &   G &   A &  J &   A &  J &  J &   A &  J &   A \\ 
   G &   G &   G &   G &   G &   G &   G &   A &  J &  J &   A &   G &   G &   G &   G &  J &   A &   A &  J \\ 
   G &   G &   G &   G &   G &   G &   G &  J &   A &  J &   A &  J &   A &   A &  J &   G &   G &   G &   G \\
\end{bmatrix}},
\end{equation}
e as constantes dentro da matriz assumem os valores $A = 1$, $B = -30$, $C = -11$, $D = 8$, $E = 12$, $F = -4$, $G = 0$, $H = -2$ e $I = 2$.

Considerando que a matriz \textbf{$S$} é dada por
\begin{equation}
	\textbf{$S$} = \textbf{$M$}\textbf{$\Lambda$}\textbf{$M$}^{-1},
    \label{eq:MRT_3}
\end{equation}
o operador de colisão fica
\begin{equation}
	\Omega_{i} = -\textbf{$M$}^{-1}\textbf{$S$}(m_{i} - m_{i}^{M}).
    \label{eq:MRT_4}
\end{equation}
Inserindo a equação \ref{eq:MRT_4} na equação \ref{eq:f_i} fica
\begin{equation}
	f_{i}(\textbf{x} + c_{i}\Delta t, t + \Delta t) = f_{i}(\textbf{x}, t) -\textbf{$M$}^{-1}\textbf{$S$}(m_{i} - m_{i}^{M}).
    \label{eq:MRT_5}
\end{equation}
Vale ressaltar que a operação de propagação, lado esquerdo da equação \ref{eq:MRT_5}, ocorre no espaço original da função de distribuição $f_{i}$.


\subsection{Condições de Contorno}

\subsubsection{\textit{Bounceback}}

De acordo com o estudo de \citeonline{viggen2014lattice}, a condição de contorno do tipo \textit{bounceback} tem como objetivo simular uma parede rígida no domínio do LBM, sendo ela do tipo implícita e localizada entre as células. Há dois tipos de \textit{bounceback}: \textit{free-slip}, que simula escorregamento livre do fluido na parede e \textit{no-slip}, que impõe que todas as componentes de velocidade juto à parede sejam nulas. Essa condição força o desenvolvimento da camada limite viscosa junto à parede. Nesse trabalho foi usado o do tipo \textit{no-slip}para que os efeitos de camada limite envolvidos no problema fossem capturados.

A condição de contorno \textit{bounceback} \textit{no-slip} é geralmente implementada na etapa de propagação a partir de uma inversão de funções de distribuição de partículas. A Figura \ref{fig:bounceback} mostra um esquemático de exemplo do processo de funcionamento dessa condição de contorno. Ao cruzar a condição de contorno no tempo seguinte $t + \Delta t$, a célula inverte as funções de distribuição de partículas para o sentido contrário dos vetores que apontam para o \textit{bounceback}.

% \begin{figure}[ht!]
% \centering
%   \includegraphics[width=.65\linewidth]{figuras/bounceback.pdf}
  
%   \label{fig:bounceback}
% \end{figure}

\begin{figure}[ht!]
  \centering
  \def\svgwidth{200pt}
  \import{figuras/}{bounceback_2.pdf_tex}
  \caption[Processo de funcionamento do \textit{bounceback} \textit{no-slip}]{Esquemático de exemplo do processo de funcionamento da condição de contorno \textit{bounceback} \textit{no-slip}. Ilustração adaptada do estudo de \citeonline{viggen2014lattice}.}
  \label{fig:bounceback}
\end{figure}

\newpage
Em relação às equações de propagação o processo abordado fica

\begin{gather*}
  f_{6}(\textbf{x}, t + \Delta t) = f_{8}(\textbf{x}, t)\text{,  } f_{8}(\textbf{x}, t + \Delta t) = f_{6}(\textbf{x}, t), \\
  f_{2}(\textbf{x}, t + \Delta t) = f_{4}(\textbf{x}, t)\text{,  } f_{4}(\textbf{x}, t + \Delta t) = f_{2}(\textbf{x}, t) \text{ e }\\
  f_{5}(\textbf{x}, t + \Delta t) = f_{7}(\textbf{x}, t)\text{,  } f_{7}(\textbf{x}, t + \Delta t) = f_{5}(\textbf{x}, t).
\label{eq:bounceback}
\end{gather*}

\subsubsection{Condição Anecóica}

Consolidar uma condição do tipo anecóica num método numérico de natureza temporal é um desafio. Nesse contexto, considerando a absorção de pressão, entropia e pulsos de despredimento de vórtices, o trabalho de \citeonline{kam2006non} propõe uma condição de contorno explícita de absorção. Em essência, este método se baseia na adaptação do método das camadas perfeitamente casadas (``\textit{perfectly} \textit{matched} \textit{layers}") para o LBM. A condição de contorno de absorção, \textit{Absorbing} \textit{Boundary} \textit{Condition} (ABC)\abreviatura{ABC}{\textit{Absorbing} \textit{Boundary} \textit{Condition}}, consiste na adição de uma região de amortecimento para que os valores de pressão e velocidade convirjam assintoticamente a valores que caracterizam um fluido em repouso. Nesse sentido, valores alvos para um fluido em resposo de densidade ($\rho_{T}$ $=$ $\rho_{0}$) e velocidade (\textbf{$u_{T}$} $=$ $0$) são usados para calcular uma função de distribuição de amortecimento $f_{i}^{T}$\simbolo{$f_{i}^{T}$}{Função de distribuição de amortecimento}. Essa função de distribuição é definida da mesma forma que $f_{i}^{M}$, porém com os valores alvos de densidade e velocidade, impostos na forma
\begin{equation}
  f_{i}^{T} = \rho_{0}\varepsilon_{i}.
  \label{eq:f_alvo}
\end{equation}
Como essa técnica é explícita, o operador de colisão $\Omega_{i}$ é adaptado e recebe um novo termo de colisão, tal que
\begin{equation}
  \Omega_{i} = -\frac{1}{\tau}(f_{i} - f_{i}^{M}) - \sigma(f_{i}^{M} - f_{i}^{T}),
  \label{eq:omega_i_abc}
\end{equation}
sendo $\sigma$ $=$ $\sigma_{T}(\delta/D)^{2}$ é o coeficiente de absorção, $\sigma_{T}$ é uma constante com valor de $0,3$, $\delta$ é a distância medida do começo da região de contorno no sentido da convergência assintótica e $D$ é o tamanho total da região de contorno no sentido da convergência assintótica assim como ilustra a Figura \ref{fig:abc}.

\begin{figure}[ht!]
\centering
  \includegraphics[width=.3\linewidth]{figuras/abc.pdf}
  \caption[Processo de funcionamento da condição de contorno anecóica]{Esquemático de exemplo do processo de funcionamento da condição de contorno anecóica. Ilustração adaptada do estudo de \citeonline{da2008numerical}.}
  \label{fig:abc}
\end{figure}

O operador de colisão da equação \ref{eq:omega_i_abc} funciona bem para o modelo SRT, porém como nesse estudo será usado o modelo MRT algumas adaptações precisam ser realizadas, pois a operação de colisão ocorre no espaço dos momentos nesse modelo. Assim como é feito nas equações \ref{eq:MRT_2} deve-se aplicar o mesmo procedimento na função de distribuição $f_{i}^{T}$ originando o termo $m_{i}^{T}$. Além disso é preciso inserir esse termo no operador de colisão da equação \ref{eq:MRT_4} resultando em 

\begin{equation}
  \Omega_{i} = -\textbf{$M$}^{-1}\textbf{$S$}(m_{i} - m_{i}^{M}) - 
  \sigma\textbf{$M$}^{-1}\textbf{$S$}(m_{i}^{M} - m_{i}^{T}).
  \label{eq:abc_mrt_2}
\end{equation}

Simplificando, a equação \ref{eq:abc_mrt_2} fica
\begin{equation}
  \Omega_{i} = -\textbf{$M$}^{-1}\textbf{$S$}[m_{i} - m_{i}^{M}(\sigma - 1) - m_{i}^{T}].
  \label{eq:abc_mrt_3}
\end{equation}

Adicionando esse termo na equação geral \ref{eq:f_i} do LBM o resultado é a equação
\begin{equation}
  f_{i}(\textbf{x} + c_{i}\Delta t, t + \Delta t) = f_{i}(\textbf{x}, t) -\textbf{$M$}^{-1}\textbf{$S$}[m_{i} - m_{i}^{M}(\sigma - 1) - m_{i}^{T}].
  \label{eq:abc_mrt_4}  
\end{equation}

A equação \ref{eq:abc_mrt_4} descreve a camada anecóica inserida ao redor do domínio computacional com o propósito de absorver toda energia acústica incidente, assim como o fluxo de massa proveniente do jato.

\section{Palabos}

O \textit{software} livre Palabos é um projeto feito no paradígma computacional de orientação a objetos, resultado da colaboração entre indústria e academia, focando produzir uma ferramenta de simulação computacional robusta, rápida e confiável. Junto com esse pacote computacional há implementados modelos numéricos de publicações e \textit{benchmarks} da litetura como mostra os estudos de \citeonline{lattice_1}, \citeonline{lattice_2}, \citeonline{lattice_3}, \citeonline{lattice_4} e \citeonline{lattice_5}. As seguintes funcionalidades do \textit{software} Palabos foram usadas nesse trabalho:

\begin{itemize}
  \item modelo base:
  \begin{itemize}
    \item MRT;
  \end{itemize}
  \item condição de contorno:
  \begin{itemize}
    \item \textit{bounceback} \textit{no-slip};
  \end{itemize}
  \item grid:
  \begin{itemize}
    \item D3Q19;
  \end{itemize}
  \item paralelismo:
  \begin{itemize}
    \item MPI em vários processadores, aumentando o desempenho computacional no mínimo 20 vezes mais que o MATLAB de acordo com \citeonline{numeric_palabos};
  \end{itemize}
  \item dados de saída:
  \begin{itemize}
    \item arquivos de dados em ASCII;
    \item arquivos de dados em formato de imagem GIF;
    \item arquivos de dados em formato VTK para visualização no \textit{software} \citeonline{paraview}.
  \end{itemize}
\end{itemize}

Mesmo com várias funcionalidades citadas, o \textit{software} \citeonline{palabos} precisa ter outras outras funcionalidades implementadas para que possa atender o escopo desse trabalho. Para atender esse requisito, o projeto \citeonline{palabos_acoustic} foi criado como uma versão do \citeonline{palabos} que contém todos os modelos e implementações desenvolvidas nesse trabalho. As funcionalidades desenvolvidas nesse trabalho são:

\begin{itemize}
  
  \item condições de contorno:
  \begin{itemize}
    \item condição de contorno anecóica de \citeonline{kam2006non} para BGK D2Q9;
    \item condição de contorno anecóica de \citeonline{kam2006non} para MRT D2Q9 e D3Q19;
    \item condição de contorno para excitação do duto com \textit{sweep} ou soma de harmônicos.
  \end{itemize}

  \item geração de malha:
  \begin{itemize}
    \item criação automática de malha com vários tamanhos e espessuras de dutos.
  \end{itemize}

  \item dados de saída:
  \begin{itemize}
    \item relatórios automáticos de execução;
    \item dados e relatórios de execução organizados automaticamente por pastas com hora e data.
  \end{itemize}
\end{itemize}

\section{Modelo Numérico}

Com os arquivos de compilação e execução corretamente configurados, pode-se modelar numericamente o problema. A Figura \ref{fig:modelo} representa a vista do corte lateral do modelo numérico tridimensional. Para a definição do domínio foi utilizado uma abordagem paramétrica, ou seja, o raio externo do duto $a$ $=$ $20$ células foi a unidade de medida para as dimensões. As dimensões \textbf{Nx} e \textbf{Ny} são iguais e possuem $20a$ de comprimento. A dimensão \textbf{Nz} possui 79,5$a$ de comprimento e foi baseada no estudo de \citeonline{allam2006investigation}, que justifica a distância da saída do duto até a parede para minimizar os efeitos da interação do jato de saída com a parede. Todo espaço de fluido do domínio foi preenchido em cada célula com frequência de relaxação $\frac{1}{\tau}$ $=$ $1,99$, $\rho$ $=$ $\rho_{0}$ $=$ $1$ e as velocidades para todos os sentidos $u_{x}$ $=$ $u_{y}$ $=$ $u_{z}$ $=$ $0$. As bordas do duto foram preenchidas com condição anecóica de espessura igual 1,5$a$ células.      

Com relação ao duto, o mesmo possui o tamanho \textbf{L} $=$ $18a$ e é delimitado pela condição de contorno \textit{bounceback} \textit{no-slip}, diâmetro externo medindo $2a$ e parede com expessura de 0,1$a$. No começo do duto há uma condição anecóica com espessura igual a 1,5$a$, que é responsável pela dissipação da onda no sentido contrário a saída. Ao lado da condição anecóica há uma condição de contorno de excitação do duto com espessura de 0,05$a$, responsável por excitar os modos axiais e impor escoamento.    

% \begin{figure}[ht!]
%   \centering
%   \def\svgwidth{400pt}
%   \import{figuras/}{modelo_numerico.pdf_tex}
%   \caption[Esquemático do modelo numérico]{Esquemático do modelo numérico: vista do corte lateral do modelo em 3D.}
%   \label{fig:modelo}
% \end{figure}

\begin{figure}[ht!]
\centering
  \includegraphics[width=1.\linewidth]{figuras/modelo_numerico_2.pdf}
  \caption[Esquemático do modelo numérico]{Esquemático do modelo numérico: vista do corte lateral do modelo em 3D.}
  \label{fig:modelo}
\end{figure}


Focando propiciar energia suficiente nos modos axiais com onda plana, a condição de excitação foi desenvolvida através de uma soma de ondas estacionárias, na faixa de frequência $0$ $<$ $ka$ $\leq$ 2,5. Dessa forma, os valores de densidade e velocidade dessa região foram mudados em cada incremento de tempo da seguinte forma para:
\begin{itemize}
  \item regime transiente ($0 \leq t < t_{transiente}$):
  \begin{gather*}
    \rho(t) = \rho_{0};% + A\sum_{n=1}^{N} sin\bigg(\frac{nka_{max}c_{s}t}{Na}\bigg);
    \\ u_{z}(t) = Mc_{s};    
    \\ u_{y}(t) = 0;
    \\ u_{x}(t) = 0.
  \label{eq:transiente}
  \end{gather*}

  \item regime estacionário ($t_{transiente} \leq t \leq t_{total} - t_{propagada}$):
  \begin{gather*}
    \rho(t) = \rho_{0} + A\sum_{n=1}^{N} sin\bigg(\frac{nka_{max}c_{s}t}{Na}\bigg);
    \\ u_{z}(t) = Mc_{s} + \frac{Ac_{s}}{\rho_{0}}\sum_{n=1}^{N} sin\bigg(\frac{nka_{max}c_{s}t}{Na}\bigg);    
    \\ u_{y}(t) = 0;
    \\ u_{x}(t) = 0.
  \label{eq:estacionario}
  \end{gather*}
\end{itemize}
tal que $ka_{max} =$ 2,5, $N$ é o número total de ondas estacionárias dentro do intervalo $0$ $<$ $ka$ $\leq$ 2,5, $n$ é uma onda estacionária pertencente a esse mesmo intervalo de frequências, $t_{transiente}$ é baseado e adaptado do estudo de \citeonline{shi2013lattice} na forma $t_{transiente} =  2\textbf{Nz}/Mc_{s}$, $t_{propagada}$ é definido como $t_{propagada} = \textbf{Nz}/c_{s}$ e é o tempo que a onda demora para percorrer o domínio completo na direção axial do duto, $t_{total} = t_{transiente} + t_{propagada} + 12000$ é o tempo total da simulação e $A$ é amplitude máxima definida como densidade e é calculada a partir de uma pressão física na forma
\begin{equation}
  A = \frac{2 \cdot 10^{-5} \cdot 10^{\text{NPS}/20}}{c^{*}\rho^{*}_{0}c_{s}},
\end{equation}
tal que $c^{*} = 343$ $m/s$ é a velocidade do som em unidades físicas, $\rho^{*}_{0} =$ 1,22 $kg/m^{3}$ é a densidade física do ar em unidades físicas e NPS é o nível de pressão sonora no valor de 80 dB.

No intuito de avaliar a condição anecóica nas fronteiras do domínio através do cálculo e análise do coeficiente de reflexão, os pontos $\textbf{A}$, $\textbf{B}$, $\textbf{C}$ e $\textbf{D}$ representados na Figura \ref{fig:modelo} são pontos de medição de pressão e velocidades uma célula ao lado das condições anecóicas. O localização dos pontos segue as seguintes coordenadas:

\begin{itemize}
  \item ponto $\textbf{A}$: $(\frac{\textbf{Nx}}{2}, \frac{\textbf{Ny}}{2}, \textbf{Nz} - 31)$;
  \item ponto $\textbf{B}$: $(\frac{\textbf{Nx}}{2}, \textbf{Ny} - 31, \textbf{Nz} - 31)$;
  \item ponto $\textbf{C}$: $(\frac{\textbf{Nx}}{2}, \textbf{Ny} - 31, \frac{\textbf{Nz}}{2})$;
   \item ponto $\textbf{D}$: $(\frac{\textbf{Nx}}{2}, \frac{3\textbf{Ny}}{4}, 12a + 31)$.
\end{itemize}
 
Já o ponto $\textbf{P}$ representa a média espacial, feita no plano transversal do duto, dos valores de pressão e velocidade na terminação. Essas médias espaciais são extraídas e calculadas ao longo do tempo para se obter os parâmetros de caracterização da acústica interna do duto: coeficiente de reflexão $R_{r}$ e coeficiente de correção da terminação $l/a$.

Para a execução do modelo numérico foi escolhido um \textit{hardware} com as seguintes características:

\begin{itemize}
  \item arquitetura: x86\_64;
  \item CPU(s): 8;
  \item modelo do processador: Intel(R) Xeon(R) CPU E5620 @2.40GHz;
  \item memória RAM: 139 GB.
\end{itemize}

%\newcommand{\MyNumberA}{30}

%\newcommand{\MyNumberB}{60}

%\MyNumberB

%\numexpr (\MyNumberA + 2* \MyNumberN)/3 \relax
%\the\numexpr (\MyNumberA * \MyNumberB)

\section{Pós-processamento}

Com os arquivos de dados temporais dos pontos $\textbf{A}$, $\textbf{B}$, $\textbf{C}$, $\textbf{D}$ e da média espacial $\textbf{P}$ salvos em disco rígido, um \textit{script} de pós-processamento da plataforma \citeonline{matlab}/\citeonline{octave} é executado. Os seguintes procedimentos são realizados no \textit{script}:

\begin{enumerate}
  \item os vetores temporais de pressão e velocidade no eixo axial são obtidos através da leitura de arquivos \textbf{.dat};
  \item uma janela Hann na forma
  \begin{equation}
    w(n) = sin^{2}\bigg(\frac{\pi n}{N - 1} \bigg),  
  \end{equation}
  tal que $N$ é o tamanho da janela e $n$ é a posição do vetor unidimensional é definida e usada para multiplar os sinais de velocidade e pressão no domínio do tempo;

  \item a transformada rápida de Fourier, \textit{Fast} \textit{Fourier} \textit{Transform} (FFT)\abreviatura{FFT}{\textit{Fast} \textit{Fourier} \textit{Transform}}, é aplicada nos vetores de velocidade e pressão no domínio do tempo;

  \item a impedância de radiação $Z_{r}$ é calculada através da divisão entre os vetores de pressões por de velocidades no domínio da frequência da seguinte forma:
  \begin{equation}
    Z_{r} = \frac{p(f)}{u_{z}(f)};
  \end{equation}

  \item o coeficiente de reflexão $R_{r}$ é calculado de acordo com a equação \ref{eq:R};

  \item o coeficiente de correção da terminação $l/a$ é calculado de acordo com a equação \ref{eq:l};

 \end{enumerate}

 Para minimizar os efeitos não lineares de ondas evanescentes na terminação do duto, um fator de correção $c = - 0,2367$ é adicionado na parte real do coeficiente de correção da terminação $l/a$.

Para fins de comparação dos resultados obtidos nesse estudo com resultados da literatura foi usado o coeficiente de correlação de Pearson na forma
\begin{equation}
  r = \Bigg|\frac{\sum_{n=1}^{N} (x_{n} - \overline{x})(y_{n} - \overline{y})}{\sqrt{\sum_{n=1}^{N} (x_{n} - \overline{x})} \sqrt{\sum_{n=1}^{N} (y_{n} - \overline{y})}}\Bigg| \times 100,
\end{equation}
tal que $r$ está em porcentagem e varia no intervalo [0, 100] sendo que quanto maior o valor mais correlacionado o resultado estará, $N$ é o número total de pontos, $x_{n}$ e $y_{n}$ são valores de dois conjuntos de pontos na posição $n$ a serem comparados e $\overline{x}$ e $\overline{y}$ são as médias definidas nas formas
\begin{equation}
  \overline{x} = \frac{\sum_{n=1}^{N} x_{n}}{N} \text{ e }
\end{equation}
\begin{equation}
  \overline{y} = \frac{\sum_{n=1}^{N} y_{n}}{N}. 
\end{equation}
