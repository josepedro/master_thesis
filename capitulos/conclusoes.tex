\chapter{Conclusões}

Nesse trabalho foi desenvolvida uma ferramenta computacional para análise do coeficiente de reflexão para modos normais em dutos na presença de escoamentos de baixo número de Mach ($M \leq 0,2$). 

Foi implementado um esquema computacional para avaliação do coeficiente de reflexão em dutos a partir do método de \textit{lattice} Boltzmann. Esse esquema foi desenvolvido em C++ orientado a objetos dentro do \textit{software} Palabos e fez uso do modelo MRT, condição de contorno de paredes rígidas e condição de contorno de absorção de energia acústica adaptada ao MRT. Os resultados mostraram que o esquema computacional funciona de acordo com os resultados da literatura e que a condição de absorção de energia acústica se comporta aproximadamente como impedância do meio.

Condições de contorno necessárias foram construídas, afim de representar o problema da reflexão de onda em dutos na presença de baixos números de Mach. As condições de contorno foram aplicadas num modelo numérico tridimensional de um duto não flangeado com espessura de paredes de $10 \%$ do tamanho do raio do duto. Além disso foram adaptadas as distâncias necessárias dos limites do domínio numérico em relação ao duto para que haja conservação da massa e que a condição de contorno de absorção possa se comportar regularmente. Os resultados mostraram que o modelo numérico é estável e representa o comportamento físico esperado num regime de baixos números de Mach.

Foi implementado, validado e analisado o comportamento acústico interno de dutos não flangeados com e sem escoamento de exaustão e com ondas planas. Os coeficientes de reflexão e de correção da terminação foram extraídos do modelo numérico com rotinas de pós-processamentos. Os mesmos foram comparados e analisados e possuem uma correlação em média de 90\%  com os resultados da litetura, demonstrando boa concordância com os fenômenos físicos abordados na litetura. Houve algumas divergências no coeficiente de correção da terminação num regime de exaustão e podem ser explicadas pelo fato do método de cálculo do pós-processamento não considerar a presença de escoamentos.  

O comportamento acústico interno de dutos não flangeados com escoamento sugado e com ondas planas foi implementado, validado e analisado. Os coeficientes de reflexão e de correção da terminação foram extraídos e pós-processados do modelo numérico num regime de escoamento sugado e comparado com os dados disponíveis na litetura. Apesar do coeficiente de correção da terminação não ter tido uma boa correlação, o coeficiente de reflexão foi calculado com 98,35\% de correlação demonstrando uma boa concordância com os dados da literatura. Apesar da literatura ter somente disponíveis resultados em baixas frequências ($ka \leq 0,25$) para escoamento sugado, foram calculados e analisados coeficientes de reflexão para vários números de Mach em médias e altas frequências ($ka > 0,25$). As análises demonstraram que o coeficiente de reflexão num contexto de escoamento sugado é altamente sensível a diferentes números de Mach, havendo sobretudo amplificação acima da faixa unitária para números de Strouhal $St \sim \frac{\pi}{2}$. Esse fenômeno pode ser explicado pelo fato do campo fluido dinâmico interagir com o campo acústico através de desprendimento de vórtices. Vale ressaltar também que a variação do coeficiente de reflexão em relação a vários Machs em $St \sim \frac{\pi}{2}$, diferentemente do que ocorre em regime de escoamento de exaustão que é monotônico, varia de forma não monotônica e possui um máximo em $M \sim 0,07$. Esse fenômeno pode ser explicado pela natureza do desprendimento de vórtices numa vena contracta, pois a partir do Mach $M \sim 0,07$ o desprendimento de vórtices na terminação do duto começa a ter a natureza de absorção acústica.  