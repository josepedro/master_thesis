%%%%%%%%%%%%%%%%%%%%%%%%%%%%%%%%%%%%%%%%%%%%%%%%%%%%%%%%%%%%%%%%%%%%%%%
% Universidade Federal de Santa Catarina             
% Biblioteca Universitária                     
%----------------------------------------------------------------------
% Exemplo de utilização da documentclass ufscThesis
%----------------------------------------------------------------------                                                           
% (c)2013 Roberto Simoni (roberto.emc@gmail.com)
%         Carlos R Rocha (cticarlo@gmail.com)
%         Rafael M Casali (rafaelmcasali@yahoo.com.br)
%%%%%%%%%%%%%%%%%%%%%%%%%%%%%%%%%%%%%%%%%%%%%%%%%%%%%%%%%%%%%%%%%%%%%%%
\documentclass{ufscThesis} % Definicao do documentclass ufscThesis	

%----------------------------------------------------------------------
% Pacotes usados especificamente neste documento
\usepackage{graphicx} % Possibilita o uso de figuras e gráficos
\usepackage{color}    % Possibilita o uso de cores no documento
\usepackage{pdfpages} % Possibilita a inclusão da ficha catalográfica
\usepackage{listings}
\usepackage{float}
\usepackage{fancyhdr}
\usepackage{subcaption}


%----------------------------------------------------------------------
% Comandos criados pelo usuário
\newcommand{\afazer}[1]{{\color{red}{#1}}} % Para destacar uma parte a ser trabalhada
\newcommand{\ABNTbibliographyname}{REFERÊNCIAS} % Necessário para abnTeX 0.8.2


%----------------------------------------------------------------------
% Identificadores do trabalho
% Usados para preencher os elementos pré-textuais
\instituicao[a]{Universidade Federal de Santa Catarina} % Opcional
\departamento[a]{Departamento de Engenharia Mecânica}
\curso[o]{Programa de Pós-Graduação}
\documento[o]{Dissertação} % [o] para dissertação [a] para tese
\titulo{Radiação normal de dutos com escoamento subsônico e diferentes condições de contorno}
\subtitulo{} % Opcional
\autor{José Pedro de Santana Neto}
\grau{Mestre em Engenharia Mecânica}
\local{Florianópolis} % Opcional (Florianópolis é o padrão)
\data{15}{Junho}{2016}
\orientador[Orientador]{Andrey Ricardo da Silva, Ph.D.}
%\coorientador[Coorientador]{Henrique Simas, Dr. Eng.}
\coordenador[Coordenador]{Armando Albertazzi Gonçalves Júnior, Dr. Eng.}

\numerodemembrosnabanca{3} % Isso decide se haverá uma folha adicional
\orientadornabanca{sim} % Se faz parte da banca definir como sim
\coorientadornabanca{nao} % Se faz parte da banca definir como sim
\bancaMembroA{Primeiro membro\\Universidade ...} %Nome do presidente da banca
\bancaMembroB{Segundo membro\\Universidade ...}      % Nome do membro da Banca
\bancaMembroC{Terceiro membro\\Universidade ...}     % Nome do membro da Banca
\bancaMembroD{Quarto membro\\Universidade ...}       % Nome do membro da Banca
%\bancaMembroE{Quinto membro\\Universidade ...}       % Nome do membro da Banca
%\bancaMembroF{Sexto membro\\Universidade ...}        % Nome do membro da Banca
%\bancaMembroG{Sétimo membro\\Universidade ...}       % Nome do membro da Banca

\dedicatoria{Este trabalho é dedicado aos meus colegas de classe e aos meus queridos pais.}

\agradecimento{Agradeço bla bla bla.}

\epigrafe{Texto da Epígrafe. Citação relativa ao tema do trabalho. É opcional. A epígrafe pode também aparecer na abertura de cada seção ou capítulo.}
{(Autor da epígrafe, ano)}

\textoResumo {O texto do resumo deve ser digitado, em um único bloco, sem espaço de parágrafo. O resumo deve ser significativo, composto de uma sequência de frases concisas, afirmativas e não de uma enumeração de tópicos. Não deve conter citações. Deve usar o verbo na voz passiva. Abaixo do resumo, deve-se informar as palavras-chave (palavras ou expressões significativas retiradas do texto) ou, termos retirados de thesaurus da área.}
\palavrasChave {Palavra-chave 1. Palavra-chave 2.  Palavra-chave 3. }
 
\textAbstract {Resumo traduzido para outros idiomas, neste caso, inglês. Segue o formato do resumo feito na língua vernácula. As palavras-chave traduzidas, versão em língua estrangeira, são colocadas abaixo do texto precedidas pela expressão ``Keywords'', separadas por ponto.}
\keywords {Keyword 1. Keyword 2. Keyword 3.}

%----------------------------------------------------------------------
% Início do documento                                
\usepackage{comandos}

\begin{document}
%--------------------------------------------------------
% Elementos pré-textuais
\capa  
\folhaderosto[comficha] % Se nao quiser imprimir a ficha, é só não usar o parâmetro
\folhaaprovacao
\paginadedicatoria
\paginaagradecimento
\paginaepigrafe
\paginaresumo
\paginaabstract
%\pretextuais % Substitui todos os elementos pre-textuais acima
\listadefiguras % as listas dependem da necessidade do usuário
\listadetabelas 
\listadeabreviaturas
\listadesimbolos
%%%%%%
%\f
\sumario
%--------------------------------------------------------
% Elementos textuais
%%%%%%%%%%%%%%%%%%%%%%%%%%%%%%%%%%%%%%%%%%%%%%%%%%%%%%%%%%%%%%%%%%%%%%%%

\chapter{Introdução}
\label{chapter:introdcao}

\section{Contexto}

% Falar sobre sistemas de exaustão com exemplos
Sistemas de exaustão hoje em dia possuem uma forte colaboração na composição de sons e ruídos. Escapamentos, sistemas de ventilação, buzinas e motores aeronáuticos são exemplos desses sistemas que estão altamente presentes no dia-dia. Cada vez mais a sociedade vem desenvolvendo consciência crítica dos danos que os ruídos desses tipos de sistemas podem acarretar a saúde da população. Tal fato é tão preponderante que, como é apresentado por \citeonline{munjal}, desde os anos da década de 1920 há registros de esforços para entender e caracterizar esses tipos sistemas afim de colaborar com a manutenção e desenvolvimento de ambientes saudáveis no contexto acústico.

% Falar sobre dutos
Há vários elementos estruturais que podem compor sistemas de exaustão, mas os dutos circulares se caracterizam como fundamentais e bastante presentes. Sua forma cilíndrica permite que vários fenômenos físicos possam ocorrer e interagir entre si, principalmente os fenômenos acústicos e de fluxo de massa (escoamentos). De acordo com \citeonline{munjal}, o corpo de estudos e conhecimentos da acústica interna de dutos está bem estabelecido, mas verifica-se na literatura vários questionamentos sobre o funcionamento do mesmo na presença de escoamentos (fenômenos aeroacústicos). Em vista disso, determinar a caracterização da acústica interna de dutos é de extrema importância visto as várias tecnologias relacionadas a sistemas de exaustão sem um amparo técnico bem estabelecido da literatura no ponto de vista da aeroacústica.

% Falar sobre coeficiente de reflexão e correção comprimento
Em geral, pode-se utilizar dois parâmetros para caracterizar o fenômeno da acústica interna de dutos:

\begin{itemize}
    \item a magnitude do coeficiente de reflexão $\|R\|$, razão entre as componentes refletida e incidente da onda no duto, a qual é dada por
    \begin{equation}
        R_{r} =\frac{Z_{r} - Z_{0}}{Z_{r} + Z_{0}},
        \label{eq:R}
    \end{equation}
    sendo $Z_{r}$ a impedância de radiação e $Z_{0}$ a impedância característica do meio;
    
    \item coeficiente de correção da terminação normalizado pelo raio do duto $l/a$ em que $a$ é o raio do duto. Tal parâmetro representa o comprimento acústico efetivo do duto. Em outras palavras, o fator $l$ é a quantidade adicional medida a partir da abertura do duto a qual deve propagar a onda incidente antes de ser refletida para o interior do duto com fase invertida. Tal coeficiente de correção da terminação $l$ é dado por
    \begin{equation}
        l = \frac{1}{k} \arctan\!\left(\frac{Z_{r}}{Z_{0} \, \mathrm{i}}\right)
        \label{eq:l}
    \end{equation}
    sendo $k$ o número de onda.
\end{itemize}

Com o uso desses dois parâmetros pode-se projetar dutos com um comportamento acústico adequado a diversas situações que exigem atenuação de ruídos em certas frequências, além de poder prever com mais acurácia já que grande parte dos estudos consideram a acústica interna de dutos sem escoamentos.

\section{Problema}

Com relação ao contexto abordado, a solução exata para o problema de um duto circular não flangeado na ausência de escoamento foi proposta por \citeonline{levine1948radiation}. A solução assume que a espessura das paredes do duto são desprezíveis e o fluido é inviscido. A partir destas simplificações, as expressões exatas para $\|R\|$ e $l$ são obtidas utilizando-se a técnica de Wiener-Hopf.

Apesar da utilidade do modelo de Levine e Schwinger, em boa parte das aplicações práticas, dutos circulares transportam escoamentos médios. Para tais circunstâncias, \citeonline{munt1990acoustic} propôs um modelo analítico exato, também baseado na técnica de Wiener-Hopf, em que se considera a presença de um escoamento subsônico no interior do duto. Considera-se nesse modelo as premissas de que o escoamento é uniforme, invíscido e que a camada cisalhante do jato é infinitamente fina. Além disso, o modelo considera a condição de Kutta na borda do duto para lidar com a singularidade da velocidade de partícula nesta região.

É importante ressaltar que modelos exatos para os parâmetros de radiação de dutos se limitam às condições geométricas simples. No entanto, observa-se na prática terminações cujas geometrias divergem significativamente daquela associadas a um duto não flangeado. Exemplos comuns destas geometrias são aquelas encontradas em difusores, chaminés, sistemas de exaustão, $nozzles$ e instrumentos musicais. A Figura \ref{fig:diferentes_dutos} ilustra casos mais realistas de terminação de dutos comumente encontrados na prática. Para estes casos, não existem modelos que considerem a influência do escoamento nas propriedades de radiação. Além disso, a análise numérica considerando os efeitos de escoamento não é trivial.

No entanto, com o advento de novas tecnologias computacionais, é possível realizar procedimentos numéricos extremamente complexos com certa agilidade e precisão. $Softwares$ como \citeonline{ansys} e \citeonline{comsol} possuem a viabilidade de realizar cálculos de fluido dinâmica computacional de sistemas complexos como carros e aviões. Essa capacidade técnica é oriunda em maior parte pelas tecnologias de processamento paralelo multinúcleo de processadores e implementações de seus respectivos $softwares$ gerenciadores como Open MPI \citeonline{openmpi}. Essa evolução tecnológica vem sendo essencial para o surgimento de novas ferramentas para a exploração e descoberta de fenômenos físicos, antes muitas vezes inviáveis de estudar por alto custo de bancadas experimentais ou alta complexidade na consolidação de um modelo matemático representativo.  


\section{Objetivos}

Considerando a problemática discutida acima, o objetivo principal desse trabalho é desenvolver uma ferramenta computacional para análise do comportamento acústico interno de dutos com diferentes condições de contorno na presença de escoamentos de baixo número de Mach (M $<$ 0,2).

Tem-se como objetivos específicos:
\begin{itemize}
    \item modelar e analisar o comportamento acústico de dutos não flangeados sem escoamento;
    \item modelar e analisar o comportamento acústico de dutos não flangeados com escoamento de saída;
    \item modelar e analisar o comportamento de dutos terminados por difusores do tipo corneta cilíndrica com diferentes raios e escoamento de saída;
    \item modelar e analisar o comportamento acústico interno de dutos com escoamento sugado e diferentes geometrias de terminação.
\end{itemize}

\begin{figure}[ht!]
\centering
  \includegraphics[width=.8\linewidth]{figuras/diferentes_dutos.png}
  \\
  \text{Fonte: \cite{dalmont2001radiation}}
  \captionof{figure}[Exemplos de várias termiações de dutos circulres.]{Exemplos de vários tipos de terminações: (a) flange circular; (b) flange circular com espessura do duto; (c) duto quadrado com flange de espessura quadrada; (d) flange normalizada; (e) flange esférica; (f) flange cilíndrica; (g) corneta; (h) disco não perfurado; (i) disco perfurado.}
  \label{fig:diferentes_dutos}
\end{figure} 
\chapter{Fundamentação Teórica}

\section{Método de lattice Boltzmann}



\input{capitulos/procedimentos_numericos.tex}
\chapter{Resultados}

Em vista da teoria vigente na literatura e pelo que foi exposto no ponto de vista metodológico, obteve-se resultados nos seguintes contextos:
\begin{itemize}
  \item análise da condição anecóica: a partir dos históricos temporais de pressão e velocidade de partícula nas fronteiras do modelo numérico, foram feitas análises críticas de reflexão acústica;
  \item duto sem escoamento: a partir dos históricos temporais de pressão e velocidade de partícula na terminação do duto, foram feitas validações e análises críticas dos parâmetros caracterizadores da acústica interna do duto sem escoamento;
  \item duto com escoamento de exaustão: a partir dos históricos temporais de pressão e velocidade de partícula na terminação do duto, foram feitas validações e análises críticas dos parâmetros caracterizadores da acústica interna do duto com escoamento de exaustão para regimes subsônicos ($M$ $\leq$ 0,2);
  \item duto com escoamento sugado: a partir dos históricos temporais de pressão e velocidade de partícula na terminação do duto, foram feitas validações, análises críticas e investigação dos parâmetros caracterizadores da acústica interna do duto com escoamento succionado para regimes subsônicos ($M$ $\leq$ 0,2).
\end{itemize} 

\section{Análise da Condição Anecóica}

Com a finalidade mensurar e analisar o comportamento da condição de contorno anecóica por meio de métricas numéricas e objetivas, foram calculados impedâncias e coeficientes de reflexão nas fronteiras do modelo numérico, ou seja, nos pontos \textbf{A}, \textbf{B}, \textbf{C} e \textbf{D} como é mostrado na Figura \ref{fig:modelo}. As Figuras \ref{fig:resultados_A}, \ref{fig:resultados_B}, \ref{fig:resultados_C} e \ref{fig:resultados_D} mostram os resultados nos respectivos pontos citados.


% \begin{figure}
% \begin{center}
% \begin{tikzpicture}
% \begin{axis}[
%     title={},
%     width=0.9\textwidth,
%     height=0.45\textwidth,
%     xlabel={Frequência},
%     ylabel={Sensibilidade},
%     x unit={\space Hz \space},
% 	y unit={\space C/g \space},
%     ytick=data,
%     xmin=0,
%     ymin=0,
%     ymax=0.55,
%     legend pos=north west,
%     grid=minor, % Display a grid	
% 	grid style={dashed,gray!90}, % Set the style
% 	]
%      %\addplot[color=black,dashed,thick,mark=*,mark options={solid},smooth] table[x index=2,y index=3] {Data/Kollias_SBMR_095_exp.txt}; \label{Kollias_exp095}
%      %\addlegendentry{Experimental (Kollias)}
%       %\addplot[color=blue,semithick] table[x index=0,y index=1] {Data/Kollias_SBMR_095_exp.txt}; \label{Kollias_sim095}
%     %\addlegendentry{Simulação (Kollias)}
%              \addplot[color=black, thick] table[x index=0,y index=1] {dados/coeficiente_reflexao_anecoica/A_real.txt};
%              \addplot[color=black,dashed,  thick] table[x index=0,y index=1] {dados/coeficiente_reflexao_anecoica/A_imag.txt};
%               \label{Comsol_095}% \addlegendentry{Simulação (Comsol)}
% \end{axis}
% \end{tikzpicture}
%  \caption{Resposta em carga do acelerômetro para $r$ igual a 0,95.}
% \end{center}
% \end{figure}

\newcommand\scalex{1}
\newcommand\scaley{0.9}
\newcommand\scaleA{0.5}

\begin{figure}
\begin{subfigure}{\scaleA \textwidth}
  \input{figuras/impedancia_A.tex}
\end{subfigure}%
\begin{subfigure}{\scaleA \textwidth}
  \input{figuras/reflexao_A.tex}
\end{subfigure}
\caption{Resultados da impedância (\ref{fig:A_impedancia}) e coeficiente de reflexão (\ref{fig:A_reflexao}) calculados no ponto $\textbf{A}$. Na Figura (\ref{fig:A_impedancia}) a linha contínua representa a parte real e a linha tracejada representa a parte imaginária.}
\label{fig:resultados_A}
\end{figure}

\begin{figure}
\begin{subfigure}{\scaleA \textwidth}
  \input{figuras/impedancia_B.tex}
\end{subfigure}%
\begin{subfigure}{\scaleA \textwidth}
  \input{figuras/reflexao_B.tex}
\end{subfigure}
\caption{Resultados da impedância (\ref{fig:B_impedancia}) e coeficiente de reflexão (\ref{fig:B_reflexao}) calculados no ponto $\textbf{B}$. Na Figura (\ref{fig:B_impedancia}) a linha contínua representa a parte real e a linha tracejada representa a parte imaginária.}
\label{fig:resultados_B}
\end{figure}

\begin{figure}
\begin{subfigure}{\scaleA \textwidth}
  \begin{tikzpicture}
  \begin{axis}[
	width=\scalexA \textwidth,
  height=\scaleyA \textwidth,
	xmin=0,
  xmax=2.5,
    ymin=0,
    ymax=0.7,
    ytick distance=0.1,
    xtick distance=0.5,
    grid=major, % Display a grid	
 	%grid style={dashed,gray!90}, % Set the style
	xlabel = \small{$ka$},
	ylabel = \small{$Z_{\textbf{C}}$},
   y tick label style={/pgf/number format/.cd,%
          scaled y ticks = false,
          set decimal separator={,},
          fixed},
      x tick label style={/pgf/number format/.cd,%
          scaled x ticks = false,
          set decimal separator={,},
          fixed}%
  ]
 \addplot[color=black, thick] table[x index=0,y index=1] {dados/coeficiente_reflexao_anecoica/C_real.txt};
   \addplot[color=black,dashed,  thick] table[x index=0,y index=1] {dados/coeficiente_reflexao_anecoica/C_imag.txt};
   
  \end{axis}
  \end{tikzpicture}
  \caption[Impedância $Z_{\textbf{C}}$]{}
  \label{fig:C_impedancia}
  %\caption[Impedância $Z_{\textbf{C}}$]{Resultado da impedância calculada no ponto $\textbf{C}$, próximo da condição anecóica localizada nas fronteiras do modelo numérico. A linha contínua representa a parte real e a linha tracejada representa a parte imaginária.}

\end{subfigure}%
\begin{subfigure}{\scaleA \textwidth}
  \input{figuras/reflexao_C.tex}
\end{subfigure}
\caption{Resultados da impedância (\ref{fig:C_impedancia}) e coeficiente de reflexão (\ref{fig:C_reflexao}) calculados no ponto $\textbf{C}$. Na Figura (\ref{fig:C_impedancia}) a linha contínua representa a parte real e a linha tracejada representa a parte imaginária.}
\label{fig:resultados_C}
\end{figure}

\begin{figure}
\begin{subfigure}{\scaleA \textwidth}
  \input{figuras/impedancia_D.tex}
\end{subfigure}%
\begin{subfigure}{\scaleA \textwidth}
  \input{figuras/reflexao_D.tex}
\end{subfigure}
\caption{Resultados da impedância (\ref{fig:D_impedancia}) e coeficiente de reflexão (\ref{fig:D_reflexao}) calculados no ponto $\textbf{D}$. Na Figura (\ref{fig:D_impedancia}) a linha contínua representa a parte real e a linha tracejada representa a parte imaginária.}
\label{fig:resultados_D}
\end{figure}


\newpage
\section{Duto sem Escoamento}

\begin{figure}[ht!]
\centering
  \begin{tikzpicture}
  \begin{axis}[
  width=0.9\textwidth,
  height=0.5\textwidth,
  xmin=0,
  xmax=1.8,
    ymin=0.4,
    ymax=1,
    ytick distance=0.1,
    grid=major, % Display a grid  
  %grid style={dashed,gray!90}, % Set the style
  xlabel = Número de Helmholtz ($ka$),
  ylabel = Coeficiente de Reflexão ($R_{r}$),
  ]
 \addplot[color=black, thick] table[x index=0,y index=1] {dados/duto_sem_escoamento/analytical_data_abs_r.txt};
 \addplot[color=black, mark=o, only marks] table[x index=0,y index=1] {dados/duto_sem_escoamento/simulation_data_abs_r.txt};
   
  \end{axis}
  \end{tikzpicture}
  \caption[Coeficiente de Reflexão $R_{r}$ sem Escoamento]{Resultado da magnitude do coeficiente de reflexão $R_{r}$ calculado no ponto $\textbf{P}$ na terminação do duto sem escoamento. A linha contínua representa o resultado analítico do estudo de \citeonline{levine1948radiation} e os pontos circulares representam os resultados calculados pela ferramenta computacional proposta nesse estudo. A correlação entre os resultados foi de 99,95 \%.}
  \label{fig:abs_r_boca}
\end{figure}

\newpage
\begin{figure}[ht!]
\centering
  \begin{tikzpicture}
  \begin{axis}[
  width=0.9\textwidth,
  height=0.5\textwidth,
  xmin=0,
  xmax=1.8,
    ymin=0.4,
    ymax=0.65,
    ytick distance=0.05,
    grid=major, % Display a grid  
  %grid style={dashed,gray!90}, % Set the style
  xlabel = Número de Helmholtz ($ka$),
  ylabel = Correção da Terminação ($l/a$),
  ]
 \addplot[color=black, thick] table[x index=0,y index=1] {dados/duto_sem_escoamento/analytical_data_loa.txt};
 \addplot[color=black, mark=o, only marks] table[x index=0,y index=1] {dados/duto_sem_escoamento/simulation_data_loa.txt};
   
  \end{axis}
  \end{tikzpicture}
  \caption[Coeficiente de Correção da Terminação $l/a$ sem Escoamento]{Resultado do coeficiente de correção da terminação $l/a$ calculado no ponto $\textbf{P}$ na terminação do duto sem escoamento. A linha contínua representa o resultado analítico do estudo de \citeonline{levine1948radiation} e os pontos circulares representam os resultados calculados pela ferramenta computacional proposta nesse estudo. A correlação entre os resultados foi de 96,23 \%.}
  \label{fig:loa_boca}
\end{figure}

\newpage
\section{Duto com Escoamento de Exaustão}

\subsection{Mach 0,2}
\begin{figure}[ht!]
\centering
  \begin{tikzpicture}
  \begin{axis}[
  width=0.9\textwidth,
  height=0.5\textwidth,
  xmin=0,
  xmax=1.8,
    ymin=0.4,
    ymax=1.2,
    ytick distance=0.1,
    grid=major, % Display a grid  
  %grid style={dashed,gray!90}, % Set the style
  xlabel = Número de Helmholtz ($ka$),
  ylabel = Coeficiente de Reflexão ($R_{r}$),
  ]
 \addplot[color=black, thick] table[x index=0,y index=1] {dados/duto_exaustao/abs_r_020_analytical.txt};
 \addplot[color=black, mark=o, only marks] table[x index=0,y index=1] {dados/duto_exaustao/abs_r_020_simulation.txt};
   
  \end{axis}
  \end{tikzpicture}
  \caption[Coeficiente de Reflexão $R_{r}$ com Escoamento de Exaustão (M $=$ 0,2)]{Resultado da magnitude do coeficiente de reflexão $R_{r}$ calculado no ponto $\textbf{P}$ na terminação do duto com escoamento de exaustão (M $=$ 0,2 e Re $=$ 5514,82). A linha contínua representa o resultado analítico do estudo de \citeonline{munt1990acoustic} e os pontos circulares representam os resultados calculados pela ferramenta computacional proposta nesse estudo. A correlação entre os resultados foi de 98,1 \%.}
  \label{fig:abs_r_boca_020}
\end{figure}

\newpage
\begin{figure}[ht!]
\centering
  \begin{tikzpicture}
  \begin{axis}[
  width=0.9\textwidth,
  height=0.5\textwidth,
  xmin=0,
  xmax=1.8,
    ymin=0.15,
    ymax=0.55,
    ytick distance=0.05,
    grid=major, % Display a grid  
  %grid style={dashed,gray!90}, % Set the style
  xlabel = Número de Helmholtz ($ka$),
  ylabel = Correção da Terminação ($l/a$),
  ]
 \addplot[color=black, thick] table[x index=0,y index=1] {dados/duto_exaustao/loa_020_analytical.txt};
 \addplot[color=black, mark=o, only marks] table[x index=0,y index=1] {dados/duto_exaustao/loa_020_simulation.txt};
   
  \end{axis}
  \end{tikzpicture}
  \caption[Coeficiente de Correção da Terminação $l/a$ com Escoamento de Exaustão (M $=$ 0,2)]{Resultado do coeficiente de correção da terminação $l/a$ calculado no ponto $\textbf{P}$ na terminação do duto com escoamento de (M $=$ 0,2 e Re $=$ 5514,82). A linha contínua representa o resultado analítico do estudo de \citeonline{munt1990acoustic} e os pontos circulares representam os resultados calculados pela ferramenta computacional proposta nesse estudo. A correlação entre os resultados foi de 79,84 \%.}
  \label{fig:loa_boca_020}
\end{figure}

\newpage
\subsection{Mach = 0,15}

\begin{figure}[ht!]
\centering
  \begin{tikzpicture}
  \begin{axis}[
  width=0.9\textwidth,
  height=0.5\textwidth,
  xmin=0,
  xmax=1.8,
    ymin=0.4,
    ymax=1.2,
    ytick distance=0.1,
    grid=major, % Display a grid  
  %grid style={dashed,gray!90}, % Set the style
  xlabel = Número de Helmholtz ($ka$),
  ylabel = Coeficiente de Reflexão ($R_{r}$),
  ]
 \addplot[color=black, thick] table[x index=0,y index=1] {dados/duto_exaustao/abs_r_015_analytical.txt};
 \addplot[color=black, mark=o, only marks] table[x index=0,y index=1] {dados/duto_exaustao/abs_r_015_simulation.txt};
   
  \end{axis}
  \end{tikzpicture}
  \caption[Coeficiente de Reflexão $R_{r}$ com Escoamento de Exaustão (M $=$ 0,15)]{Resultado da magnitude do coeficiente de reflexão $R_{r}$ calculado no ponto $\textbf{P}$ na terminação do duto com escoamento de exaustão (M $=$ 0,15 e Re $=$ 2057,71). A linha contínua representa o resultado analítico do estudo de \citeonline{munt1990acoustic} e os pontos circulares representam os resultados calculados pela ferramenta computacional proposta nesse estudo. A correlação entre os resultados foi de 99,80 \%.}
  \label{fig:abs_r_boca_015}
\end{figure}

\newpage
\begin{figure}[ht!]
\centering
  \begin{tikzpicture}
  \begin{axis}[
  width=0.9\textwidth,
  height=0.5\textwidth,
  xmin=0,
  xmax=1.8,
    ymin=0.15,
    ymax=0.55,
    ytick distance=0.05,
    grid=major, % Display a grid  
  %grid style={dashed,gray!90}, % Set the style
  xlabel = Número de Helmholtz ($ka$),
  ylabel = Correção da Terminação ($l/a$),
  ]
 \addplot[color=black, thick] table[x index=0,y index=1] {dados/duto_exaustao/loa_015_analytical.txt};
 \addplot[color=black, mark=o, only marks] table[x index=0,y index=1] {dados/duto_exaustao/loa_015_simulation.txt};
   
  \end{axis}
  \end{tikzpicture}
  \caption[Coeficiente de Correção da Terminação $l/a$ com Escoamento de Exaustão (M $=$ 0,15)]{Resultado do coeficiente de correção da terminação $l/a$ calculado no ponto $\textbf{P}$ na terminação do duto com escoamento de (M $=$ 0,15 e Re $=$ 2057,71). A linha contínua representa o resultado analítico do estudo de \citeonline{munt1990acoustic} e os pontos circulares representam os resultados calculados pela ferramenta computacional proposta nesse estudo. A correlação entre os resultados foi de 94,28 \%.}
  \label{fig:loa_boca_015}
\end{figure}


\newpage
\section{Duto com Escoamento Sugado}

\begin{figure}[ht!]
\centering
  \begin{tikzpicture}
  \begin{axis}[
  width=0.9\textwidth,
  height=0.5\textwidth,
  x tick label style={
      /pgf/number format/.cd,
          fixed,
          fixed zerofill,
          precision=2,
      /tikz/.cd
  },
  xmin=0,
  xmax=0.2,
  ymin=0.68,
  ymax=1,
  ytick distance=0.05,
  xtick distance=0.05,
  grid=major, % Display a grid  
  %grid style={dashed,gray!90}, % Set the style
  xlabel = Número de Mach ($M$),
  ylabel = Coeficiente de Reflexão ($R_{M}$),
  ]
 \addplot[color=black, thick] table[x index=0,y index=1] {dados/duto_sugado/davis_analytical.txt};
 \addplot[color=black, mark=o, only marks] table[x index=0,y index=1] {dados/duto_sugado/davis_simulation.txt};
   
  \end{axis}
  \end{tikzpicture}
  \caption[Coeficiente de reflexão $R_{M}$ com escoamento sugado]{Resultado do coeficiente de reflexão $R_{M}$ em relação ao Mach para baixas frequências ($ka$ $<$ $0,25$) com escoamento sugado. A linha contínua apresenta o resultado do estudo de \citeonline{davies1987} e os pontos circulares representam os resultados calculados pela ferramenta computacional proposta nesse estudo. A correlação entre os resultados foi de 95,45 \%.}

  \label{fig:abs_r_boca_sugado}
\end{figure}

\newpage
\begin{figure}[ht!]
\centering
  \begin{tikzpicture}
  \begin{axis}[
  width=0.9\textwidth,
  height=0.5\textwidth,
  x tick label style={
      /pgf/number format/.cd,
          fixed,
          fixed zerofill,
          precision=2,
      /tikz/.cd
  },
  xmin=0,
  xmax=0.2,
  ymin=0.1,
  ymax=1.0,
  ytick distance=0.1,
  xtick distance=0.05,
  grid=major, % Display a grid  
  %grid style={dashed,gray!90}, % Set the style
  xlabel = Número de Mach ($M$),
  ylabel = Correção da Terminação ($l_{M}$),
  ]
 \addplot[color=black, thick] table[x index=0,y index=1] {dados/duto_sugado/davis_analytical_loa.txt};
 \addplot[color=black, mark=o, only marks] table[x index=0,y index=1] {dados/duto_sugado/davis_simulation_loa.txt};
   
  \end{axis}
  \end{tikzpicture}
  \caption[Coeficiente de correção da terminação ($l_{M}$) com escoamento sugado]{Resultado do coeficiente de correção da terminação $l_{M}$ em relação ao Mach para baixas frequências ($ka$ $<$ $0,25$) com escoamento sugado. A linha contínua apresenta o resultado do estudo de \citeonline{davies1987} e os pontos circulares representam os resultados calculados pela ferramenta computacional proposta nesse estudo. A correlação entre os resultados foi de 62,53 \%.}

  \label{fig:abs_r_boca_sugado_loa}
\end{figure}
\chapter{Conclusões}

Nesse trabalho foi desenvolvida uma ferramenta computacional para análise do coeficiente de reflexão para modos normais em dutos na presença de escoamentos de baixo número de Mach ($M \leq 0,2$). 

Foi implementado um esquema computacional para avaliação do coeficiente de reflexão em dutos a partir do método de \textit{lattice} Boltzmann. Esse esquema foi desenvolvido em C++ orientado a objetos dentro do \textit{software} Palabos e fez uso do modelo MRT, condição de contorno de paredes rígidas e condição de contorno de absorção de energia acústica adaptada ao MRT. Os resultados mostraram que o esquema computacional funciona de acordo com os resultados da literatura e que a condição de absorção de energia acústica se comporta aproximadamente como impedância do meio.

Condições de contorno necessárias foram construídas, afim de representar o problema da reflexão de onda em dutos na presença de baixos números de Mach. As condições de contorno foram aplicadas num modelo numérico tridimensional de um duto não flangeado com espessura de paredes de $10 \%$ do tamanho do raio do duto. Além disso foram adaptadas as distâncias necessárias dos limites do domínio numérico em relação ao duto para que haja conservação da massa e que a condição de contorno de absorção possa se comportar regularmente. Os resultados mostraram que o modelo numérico é estável e representa o comportamento físico esperado num regime de baixos números de Mach.

Foi implementado, validado e analisado o comportamento acústico interno de dutos não flangeados com e sem escoamento de exaustão e com ondas planas. Os coeficientes de reflexão e de correção da terminação foram extraídos do modelo numérico com rotinas de pós-processamentos. Os mesmos foram comparados e analisados e possuem uma correlação em média de 90\%  com os resultados da litetura, demonstrando boa concordância com os fenômenos físicos abordados na litetura. Houve algumas divergências no coeficiente de correção da terminação num regime de exaustão e podem ser explicadas pelo fato do método de cálculo do pós-processamento não considerar a presença de escoamentos.  

O comportamento acústico interno de dutos não flangeados com escoamento sugado e com ondas planas foi implementado, validado e analisado. Os coeficientes de reflexão e de correção da terminação foram extraídos e pós-processados do modelo numérico num regime de escoamento sugado e comparado com os dados disponíveis na litetura. Apesar do coeficiente de correção da terminação não ter tido uma boa correlação, o coeficiente de reflexão foi calculado com 98,35\% de correlação demonstrando uma boa concordância com os dados da literatura. Apesar da literatura ter somente disponíveis resultados em baixas frequências ($ka \leq 0,25$) para escoamento sugado, foram calculados e analisados coeficientes de reflexão para vários números de Mach em médias e altas frequências ($ka > 0,25$). As análises demonstraram que o coeficiente de reflexão num contexto de escoamento sugado é altamente sensível a diferentes números de Mach, havendo sobretudo amplificação acima da faixa unitária para números de Strouhal $St \sim \frac{\pi}{2}$. Esse fenômeno pode ser explicado pelo fato do campo fluido dinâmico interagir com o campo acústico através de desprendimento de vórtices. Vale ressaltar também que a variação do coeficiente de reflexão em relação a vários Machs em $St \sim \frac{\pi}{2}$, diferentemente do que ocorre em regime de escoamento de exaustão que é monotônico, varia de forma não monotônica e possui um máximo em $M \sim 0,07$. Esse fenômeno pode ser explicado pela natureza do desprendimento de vórtices numa vena contracta.  


%%%%%%%%%%%%%%%%%%%%%%%%%%%%%%%%%%%%%%%%%%%%%%%%%%%%%%%%%%%%%%%%%%%%%%%%%


\bibliographystyle{ufscThesis/ufsc-alf}
\bibliography{bibliografia}
%%%%%%%%%%%%%%%%%%%%%	REFERÊNCIAS    %%%%%%%%%%%%%%%%%%%%%%%%%%%%%%%%%%%

%--------------------------------------------------------
% Elementos pós-textuais
%\apendice
%\chapter{Exemplificando um Apêndice}
%Texto do Apêndice aqui. 

%\anexo
%\chapter{Exemplificando um Anexo}
%Texto do anexo aqui.
\end{document}
