%%%%%%%%%%%%%%%%%%%%%%%%%%%%%%%%%%%%%%%%%%%%%%%%%%%%%%%%%%%%%%%%%%%%%%%
% Universidade Federal de Santa Catarina             
% Biblioteca Universitária                     
%----------------------------------------------------------------------
% Exemplo de utilização da documentclass ufscThesis
%----------------------------------------------------------------------                                                           
% (c)2013 Roberto Simoni (roberto.emc@gmail.com)
%         Carlos R Rocha (cticarlo@gmail.com)
%         Rafael M Casali (rafaelmcasali@yahoo.com.br)
%%%%%%%%%%%%%%%%%%%%%%%%%%%%%%%%%%%%%%%%%%%%%%%%%%%%%%%%%%%%%%%%%%%%%%%
\documentclass{ufscThesis} % Definicao do documentclass ufscThesis	

%----------------------------------------------------------------------
% Pacotes usados especificamente neste documento
%\usepackage[utf8]{inputenc}
%\usepackage[T1]{fontenc}
\usepackage{graphicx} % Possibilita o uso de figuras e gráficos
\usepackage{color}    % Possibilita o uso de cores no documento
\usepackage{pdfpages} % Possibilita a inclusão da ficha catalográfica
\usepackage{listings}
\usepackage{float}
\usepackage{fancyhdr}
\usepackage{subcaption}
\usepackage{pgfplots}
\usepgfplotslibrary{units}
\usepackage{array}
\usepackage{listings}
\usepackage{import}
\usepackage{amsmath}
\usepackage{sagetex}
\usepackage{subfig}
\usepackage{amsmath}%
\usepackage{MnSymbol}%
\usepackage{wasysym}%
\usepackage{filecontents}
\usepackage{lscape}
\usepackage[decimalsymbol=comma]{siunitx}
\pgfplotsset{width=.9\linewidth,compat=1.9}


%----------------------------------------------------------------------
% Comandos criados pelo usuário
\newcommand{\afazer}[1]{{\color{red}{#1}}} % Para destacar uma parte a ser trabalhada
\newcommand{\ABNTbibliographyname}{REFERÊNCIAS} % Necessário para abnTeX 0.8.2


%----------------------------------------------------------------------
% Identificadores do trabalho
% Usados para preencher os elementos pré-textuais
\instituicao[a]{Universidade Federal de Santa Catarina} % Opcional
\departamento[a]{Departamento de Engenharia Mecânica}
\curso[o]{Programa de Pós-Graduação}
\documento[o]{Dissertação} % [o] para dissertação [a] para tese
\titulo{Ferramenta Computacional para Análise da Acústica Interna de Dutos}
\subtitulo{} % Opcional
\autor{José Pedro de Santana Neto}
\grau{Mestre em Engenharia Mecânica}
\local{Florianópolis} % Opcional (Florianópolis é o padrão)
\data{15}{Agosto}{2017}
\orientador[Orientador]{Andrey Ricardo da Silva, Ph.D.}
%\coorientador[Coorientador]{Henrique Simas, Dr. Eng.}
\coordenador[Coordenador]{Jonny Carlos da Silva, Dr. Eng.}

\numerodemembrosnabanca{2} % Isso decide se haverá uma folha adicional
\orientadornabanca{nao} % Se faz parte da banca definir como sim
\coorientadornabanca{nao} % Se faz parte da banca definir como sim
\bancaMembroA{Júlio Apolinário Cordioli, Dr. Eng.\\Universidade Federal de Santa Catarina} %Nome do presidente da banca
\bancaMembroB{Luis Orlando Emerich dos Santos, Dr. Eng.\\Universidade Federal de Santa Catarina}      % Nome do membro da Banca
\bancaMembroC{Arcanjo Lenzi, Ph.D.\\Universidade Federal de Santa Catarina}     % Nome do membro da Banca
\bancaMembroD{Quarto membro\\Universidade ...}       % Nome do membro da Banca
%\bancaMembroE{Quinto membro\\Universidade ...}       % Nome do membro da Banca
%\bancaMembroF{Sexto membro\\Universidade ...}        % Nome do membro da Banca
%\bancaMembroG{Sétimo membro\\Universidade ...}       % Nome do membro da Banca

\dedicatoria{Este trabalho é dedicado às pessoas que possuem a estranha mania de ter fé na vida.}

\agradecimento{Agradeço primeiramente a Deus, por permitir-me nesse mundo, vivendo, aprendendo e contemplando a beleza da natureza primordial de todas as coisas.

A minha amada e querida mãe Francisca, pela paciência, compreensão, tolerância, conselhos, carinho, dedicação, afeto, amizade, silêncio, sorrisos e um intenso amor. Meu primeiro aprendizado na vida mais puro e original de amor foi através dela. Meus sinceros e eternos agradecimentos.

A meu pai Luciano, mesmo não estando presente mais, me inspirou a escolha da minha formação e me ensinou a olhar o mundo com meus próprios olhos.

A meu irmão João, companheiro e amigo de sempre. Seus conselhos e seu exemplo têm me ensinado muito a ser uma pessoa melhor.

A minha amada e querida namorada Simone, pelo companheirismo, amizade, carinho, afeto, conselhos, paciência e sobretudo muito amor. Infinitamente agradecido por tudo.  

A meu padrinho Inácio, meu tio Antônio, minha madrinha Nevinha, minhas tias Titia e Tia Marli por seus profundos conselhos sobre a vida, apoio, carinho e amor. A toda minha família pelo apoio, confiança e compressão.

A meus amigos de Brasília Thiago, Leandro, Vilmey, Yan e Henrique pelo companheirismo indescritível de muitos anos e apoio de sempre.

A meu orientador professor Andrey, pelo exemplo, inspiração, conselhos, apoio, confiança e investimento de longas conversas. Esse trabalho necessariamente foi fruto de longas horas de esforço e trabalho em conjunto.

Aos professores Júlio e Arcanjo (Chefe), pelos valiosos ensinamentos e exemplos de profissionais-cientistas.

Aos meus amigos de Florianópolis Wagner, Danilo, Matheus, Luisa, André Spillere, André Loch, Zargos, Gil, Caetano, Giordano e o pessoal do GOJ e outros que esqueci de citar pela compreensão, apoio e motivação.

A equipe do LVA e da pós-graduação em Engenharia Mecânica pelo suporte e aprendizado na produção científica.

E as pessoas que passaram na minha vida e influenciaram de alguma forma nesse trabalho. Meus agradecimentos.
}

\epigrafe{Muitos acham que o som é um corpo rígido vibrando, mas o som é fluido assim como a vida.}
{(José Pedro de Santana Neto, 2017)}

\textoResumo {A análise da irradiação sonora de dutos que transportam escoamentos ainda é um desafio,
tendo em vista a complexa interação entre o som e o campo fluido dinâmico, a qual transforma 
de maneira significativa o comportamento desses sistemas.  Neste contexto, abordagens analíticas ou
experimentais podem se tornar inviáveis devido à grande complexidade matemática ou aos altos custos de bancada. Para contornar esse problema, faz-se o uso de métodos numéricos. Nesse trabalho é desenvolvido e validado uma ferramenta computacional para análise do coeficiente de reflexão de modos normais em dutos na presença de escoamentos de baixa compressibilidade, limitados pelo número de Mach $= 0,2$. Para tanto, utilizou-se o método de \textit{lattice} Boltzmann, a partir do qual foram implementadas diferentes condições de contorno necessárias para a captura dos fenômenos fundamentais envolvidos no problema. Para a implementação computacional foi utilizado o \textit{software} livre Palabos e nele foi constituído um modelo numérico tridimensional de um duto não flangeado. Nesse modelo numérico foram abordados condições sem escoamento, com escoamento de exaustão e com escoamento succionado para validar a ferramenta computacional, resultando em boas concordâncias com a literatura vigente, além de propiciar mais informações sobre regimes de sucção no que diz respeito a interação de vórtices com o campo acústico.}
\palavrasChave {Aeroacústica. Ferramenta Computacional. Acústica Interna de Dutos. Método de \textit{lattice} Boltzmann. Palabos. Coeficiente de Reflexão.}
 
\textAbstract {The analysis of the sound irradiation of ducts that transport flows is still a challenge, considering the complex interaction between the sound and the dynamic fluid field, which significantly transforms the behavior of these systems. In this context, analytical or experimental approaches may become impractical due to the great mathematical complexity or high experimental costs. To overcome this problem, numerical methods are used. In this work, a computational tool for analysis of the reflection coefficient of normal modes in ducts in the presence of low compressibility flows, limited by the number of Mach $ = 0,2 $, is developed and validated. In order to do so, the lattice Boltzmann method was used, that differents boundary conditions were implemented to capture the fundamental phenomena involved in the problem. For the computational implementation, the free software Palabos was used and a three-dimensional numerical model of a non-flanged duct was used. In this numerical model, conditions with no flow, exhaust flow and suction flow were approached to validate the computational tool, resulting in good agreements with the current literature, providing more informations about suction regimes with respect to the interaction of vortices with the acoustic field.}
\keywords {Aeroacoustics. Computational Tool. Internal Acoustics of Pipelines. Lattice Boltzmann Method. Palabos. Coefficient of Reflection.}

%----------------------------------------------------------------------
% Início do documento                                
\usepackage{comandos}

\begin{document}
%--------------------------------------------------------
% Elementos pré-textuais
\capa  
\folhaderosto[comficha] % Se nao quiser imprimir a ficha, é só não usar o parâmetro
\folhaaprovacao
\paginadedicatoria
\paginaagradecimento
\paginaepigrafe
\paginaresumo
\paginaabstract
%\pretextuais % Substitui todos os elementos pre-textuais acima
\listadefiguras % as listas dependem da necessidade do usuário
\listadetabelas 
\listadeabreviaturas
\listadesimbolos
%%%%%%
%\f
\sumario
%--------------------------------------------------------
% Elementos textuais
%%%%%%%%%%%%%%%%%%%%%%%%%%%%%%%%%%%%%%%%%%%%%%%%%%%%%%%%%%%%%%%%%%%%%%%%

\chapter{Introdução}
\label{chapter:introdcao}

\section{Contexto}

% Falar sobre sistemas de exaustão com exemplos
Sistemas de exaustão hoje em dia possuem uma forte colaboração na composição de sons e ruídos. Escapamentos, sistemas de ventilação, buzinas e motores aeronáuticos são exemplos desses sistemas que estão altamente presentes no dia-dia. Cada vez mais a sociedade vem desenvolvendo consciência crítica dos danos que os ruídos desses tipos de sistemas podem acarretar a saúde da população. Tal fato é tão preponderante que, como é apresentado por \citeonline{munjal}, desde os anos da década de 1920 há registros de esforços para entender e caracterizar esses tipos sistemas afim de colaborar com a manutenção e desenvolvimento de ambientes saudáveis no contexto acústico.

% Falar sobre dutos
Há vários elementos estruturais que podem compor sistemas de exaustão, mas os dutos circulares se caracterizam como fundamentais e bastante presentes. Sua forma cilíndrica permite que vários fenômenos físicos possam ocorrer e interagir entre si, principalmente os fenômenos acústicos e de fluxo de massa (escoamentos). De acordo com \citeonline{munjal}, o corpo de estudos e conhecimentos da acústica interna de dutos está bem estabelecido, mas verifica-se na literatura vários questionamentos sobre o funcionamento do mesmo na presença de escoamentos (fenômenos aeroacústicos). Em vista disso, determinar a caracterização da acústica interna de dutos é de extrema importância visto as várias tecnologias relacionadas a sistemas de exaustão sem um amparo técnico bem estabelecido da literatura no ponto de vista da aeroacústica.

% Falar sobre coeficiente de reflexão e correção comprimento
Em geral, pode-se utilizar dois parâmetros para caracterizar o fenômeno da acústica interna de dutos:

\begin{itemize}
    \item a magnitude do coeficiente de reflexão $\|R\|$, razão entre as componentes refletida e incidente da onda no duto, a qual é dada por
    \begin{equation}
        R_{r} =\frac{Z_{r} - Z_{0}}{Z_{r} + Z_{0}},
        \label{eq:R}
    \end{equation}
    sendo $Z_{r}$ a impedância de radiação e $Z_{0}$ a impedância característica do meio;
    
    \item coeficiente de correção da terminação normalizado pelo raio do duto $l/a$ em que $a$ é o raio do duto. Tal parâmetro representa o comprimento acústico efetivo do duto. Em outras palavras, o fator $l$ é a quantidade adicional medida a partir da abertura do duto a qual deve propagar a onda incidente antes de ser refletida para o interior do duto com fase invertida. Tal coeficiente de correção da terminação $l$ é dado por
    \begin{equation}
        l = \frac{1}{k} \arctan\!\left(\frac{Z_{r}}{Z_{0} \, \mathrm{i}}\right)
        \label{eq:l}
    \end{equation}
    sendo $k$ o número de onda.
\end{itemize}

Com o uso desses dois parâmetros pode-se projetar dutos com um comportamento acústico adequado a diversas situações que exigem atenuação de ruídos em certas frequências, além de poder prever com mais acurácia já que grande parte dos estudos consideram a acústica interna de dutos sem escoamentos.

\section{Problema}

Com relação ao contexto abordado, a solução exata para o problema de um duto circular não flangeado na ausência de escoamento foi proposta por \citeonline{levine1948radiation}. A solução assume que a espessura das paredes do duto são desprezíveis e o fluido é inviscido. A partir destas simplificações, as expressões exatas para $\|R\|$ e $l$ são obtidas utilizando-se a técnica de Wiener-Hopf.

Apesar da utilidade do modelo de Levine e Schwinger, em boa parte das aplicações práticas, dutos circulares transportam escoamentos médios. Para tais circunstâncias, \citeonline{munt1990acoustic} propôs um modelo analítico exato, também baseado na técnica de Wiener-Hopf, em que se considera a presença de um escoamento subsônico no interior do duto. Considera-se nesse modelo as premissas de que o escoamento é uniforme, invíscido e que a camada cisalhante do jato é infinitamente fina. Além disso, o modelo considera a condição de Kutta na borda do duto para lidar com a singularidade da velocidade de partícula nesta região.

É importante ressaltar que modelos exatos para os parâmetros de radiação de dutos se limitam às condições geométricas simples. No entanto, observa-se na prática terminações cujas geometrias divergem significativamente daquela associadas a um duto não flangeado. Exemplos comuns destas geometrias são aquelas encontradas em difusores, chaminés, sistemas de exaustão, $nozzles$ e instrumentos musicais. A Figura \ref{fig:diferentes_dutos} ilustra casos mais realistas de terminação de dutos comumente encontrados na prática. Para estes casos, não existem modelos que considerem a influência do escoamento nas propriedades de radiação. Além disso, a análise numérica considerando os efeitos de escoamento não é trivial.

No entanto, com o advento de novas tecnologias computacionais, é possível realizar procedimentos numéricos extremamente complexos com certa agilidade e precisão. $Softwares$ como \citeonline{ansys} e \citeonline{comsol} possuem a viabilidade de realizar cálculos de fluido dinâmica computacional de sistemas complexos como carros e aviões. Essa capacidade técnica é oriunda em maior parte pelas tecnologias de processamento paralelo multinúcleo de processadores e implementações de seus respectivos $softwares$ gerenciadores como Open MPI \citeonline{openmpi}. Essa evolução tecnológica vem sendo essencial para o surgimento de novas ferramentas para a exploração e descoberta de fenômenos físicos, antes muitas vezes inviáveis de estudar por alto custo de bancadas experimentais ou alta complexidade na consolidação de um modelo matemático representativo.  


\section{Objetivos}

Considerando a problemática discutida acima, o objetivo principal desse trabalho é desenvolver uma ferramenta computacional para análise do comportamento acústico interno de dutos com diferentes condições de contorno na presença de escoamentos de baixo número de Mach (M $<$ 0,2).

Tem-se como objetivos específicos:
\begin{itemize}
    \item modelar e analisar o comportamento acústico de dutos não flangeados sem escoamento;
    \item modelar e analisar o comportamento acústico de dutos não flangeados com escoamento de saída;
    \item modelar e analisar o comportamento de dutos terminados por difusores do tipo corneta cilíndrica com diferentes raios e escoamento de saída;
    \item modelar e analisar o comportamento acústico interno de dutos com escoamento sugado e diferentes geometrias de terminação.
\end{itemize}

\begin{figure}[ht!]
\centering
  \includegraphics[width=.8\linewidth]{figuras/diferentes_dutos.png}
  \\
  \text{Fonte: \cite{dalmont2001radiation}}
  \captionof{figure}[Exemplos de várias termiações de dutos circulres.]{Exemplos de vários tipos de terminações: (a) flange circular; (b) flange circular com espessura do duto; (c) duto quadrado com flange de espessura quadrada; (d) flange normalizada; (e) flange esférica; (f) flange cilíndrica; (g) corneta; (h) disco não perfurado; (i) disco perfurado.}
  \label{fig:diferentes_dutos}
\end{figure} 
\chapter{Revisão Bibliográfica}

Nesse capítulo será apresentada uma revisão bibliográfica dos tópicos concernentes a acústica interna de dutos circulares. Os tópicos estão separados em modelos analíticos exatos, modelos analíticos aproximados, trabalhos experimentais, modelos numéricos e trabalhos relacionados ao desenvolvimento e aplicação do método de \textit{lattice} Boltzmann para problemas de acústica.

\section{Modelos Analíticos Exatos} 

A propagação de modos normais (ondas planas) é um problema clássico em acústica e continua tendo importância significativa mediante ao advento de novas tecnologias relacionadas a sistemas de exaustão e sucção. Em geral, pode-se utilizar dois parâmetros para caracterizar o fenômeno da acústica interna de dutos:

\begin{itemize}
    \item a magnitude do coeficiente de reflexão $|R_{r}|$, razão entre as componentes refletida e incidente da onda no duto, a qual é dada por
    \begin{equation}
        |R_{r}|\simbolo{$R_{r}$}{Coeficiente de reflexão na terminação do duto} =\bigg|\frac{Z_{r} - Z_{0}}{Z_{r} + Z_{0}}\bigg|,
        \label{eq:R}
    \end{equation}
    sendo $Z_{r}$\simbolo{$Z_{r}$}{Impedância de radiação} a impedância de radiação e $Z_{0}$\simbolo{$Z_{0}$}{Impedância característica do meio} a impedância característica do meio, definida por $Z_{0} = \rho_{0}c_{0}$, tal que $\rho_{0}$ e $c_{0}$ são, respectivamente, as constantes de densidade média do meio\simbolo{$\rho_{0}$}{Densidade média do meio} e velocidade do som\simbolo{$c_{0}$}{Velocidade do som};
    
    \item coeficiente de correção da terminação normalizado pelo raio do duto $l/a$ em que $a$ é o raio do duto. Tal parâmetro representa o comprimento acústico efetivo do duto. Em outras palavras, o fator $l$ é a quantidade adicional medida a partir da abertura do duto a qual deve propagar a onda incidente antes de ser refletida para o interior do duto com fase invertida. Tal coeficiente de correção da terminação $l$ é dado por
    \begin{equation}
        l = \frac{1}{k} \arctan\!\left(\frac{Z_{r}}{Z_{0} \, \mathrm{j}\simbolo{$j$}{Unidade imaginária}}\right)
        \label{eq:l}
    \end{equation}
    sendo o número de onda $k = \frac{\omega}{c_0}$\simbolo{$k$}{Número de onda} e $\omega$\simbolo{$\omega$}{Frequência angular em radianos} a frequência angular em radianos.
\end{itemize}

Em relação aos parâmetros discutidos acima, a solução exata, obtida através da técnica de Wiener-Hopf, para o problema de um duto não flangeado na ausência de escoamento foi proposta por \citeonline{levine1948radiation}. Esse modelo assume um duto semi-infinito com paredes infinitamente finas, fluido invíscido e presença somente de ondas planas. 

 Porém em boa parte das aplicações práticas dutos transportam escoamentos médios. Para tais circunstâncias, \citeonline{munt1990acoustic} propôs um modelo analítico exato, também baseado na técnica de Wiener-Hopf, em que se considera a presença de um escoamento subsônico no interior do duto. Considera-se nesse modelo as premissas de que o escoamento é uniforme, invíscido e que a camada cisalhante do jato é infinitamente fina. Além disso, o modelo considera a condição de Kutta na borda do duto para lidar com a singularidade da velocidade de partícula nesta região. As Figuras \ref{fig:comp1} e \ref{fig:comp2} apresentam as comparações entre casos com e sem escoamento para um duto não flangeado em termos de $R_{r}$ e $l/a$.

\begin{figure}[ht!]
\centering
  \input{figuras/abs_r_comparacao.tex}
\end{figure}

\newpage
Como é mostrado na Figura \ref{fig:comp1}, a magnitude do coeficiente de reflexão $|R_{r}|$\simbolo{$|R_{r}|$}{Magnitude do coeficiente de reflexão na terminação do duto} aumenta consideravelmente na presença de um escoamento subsônico. Além disso, pode-se perceber que, em algumas frequências, $|R_{r}|$ torna-se maior do que a unidade, implicando que a amplitude da onda refletida torna-se maior do que a da onda incidente. Este fenômeno ocorre, sobretudo, pela interação do escoamento com a borda do duto, a qual transforma energia cinética rotacional em energia acústica, como discutido por \citeonline{peters1993}. Além disso vale ressaltar que o maior valor de $|R_{r}|$ está associado com a frequência de desprendimento de vórtices na saída do duto. Em outras palavras, quando o número de Strouhal ($St$) atinge o valor de $\frac{\pi}{2}$, o tempo necessário para o vórtice desprendido na saída do duto propagar a distância de um diâmetro é igual ao período do campo acústico no interior do duto, causando assim o ponto máximo do coeficiente de reflexão.

\begin{figure}[ht!]
\centering
  \begin{tikzpicture}
  \begin{axis}[
   width=.9 \textwidth,
  height=.7 \textwidth,
  xmin=0,
  xmax=1.8,
  ymin=0.25,
  ymax=0.65,
  ytick distance=0.05,
  grid=major, % Display a grid  
  %grid style={dashed,gray!90}, % Set the style
  xlabel = \small{$ka$},
  ylabel = \small{$l/a$},
   y tick label style={/pgf/number format/.cd,%
          scaled y ticks = false,
          set decimal separator={,},
          fixed},
      x tick label style={/pgf/number format/.cd,%
          scaled x ticks = false,
          set decimal separator={,},
          fixed}%
  ]
 \addplot[color=black, thick] table[x index=0,y index=1] {dados/outros_dados/loa_LS.txt};
 \addplot[color=black, dashed] table[x index=0,y index=1] {dados/outros_dados/loa_munt.txt};
   
  \end{axis}
  \end{tikzpicture}
  \caption[Coeficientes de correção de terminação $l/a$]{Resultados analíticos exatos para o coeficiente de correção da terminação normalizado pelo raio $l/a$ de um duto não flangeado. A linha contínua apresenta o resultado sem escoamento de \citeonline{levine1948radiation} e a linha tracejada apresenta o resultado com escoamento de Mach = 0,15 de \citeonline{munt1990acoustic}.}
  \label{fig:comp2}
\end{figure}

\newpage
De acordo com a Figura \ref{fig:comp2}, a correção normalizada da terminação $l/a$ torna-se consideravelmente menor do que aquela obtida na ausência de escoamento, sobretudo para baixos números de Helmholtz ($ka$)\simbolo{$ka$}{Número de Helmholtz}. Em outras palavras, para baixas frequências e na presença de um escoamento a onda acústica é refletida em uma região mais próxima da abertura, em comparação à situação sem escoamento. Isso acontece porque o efeito de inércia provocado pela massa de fluido na saída do duto é diminuída pela presença de escoamento. De fato, este fenômeno pode ser observado pela diminuição da parte imaginária da impedância nas baixas frequência, como observado por \citeonline{peters1993}.

\section{Modelos Analíticos Aproximados}

No que diz respeito a modelos analíticos aproximados, o trabalho de \citeonline{carrier1955sound} foi um dos primeiros a abordar o cálculo do coeficiente de reflexão e correção da terminação com escoamento de exaustão num duto não flangeado. Para tal foi considerado um gás perfeito invíscido com o tipo de escoamento uniforme (\textit{plug}). Nessa abordagem usou-se a técnica de Wiener-Hopf com o método de Prandtl-Glauert e a premissa de um duto semi-infinito com paredes infinitamente finas. Esse modelo é limitado a Machs subsônicos ($M$ $<$ $0,4$)\simbolo{$M$}{Número de Mach} e ondas planas, ou seja, valores de $ka$ $<$ $1,8$.

\citeonline{mani1973refraction} deu continuidade a mesma abordagem de \citeonline{carrier1955sound} com escoamento de exaustão para Machs subsônicos ($M$ $<$ $0,3$) e ondas planas, porém considerando deslocamento transversais de partículas na interface entre o ar em repouso externo e o jato de saída do duto como condição de cortorno do problema. Esse tipo de solução mostra diversos fenômenos antes não previstos com os outros modelos citados como efeitos de convecção, zonas de silêncio relativo e refrações.

Também na mesma linha de desenvolvimento de \citeonline{carrier1955sound}, \citeonline{savkar1975radiation} desenvolveu um modelo de modos de alta ordem ($ka$ $<$ $4,59$) com escoamento de exaustão e sucção do tipo \textit{plug} ($M$ $<$ $0,4$) com variação de temperatura. A continuidade do deslocamento das partículas acústicas transversais também foi considerada na interface entre o ar em repouso externo e o jato de saída do duto, possibilitando assim análises de fenômenos de convectivos. Como metodologia para construção desse modelo foram aplicadas as técnicas de Wiener-Hopf e a aproximação matemática do trabalho de \citeonline{carrier1955sound}.

Já o trabalho de \citeonline{hirschberg2014} propõe uma expressão analítica aproximada do coeficiente de reflexão para baixas frequências ($ka$ $<$ $1$), baixos números de Mach ($M$ $<$ $0,2$) e jatos quentes. Esse modelo considera os efeitos de convecção e temperatura e foi consolidado a partir da aproximação proposta pelo trabalho de \citeonline{howe1979}.   

\section{Trabalhos Experimentais}

\subsection{Escoamento de Exaustão}

No que diz respeito a escoamentos de exaustão o trabalho de \citeonline{alfredson1970radiation} insvestigou os coeficientes de reflexão e correção da terminação e o fator de amortecimento de ondas acústicas. Para tal foi utilizado um duto excitado com um pulso de pressão, submetido a escoamentos subsônicos ($M$ $<$ $0,2$) e dados extraídos com a técnica dos dois microfones ajustada para valores de $ka$ $<$ $1$. A principal conclusão desse trabalho é o fato da magnitude do coeficiente de reflexão ser maior nos casos com escoamento.

O trabalho de \citeonline{peters1993} investigou os coeficientes de reflexão e de dissipação de ondas acústicas devido aos efeitos térmicos e de viscosidade na presensa e ausência de cornetas. A técnica de dois microfones foi utilizada para extração dos dados num regime de baixas frequências ($ka$ $<$ $1,5$) e valores subsônicos do número de Mach ($M$ $<$ $0,2$). Por fim, o autor argumenta que a inserção de uma corneta no final do duto aumenta o coeficiente de reflexão por conta do aumento da instabilidade da camada limite na parede da corneta.

O trabalho de \citeonline{allam2006investigation} utilizou um sistema superdeterminado de medição para investigação do coeficiente de reflexão de um duto não flangeado. Para minimizar a relação sinal ruído e assim obter com mais acurácia o coeficiente de correção da terminação do duto, surgiu-se como motivação o desenvolvimento de um sistema em que há mais microfones do que incógnitas a serem calculadas, em outras palavras, extendeu-se a metodologia de medição de 2 microfones para mais que 4 microfones. Há de se considerar também que a parte imaginária do número de onda, parte associada com a dissipação de energia por viscosidade, não pode ser obtida quando há escoamento e por isso foi incluída como incógnita. Em linhas gerais esse trabalho permitiu a validação experimental do trabalho de \citeonline{munt1990acoustic} e a consolidação de um sistema confiável de medição para esse tipo de problema.

\citeonline{english2010} investigou também de forma experimental os coeficientes de reflexão e de terminação de dutos circulares com diferentes espessuras, através da técnica de extração de autoespectro e espectro cruzado em pares de microfones calibrados para o intervalo $0$ $<$ $ka$ $<$ $0,7$. Focando para números de Mach entre 0 e 0,3, seus resultados mostram que os coeficientes de reflexão estão com valores acima dos que são encontrados no trabalho de \citeonline{munt1990acoustic} e \citeonline{allam2006investigation}. O autor explica esse fato relatando que as premissas de camada limite viscosa e parede de duto infinitamente finas assumidas pelo modelo de Munt subestimam a transferencia de energia cinética rotacional do jato em energia acústica. No entanto, o autor não discute porque os resultados anteriores, obtidos por \citeonline{allam2006investigation}, apresentam maior concordância com o modelo de Munt. 


Já o trabalho de \citeonline{tiikoja2014} focou na influência da temperatura no coeficiente de reflexão de dutos não flangeados. Para tal fim utilizaram a técnica de 2 microfones num sistema superdeterminado com 3 microfones, ajustados num contexto de ondas planas ($ka$ $<$ $1,8$) e Machs de até 0,3 e 0,12 para jatos frios e quentes respectivamente. Tendo como referências as curvas validadas de jatos frios de \citeonline {munt1990acoustic}, foi observado como resultado do estudo que para jatos quentes as curvas dos coeficientes de reflexão e de terminação sofrem um aumento de amplitude e um deslocamento do pico máximo em direção para baixas frequências, em outras palavras, com o aumento da temperatura o gás obtém maior energia cinética e os efeitos de vorticidade na terminação do duto ficam mais intensos com relação a temperaturas mais baixas.



\subsection{Escoamento Sugado}

Em relação a trabalhos experimentais, \citeonline{ingard1975} investigaram o coeficiente de reflexão em dutos quadrados em regime de escoamento succionado de Mach 0,4. O método de medição se baseou na técnica de dois microfones e os mesmos foram ajustados para números de Helmholtz ($ka$) menores que 0,5. Em vista desse contexto experimental, o autor desenvolveu uma fórmula semi-empírica para baixas frequências do coeficiente de reflexão e é modelada na forma
\begin{equation}
        |R_{r}| = |R_{0}|\bigg[\frac{(1 - M)}{(1 + M)}\bigg]^{n},
    \label{eq:R_r_ingard}
\end{equation}
sendo que $n$ é uma constante no valor aproximado de 1,33 e $|R_{0}|$ é o módulo do coeficiente de reflexão sem escoamento obtido a partir do trabalho de \citeonline{levine1948radiation}. 

Na mesma linha de investigação, \citeonline{davies1987} investigou o coeficiente de reflexão para baixas frequências ($0,01$ $<$ $ka$ $<$ $0,25$) e Machs subsônicos ($M$ $<$ $0,3$), porém com dutos circulares não-flangeados, flangeados e com difusores na borda. O autor destaca que a disposição geométrica da terminação do duto, quando submetida a fenômenos de escoamentos succionados, desenvolve uma ``\textit{vena} contracta'', que pode ser estimada e associada com o fator de perda $Kp$\simbolo{$Kp$}{Fator de perda}. Em vista dos procedimentos desse trabalho, o autor compara os resultados com o estudo de \citeonline{ingard1975} e sugere que o termo $n$ da equação \ref{eq:R_r_ingard} tenha o valor aproximado de 0,9 para casos de dutos circulares. 

Mesmo na ausência de uma investigação sistemática focando o coeficiente de correção da terminação, o trabalho de \citeonline{davies1987} também sugere uma equação para o qual é proposto na seguinte forma
\begin{equation}
    l/a = l_{0}(1 - M^{2}),
    \label{eq:l_M}
\end{equation} 
sendo $l_{0}$ módulo do coeficiente de correção da terminação sem escoamento obtido a partir do trabalho de \citeonline{levine1948radiation}.


\begin{figure}[h!]
\centering
  \begin{tikzpicture}
  \begin{axis}[
      %axis lines = left,
      xlabel = $M$,
      ylabel = $|R_{r}|$,
      width=0.9\textwidth,
      height=0.45\textwidth,
      ytick distance=0.1,
      x tick label style={
        /pgf/number format/.cd,
            fixed,
            fixed zerofill,
            precision=2, set decimal separator={,},
        /tikz/.cd
    },
    xmin=0,
    xmax=0.3,
    ymin=0.43,
    ymax=1,
    grid=major,
    y tick label style={/pgf/number format/.cd,%
          scaled y ticks = false,
          set decimal separator={,},
          fixed}
  ]
  \addplot [
      domain=0:0.3, 
      samples=100, 
      color=black
  ]
  {((1 - x)/(1 + x))^0.9};

  \addplot [
      domain=0:0.3, 
      samples=100, 
      color=black,
      dashed
  ]
  {((1 - x)/(1 + x))^1.333};
   
  \end{axis}
  \end{tikzpicture}
  \caption[Coeficiente de reflexão $|R_{r}|$ com escoamento sugado]{Resultado da magnitude do coeficiente de reflexão $|R_{r}|$ em relação ao Mach para baixas frequências com escoamento sugado. A linha contínua apresenta o cálculo obtido a partir do trabalho de \citeonline{davies1987} e a linha tracejada apresenta o resultado obtido a partir do trabalho de \citeonline{ingard1975}.}
  \label{fig:comp3}
\end{figure}

A Figura \ref{fig:comp3} mostra o gráfico resultante da equação \ref{eq:R_r_ingard} para os estudos de \citeonline{ingard1975} e  \citeonline{davies1987}. Pode-se perceber que $|R_{r}|$ decai de acordo com o aumento do Mach, em outras palavras, para baixas frequências, a onda plana possui maior facilidade de se radiar para o meio externo a medida que o Mach é aumentado. 


\begin{figure}[h!]
\centering
  \begin{tikzpicture}
  \begin{axis}[
      %%axis lines = left,
      xlabel = $M$,
      ylabel = $l/a$,
      width=0.9\textwidth,
     height=0.45\textwidth,
      ytick distance=0.01,
       x tick label style={
        /pgf/number format/.cd,
            fixed,
            fixed zerofill,
            precision=2, set decimal separator={,},
        /tikz/.cd,
    },
    xmin=0,
    xmax=0.3,
    ymin=0.535,
    ymax=0.6,
    grid=major,
     y tick label style={/pgf/number format/.cd,%
          scaled y ticks = false,
          set decimal separator={,},
          fixed}
  ]
  %Below the red parabola is defined
  \addplot [
      domain=0:0.3, 
      samples=100, 
      color=black
  ]
  {0.6*(1 - x^2)};
   
  \end{axis}
  \end{tikzpicture}
  \caption[Coeficiente de correção da terminação $l/a$]{Resultado do coeficiente de correção da terminação $l/a$ em relação ao Mach para baixas frequências ($ka$ $<$ $0,25$). A linha contínua apresenta o cálculo obtido com a equação \ref{eq:l_M} do trabalho de \citeonline{davies1987}.}
  \label{fig:comp4}
\end{figure}

 A Figura \ref{fig:comp4} mostra o gráfico resultante da equação \ref{eq:l_M} e pode-se perceber que $l_{M}$ decai de acordo com o aumento do Mach, ou seja, para baixas frequências, o comprimento efetivo acustico do duto diminui a medida que o Mach do escoamento succionado é aumentado, o que constitui um paradoxo. Mesmo com esses coeficientes modelados com apoio de dados experimentais a literatura carece de informações sobre esses parâmetros para frequências mais altas ($ka$ $>$ $0,25$).


\section{Modelos Numéricos}

Já em relação a trabalhos envolvendo métodos numéricos, \citeonline{selamet2001wave} analisaram os coeficientes de reflexão e de terminação de dutos circulares com diferentes geometrias sem escoamento num contexto de ondas planas ($ka$ $<$ $1,8$). Para isso utilizaram método dos elementos de contorno e obtiveram para cada configuração de geometria as seguintes conclusões:
\begin{itemize}
  \item não-flangeados: para diferentes tamanhos o comportamento se diferencia muito pouco de um caso de um duto semi-infinito;
  \item extendidos a partir de uma parede rígida e infinita: os coeficientes de reflexão e de correção da terminação possuem oscilações com relação a linha de base desses mesmos coeficientes para dutos não-flangeados, pois há interações entre as ondas da terminação do duto e as refletidas da parede;
  \item extendidos obliquamente a partir de uma parede rígida e infinita: no geral o coeficiente de reflexão decresce para altos números de $ka$ com relação ao extendido de forma perpendicular;
  \item terminados em forma de sino: redução significativa do coeficiente de reflexão para altos números de $ka$;
  \item terminados com cavidade anular: adiciona picos nos coeficientes de reflexão com relação aos não-flangeados.
\end{itemize}

Seguindo uma linha de análise semelhante, \citeonline{dalmont2001radiation} analisaram coeficientes de terminação de dutos circulares com diversas geometrias de terminação num contexto de ondas planas ($ka$ $<$ $1,8$), sobretudo as que são comumente encontradas em instrumentos de sopro. As análises foram feitas comparando-se resultados experimentais com resultados numéricos obtidos a partir dos métodos dos elementos finitos e elementos de contorno. A partir do ajuste dos modelos numéricos, derivaram modelos semi-empíricos simplificados para os coeficientes de reflexão encontrados nas geometrias estudadas. 

Tendo como motivação a validação do método de \textit{lattice} Boltzmann para problemas de acústica de dutos, \citeonline{da2006lattice} abordaram análises dos coeficientes de reflexão e de terminação de dutos circulares não flangeados, sem escoamento e focando ondas planas ($ka$ $<$ $1,8$). As boas correlações dos dados numéricos com os dados vigentes da teoria de \citeonline{levine1948radiation} mostram que o método é bastante útil para predizer fenômenos complexos envolvendo acústica de dutos.

Complementando o trabalho anterior, \citeonline{da2009sound} investigaram os coeficientes de reflexão e de terminação de dutos circulares com terminações de corneta e com escoamento subsônico. Para isso implementaram o modelo usando o método de \textit{lattice} Boltzmann com condições de contorno absorventes, axissimetria de acordo com o trabalho de \citeonline{reis2007} e paredes curvas para a consolidação das terminações em cornetas. Com esse trabalho foram validados os resultados de \citeonline{munt1990acoustic} e \citeonline{allam2006investigation} além de mostrar que na presença de cornetas o coeficiente de reflexão aumenta bastante no pico associado ao número de Strouhal $St \approx \frac{\pi}{2}$\simbolo{$St$}{Número de Strouhal}. Tal fato é aderente aos vários trabalhos que abordam o escoamento de exaustão e é explicado pelo fato de valores maiores do 1 para a magnitude do coeficiente de reflexão podem ser encontrados tanto em dutos não-flangeados quanto cornetas. No entanto, no caso das cornetas, este aumento é consideravelmente maior devido a indução de vorticidade causada pela terminação circular. Além disso, os resultados observados sugerem que a região de Strouhal $St \approx \frac{\pi}{2}$ acontecem quando o período do campo acústico interno conicide com o tempo necessário para que um vórtice propague a distância equivalente a um raio de corneta.

\citeonline{silva2012approximate} usaram o método de elementos de contorno para analisar a influência do raio de uma terminação flangeada no comportamento do coeficiente de reflexão de dutos circulares na ausência de escoamento. Para tanto validaram o modelo com os resultados de \citeonline{levine1948radiation}, dutos não flangeados, e de \citeonline{nomura1960}, dutos flangeados circulamente. Como resultado da análise propuseram expressões aproximadas para o cálculo dos coeficientes de reflexão e de terminação.

\chapter{Metodologia}

Assim como foi discutido no capítulo anterior, para obter os parâmetros de caracterização da acústica interna de dutos circulares com escoamento, é preciso de um modelo que suporte a interação fluxo de massa e variação de pressão. Portanto modelos baseados em métodos de fluido dinâmica computacional se mostram adequados, principalmente aqueles baseados no método de \textit{lattice} Boltzmann assim como foi mostrado anteriormente nos trabalhos de \citeonline{da2006lattice} e \citeonline{da2009sound}. Nesse sentido, há traballhos que validam, aplicam e desenvolvem metodologias de \textit{lattice} Boltzmann no campo de estudo da aeroacústica.

Um desses estudos é o de \citeonline{crouse2006fundamental}, que mostraram a eficácia do método de \textit{lattice} Boltzmann em recuperar as equações de Navier-Stokes de transiente, compressível e viscosa. Há de se ressaltar que validaram também o modelo numérico de um ressonador de Helmholtz com um modelo experimental do mesmo, demonstrando assim a viabilidade da aplicação para problemas de acústica.

No que se trata de desenvolvimento de ferramentas auxiliares para tratar problemas acústicos, \citeonline{kam2006non} desenvolveu uma condição de contorno que caracteriza dissipação de energia acústica e fluido dinâmica, ou seja, uma condição de contorno anecóica. Ela se baseia no acoplamento de mais um termo na equação de Boltzmann para gerar uma região de amortecimento, redirecionando os valores de densidade e velocidade das partículas para um valor alvo, que seria no caso valores médios de densidade e velocidade num fluido em repouso.

\citeonline{marie2009} analisou e comparou esquemas de alta ordem das equações de Navier-Stokes linearizadas com o método de \textit{lattice} Boltzmann. O objeto de estudo para comparação foi análises de dispersão e dissipação de ondas acústicas em regime isotérmico. Conclui-se com esse trabalho que para um determinado erro de dispersão o método de \textit{lattice} Boltzmann se comportou como mais rápido.

No que diz respeito a aplicação do método de \textit{lattice} Boltzmann num problema de aeroacústica, \citeonline{lew2010noise} desenvolveram um modelo numérico em 3D para predição de ruído em um jato turbulento subsônico. Como validação os resultados foram comparados com resultados experimentais e cálculos numéricos feitos a base de \textit{Large} \textit{Eddy} \textit{Simulation} (LES)\abreviatura{LES}{\textit{Large} \textit{Eddy} \textit{Simulation}}. Esse estudo demonstrou as principais vantagens de se trabalhar com o método de \textit{lattice} Boltzmann como por exemplo o baixo custo computacional e a facilidade em inserir \textit{nozzles} com formas complexas no domínio computacional.

Também na área de aeroacústica computacional, o trabalho de \citeonline{shi2013lattice} propõe um modelo em \textit{lattice} Boltzmann para obter dados de diretividade da radiação sonora num duto circular submetido a escoamento subsônico. Os resultados de diretividade foram comparados com os modelos de \citeonline{levine1948radiation} e \citeonline{gabard2006}, mostrando uma boa convergência principalmente nas baixas frequências.

Já no sentido de tratamento de fenômenos da acústica básica, \citeonline{viggen2013acoustic} investigou os efeitos da adição de termos fontes no método de \textit{lattice} Boltzmann, mapeando eles nos parâmetros macroscópicos através da ferramenta matemática de expansão de Chapman-Enskog. Como resultado conseguiu reproduzir fenômenos de diretivade de monopolos, dipolos e quadrupolos.

\citeonline{da2015assessment} abordaram também o uso do método de \textit{lattice} Boltzmann acoplado com \textit{Large} \textit{Eddy} \textit{Simulation} (LES) na investigação do ruído gerado na interação do escoamento de um jato com uma placa plana. Os dados de níveis de pressão sonora em campo distante foram obtidos usando a superfície de Ffowcs-Williams e Hawkings (FW-H)\abreviatura{FW-H}{Superfície de Ffowcs-Williams e Hawkings} e os mesmos possuem uma boa convergência com dados experimentais.

Investigar a acústica interna de dutos circulares com escoamentos é um processo que deve ter suporte de ferramentas bem específicas, como por exemplo o método numérico de \textit{lattice} Boltzmann. Esse capítulo portanto abordará esse método e as condições de uso implementadas, validadas e verificadas num \textit{software} de código aberto chamado \citeonline{palabos}. Abordar o uso de um \textit{software} de código aberto possibilita a verificação transparente dos processos de cálculo bem como adaptações com novas implementações dentro do projeto, focando a melhor aderência da ferramenta computacional para resolução do problema.


\section{O Método de Lattice Boltzmann}

O método de \textit{lattice} Boltzmann possui bastante utilidade quando se trata de problemas aeroacústicos, pequenas flutuações de pressão e fenômentos de turbulência. Isso se deve pelo fato do método ter surgido de uma outra abordagem de fenômenos mecânicos aplicados a fluidos - uma abordagem microscópica de interações entre moléculas.

Essa abordagem se chama Dinâmica Molecular (DM)\abreviatura{DM}{Dinâmica Molecular} e é baseada nas formulações Newtonianas de choque e propagação de partículas, em outras palavras, as posições no espaço e as velocidades podem ser obtidas a partir da aplicação da segunda lei de Newton para cada partícula. Segundo essa ideia, outras propriedades do fluido como densidade, pressão e temperatura podem ser facilmente recuperadas através do cáculo da média correspondente a um conjunto de partículas. Porém o principal problema dessa abordagem é que há uma grande quantidade de equações para se resolver num pequeno volume de fluido, pois, considerando que, de acordo com o número de Avogrado, nesse mesmo volume há na ordem de $10^{23}$ moléculas para calular os movimentos cinéticos. Tal fato se torna inviável para implementação mesmo com computadores potentes como \textit{clusters} de alto desempenho.

Uma solução para contornar o problema da grande quantidade de equações do movimento é abordar o fenômeno físico estatisticamente, ou seja, formular a evolução do movimento do fluido no tempo em termos de uma equação de transporte: uma função de distribuição de partículas. Uma equação de transporte bastante apropriada é a equação de Boltzmann que, ao ser discretizada, pode ser resolvida de forma numérica originando assim o método de \textit{lattice} Boltzmann ou \textit{lattice} \textit{Boltzmann} \textit{Method} (LBM)\abreviatura{LBM}{\textit{Lattice} \textit{Boltzmann} \textit{Method}}.

Historicamente o método de \textit{lattice} Boltzmann se originou a partir de um modelo de DM chamada \textit{Lattice} \textit{Gas} \textit{Automata} (LGA)\abreviatura{LGA}{\textit{Lattice} \textit{Gas} \textit{Automata}}. Esse modelo surgiu nos anos 80 com o estudo de \citeonline{frisch1986} mostrando a recuperação das equações de Navier-Stokes para pequenos números de Knudsen. Como esse modelo funciona somente para choque de partículas singulares, houve a necessidade de um modelo mais sofisticado e completo, então nos anos 90 e 2000 os trabalhos de \citeonline{sterling1996} e \citeonline{wolf2004} consolidaram o LBM sanando essa limitação com um processo de choques de conjunto de partículas.

O LBM possui muitas vantagens em relação a técnicas tradicionais de fluido dinâmica computacional aplicadas a aeroacústica: resolve o campo acústico e o campo fluido dinâmico numa mesma iteração em cada incremento de tempo, extração direta do campo de pressão e fácil implementação paralela elevando assim a performance frente a outros métodos.

\subsection{Modelo BGK}

O LBM, baseado em operações de colisão e propagação de funções de distribuição de partículas com massa, é a equação de Boltzman discretizada no tempo e espaço. Cada conjunto de funções de distribuição localizadas num ponto no espaço $\textbf{x}$ e tempo $t$ pode ser chamada de célula e, segundo o trabalho de \citeonline{he}, a equação de Boltzman, que formula o comportamento de cada célula, pode ser escrita na expressão 
\begin{equation}
	f_{i}(\textbf{x} + c_{i}\Delta t, t + \Delta t) = f_{i}(\textbf{x}, t) + \Omega_{i}(f(\textbf{x}, t)),
    \label{eq:f_i}
\end{equation}
sendo que $i$ é um número inteiro que delimita direções no espaço de propagação de partículas, $f_{i}$ são funções de distribuição na direção $i$, $c_{i}$ são velocidades de propagação na direção $i$ e $\Delta t$ é o incremento discreto de tempo. 
\simbolo{$f_{i}$}{Função de distribuição LBM na direção $i$}
\simbolo{$i$}{Direção de propagação LBM}
\simbolo{$c_{i}$}{Velocidades de propagação na direção $i$}
\simbolo{$\textbf{x}$}{Localização espacial de uma célula LBM}
\simbolo{$t$}{Localização temporal de uma célula LBM}
\simbolo{$\Delta t$}{Incremento discreto de tempo}

A equação \ref{eq:f_i} é dividida nas duas operações básicas do método: propagação e colisão. O lado esquerdo dessa equação representa a operação de propagação, na qual os valores das funções de distribuição de cada célula são movidos para cada direção de propagação para uma próxima célula no espaço em cada iteração no tempo. Feita a operação de propagação, é realizada a operação de colisão, representada pelo lado direito da equação, na qual o termo $\Omega_{i}$\simbolo{$\Omega_{i}$}{Operador de colisão LBM} representa o operador de colisão.

Uma das formas de calcular o operador de colisão $\Omega_{i}$ é usar a formulação proposta no estudo de \citeonline{bgk}. A aplicação dessa formulação consolida o modelo BGK (Bhatnagar–Gross–Krook)\abreviatura{BGK}{Bhatnagar–Gross–Krook} ou modelo de tempo de relaxação único: \textit{single}-\textit{relaxation}-\textit{time} (SRT)\abreviatura{SRT}{\textit{single}-\textit{relaxation}-\textit{time}}. Nesse sentido, o operador de colisão é definido por
\begin{equation}
	\Omega_{i} = -\frac{1}{\tau}(f_{i} - f_{i}^{M}),
    \label{eq:omega_i}
\end{equation}
tal que $\tau$\simbolo{$\tau$}{Período de colisão LBM} é o período de colisão e $f_{i}^{M}$\simbolo{$f_{i}^{M}$}{Função de distribuição de Maxwell ou de equilíbrio} é a função de distribuição de Maxwell ou função de distribuição de equilíbrio.

A função de distribuição de Maxwell $f_{i}^{M}$ pode ser calculada aplicando o princípio de máxima entropia de acordo com as retrições das leis de conservação de massa e quantidade de movimento, assim como é proposto por \citeonline{wolf}. Dessa forma a função de distribuição de Maxwell é definida por
\begin{equation}
	f_{i}^{M} = \rho \varepsilon _{i}\bigg( 1 + \frac{\textbf{u}.c_{i}}{c_{s}^{2}} + \frac{\textbf{u}.c_{i}^{2} - c_{s}^{2}\textbf{u}}{2c_{s}^{4}}\bigg),
    \label{eq:f_i_M}
\end{equation}
sendo que $\rho$ é a densidade local do fluido, $\varepsilon_{i}$ são pesos de velocidades para cada direção de propagação $i$, $\textbf{u}$ é a velocidade local do fluido, $c_{i}$ é um vetor de velocidades de propagação da célula para cada direção $i$ e $c_{s}$ é a velocidade do som.
\simbolo{$\rho$}{Densidade local do fluido}
\simbolo{$\varepsilon_{i}$}{Pesos de velocidades para cada direção de propagação $i$}
\simbolo{$\textbf{u}$}{Velocidade local do fluido}
\simbolo{$c_{s}$}{Velocidade do som}

\newpage
Os parâmetros macroscópicos de densidade local do fluido $\rho$, velocidade local do fluido $\textbf{u}$, pressão local do fluido $p$ e a viscosidade cinemática $\nu$ podem ser obtidos a partir dos momentos da função de distribuição $f_{i}$ nas formas
\begin{equation}
	\rho = \sum{f_{i}},
    \label{eq:rho}
\end{equation}
\begin{equation}
	\rho \textbf{u} = \sum{f_{i} c_{i}},
    \label{eq:u}
\end{equation}
\begin{equation}
	p = \rho c^{2}_{s} \text{ e }
    \label{eq:p}
\end{equation}
\begin{equation}
	\nu = c^{2}_{s} \bigg(\tau - \frac{1}{2}\bigg).
    \label{eq:nu}
\end{equation}
\simbolo{$p$}{Pressão local do fluido}
\simbolo{$\nu$}{Viscosidade cinemática do fluido}

Há vários modelos de célula do tipo BGK, o grupo do tipo $D_{n}Q_{b}$ ($n$ dimensões e $b$ direções de propagação ou velocidades) é um dos mais usados e foi proposto por \citeonline{qian1992lattice}. A tabela \ref{table:modelos} mostra os parâmetros para cada um dos modelos do tipo $D_{n}Q_{b}$.

\begin{table}[ht!]
\centering
\caption{Modelos $D_{n}Q_{b}$}
\label{table:modelos}
\begin{tabular}{|c|c|c|c|}
\hline
Modelo & $c_{i}$ & $\varepsilon_{i}$ & $c_{s}^{2}$ \\ \hline
%-----------------------------------------------------------------------------
D1Q3   & $0$,                        & $2/3$,                   & $1/3$ \\
   	   & $\pm 1$                     & $1/6$                    &     \\ \hline
%-----------------------------------------------------------------------------
   	   & $0$,                        & $6/12$,				    &  \\
D1Q5   & $\pm 1$,                    & $2/12$,			        & $1$ \\  
       & $\pm 2$                     & $1/12$			        &    \\ \hline
%-----------------------------------------------------------------------------
D2Q7   & $(0,0)$,                  & $1/2$,                    & $1/4$ \\ 
	   & $(\pm 1/2, \pm \sqrt{3}/2)$ & $1/12$                  &    \\ \hline
%-----------------------------------------------------------------------------
       & $(0,0)$,                     & $4/9$,                    &    \\
D2Q9   & $(\pm 1,0)$, $(0,\pm 1)$,    & $1/9$,                    & $1/3$   \\
   	   & $(\pm 1,\pm 1)$              & $1/36$                    &    \\ \hline
%-----------------------------------------------------------------------------
	   & $(0,0,0)$,                           & $2/9$,            &   \\ 
D3Q15  & $(\pm 1,0,0)$, $(0,\pm 1,0)$, $(0,0,\pm 1)$, & $1/9$,   & $1/3$ \\ 
	   & $(\pm 1, \pm 1,\pm 1)$               & $1/72$           &   \\ \hline
%-----------------------------------------------------------------------------
	   & $(0,0,0)$,                           & $1/3$,                        &    \\ 
D3Q19  & $(\pm 1,0,0)$, $(0,\pm 1,0)$, $(0,0,\pm 1)$, & $1/18$,               & $1/3$ \\ 
	   & $(\pm 1,\pm 1,0)$, $(\pm 1,0,\pm 1)$, $(0,\pm 1,\pm 1)$ & $1/36$,    &    \\ \hline 
\end{tabular}
\end{table}

De acordo com a tabela \ref{table:modelos}, é possível ter uma visão clara que para cada modelo de célula do tipo $D_{n}Q_{b}$ há diferentes vetores de velocidades de propagação ($c_{i}$), seus respectivos pesos $\varepsilon_{i}$ e as suas constantes de velocidade do som ($c_{s}$). Com esses parâmetros já se torna possível calcular a função de Maxwell ($f_{i}^{M}$) para cada operação de colisão em cada iteração de tempo. Para esse trabalho usou-se o modelo D3Q19 e a Figura \ref{fig:d3q19} ilustra um esquemático desse tipo de célula.

\begin{figure}[ht!]
\centering
  \includegraphics[width=.8\linewidth]{figuras/d3q19.pdf}
  \caption[Esquemático do D3Q19]{Esquemático do modelo D3Q19. Ilustração adaptada do estudo de \citeonline{premnath2013investigation}.}
  \label{fig:d3q19}
\end{figure}

Na Figura \ref{fig:d3q19} é possível visualizar espacialmente as direções de propagação da célula. Vale ressaltar que para cada direção há o cáculo da função de Maxwell ($f_{i}^{M}$) e, por conseguinte, a operação de propagação das funções de distribuição para a célula adjacente no sentido de cada direção.

\subsection{Múltiplos Tempos de Relaxação}

A equação \ref{eq:omega_i} retrata um operador de colisão com tempo de relaxação único. Essa abordagem é funcional porém, em regimes de pouca viscosidade cinemática como é o caso do ar, começa a desenvolver várias instabilidades e divergências como mostra o estudo de \citeonline{lallemand2000theory}. Para esses tipos de problemas a abordagem de múltiplos tempos de relaxação, \textit{multiple}-\textit{relaxation}-\textit{time} (MRT)\abreviatura{MRT}{\textit{multiple}-\textit{relaxation}-\textit{time}}, pode ser usada assim como é mostrado nos estudos de \citeonline{viggen2014lattice}.

De acordo com o esquema proposto por \citeonline{d1994generalized}, a formulação de MRT se baseia na troca do parâmetro de único tempo de relaxação $\tau$ por uma matriz \textbf{$\Lambda$} de vários tempos de relaxação. Todavia a matriz \textbf{$\Lambda$} é construída de acordo com uma matriz \textbf{$M$} que projeta as funções de distribuição $f_{i}$ e $f_{i}^{M}$ no espaço dos momentos. De acordo com \citeonline{lallemand2000theory}, a possibilidade desse método ser mais estável é oriunda da capacidade de operar a colisão das células com um tempo de relaxação apropriado para cada um dos vários momentos, projetados a partir das funções de distribuição $f_{i}$ e $f_{i}^{M}$. Em vista do exposto o operador de colisão da equação \ref{eq:omega_i} se transforma em
\begin{equation}
	\Omega_{i} = -\textbf{$\Lambda$}(f_{i} - f_{i}^{M}).
    \label{eq:MRT_1}
\end{equation}
Porém a operação de colisão é realizada no espeço dos momentos, logo é preciso projetar $f_{i}$ e $f_{i}^{M}$ no espeço dos momentos ficando
\begin{equation}
	m_{i} = \textbf{$M$}f_{i} \text{ e } m_{i}^{M} = \textbf{$M$}f_{i}^{M}.
    \label{eq:MRT_2}
\end{equation}
Considerando que a matriz \textbf{$S$} é dada por
\begin{equation}
	\textbf{$S$} = \textbf{$M$}\textbf{$\Lambda$}\textbf{$M$}^{-1},
    \label{eq:MRT_3}
\end{equation}
o operador de colisão fica
\begin{equation}
	\Omega_{i} = -\textbf{$M$}^{-1}\textbf{$S$}(m_{i} - m_{i}^{M}).
    \label{eq:MRT_4}
\end{equation}
Inserindo a equação \ref{eq:MRT_4} na equação \ref{eq:f_i} fica
\begin{equation}
	f_{i}(\textbf{x} + c_{i}\Delta t, t + \Delta t) = f_{i}(\textbf{x}, t) -\textbf{$M$}^{-1}\textbf{$S$}(m_{i} - m_{i}^{M}).
    \label{eq:MRT_5}
\end{equation}
Vale ressaltar que a operação de propagação, lado esquerdo da equação \ref{eq:MRT_5}, ocorre no espaço original da função de distribuição $f_{i}$.

\subsection{Transformações para Unidades Físicas}

Quando as equações \ref{eq:rho}, \ref{eq:u}, \ref{eq:p} e \ref{eq:nu} são usadas para recuperar os atributos macroscópicos do fluido a unidade de medida não é uma unidade física. Segudo o trabalho de \citeonline{da2016prediction}, para se ter esses atributos em unidade física é preciso aplicar regras de conversão. Essas regras de conversão se baseiam em duas constantes que são definidas a partir de unidades físicas: velocidade característica definida por   
\begin{equation}
	\zeta = c^{*}/c_{s},
    \label{eq:conversao_1}
\end{equation}
tal que $c^{*}$ é a velocidade física do som, e discretização $\Delta x$ definida pela quantidade de metros por tamanho de célula.

Com os parâmetros $c^{*}$ e $\Delta x$ pode-se realizar as seguintes conversões para unidades físicas:
\begin{gather*}
	\textbf{$u^{*}$} = \zeta \textbf{$u$}\text{, } \\ \textbf{$x^{*}$} = \Delta x\textbf{$x$}
	\text{, } \\ t^{*} = \frac{\Delta x}{\zeta}t \text{, } \\ \nu^{*} = \zeta \Delta x \nu
	\text{, } \\ \rho^{*} = \frac{\zeta}{\Delta x} \rho \text{, } \\ p^{*} = p \zeta^{2}  \rho_{0} \text{ e } \\ f^{*} = f\frac{\zeta}{\Delta x},
    \label{eq:conversao_1}
\end{gather*}
tal que as variáveis assinaladas com $*$ estão em unidades físicas e $f^{*}$ e $f$ são unidades de frequências física e do LBM respectivamente\simbolo{$f^{*}$}{Frequência física}\simbolo{$f$}{Frequência em LBM}.

\subsection{Condições de Contorno}

Como ocorre nas técnicas numéricas tradicionais, o LBM possui também possui condições de contorno. Como mostra o estudo \citeonline{viggen2014lattice}, as condições de contorno para LBM podem ser classificadas em dois tipos: explícita e implícita. As condições de contorno explícitas são aquelas aplicadas em cada célula, tendo a natureza totalmente alinhada entre elas no domínio, normalmente condições como essas são aplicações personalizadas do cálculo do operador de colisão $\Omega_{i}$. As condições de contorno implícitas são aquelas aplicadas numa região do domínio não alinhado entre as células, normalmente essas condições são aplicadas na operação de propagação. A seguir serão apresentadas as condições de contorno abordadas nesse trabalho, cada uma de cada classificação apresentada.

\subsubsection{\textit{Bounceback}}

De acordo com o estudo de \citeonline{viggen2014lattice}, a condição de contorno do tipo \textit{bounceback} tem como objetivo simular uma parede rígida no domínio do LBM, sendo ela do tipo implícita e localizada entre as células. Há dois tipos de \textit{bounceback}: \textit{free-slip}, que simula escorregamento livre do fluido na condição de contorno e \textit{no-slip}, que simula camada limite do fluido na condição de contorno. Nesse trabalho foi usado o do tipo \textit{no-slip} pois num caso real o escoamento desenvolve camada limite na parede rígida.

A condição de contorno \textit{bounceback} \textit{no-slip} é geralmente implementada na etapa de propagação a partir de uma inversão de funções de distribuição de partículas. A Figura \ref{fig:bounceback} mostra um esquemático de exemplo do processo de funcionamento dessa condição de contorno. 

\begin{figure}[ht!]
\centering
  \includegraphics[width=.65\linewidth]{figuras/bounceback.pdf}
  \caption[Processo de funcionamento do \textit{bounceback} \textit{no-slip}]{Esquemático de exemplo do processo de funcionamento da condição de contorno \textit{bounceback} \textit{no-slip}. Ilustração adaptada do estudo de \citeonline{viggen2014lattice}.}
  \label{fig:bounceback}
\end{figure}

De acordo como é mostrado na Figura \ref{fig:bounceback}, a célula antes de chegar na condição de contorno possui suas funções de distribuição de partículas em cada sentido original e, ao cruzar a condição de contorno, em vez da mesma continuar se movimentando com as mesmas funções de distribuição, as mesmas contidas nos vetores de direção de propagação que apontam para o \textit{bounceback} são trocadas para o lado oposto assim como é ilustrado na Figura \ref{fig:bounceback}. No ponto de vista das equações de propagação o processo abordado fica

\begin{gather*}
  f_{6}(\textbf{x}, t + \Delta t) = f_{8}(\textbf{x}, t)\text{,  } f_{8}(\textbf{x}, t + \Delta t) = f_{6}(\textbf{x}, t), \\
  f_{2}(\textbf{x}, t + \Delta t) = f_{4}(\textbf{x}, t)\text{,  } f_{4}(\textbf{x}, t + \Delta t) = f_{2}(\textbf{x}, t) \text{ e }\\
  f_{5}(\textbf{x}, t + \Delta t) = f_{7}(\textbf{x}, t)\text{,  } f_{7}(\textbf{x}, t + \Delta t) = f_{5}(\textbf{x}, t).
\label{eq:bounceback}
\end{gather*}

\subsubsection{Condição Anecóica}

Consolidar uma condição do tipo anecóica, num método numérico de natureza temporal, é um desafio devido a propriedade de condição anecóica ou absorção de energia acústica estar relacionada ao domínio da frequência ($Z_{0}$ $=$ $\rho_{0}c_{0}$). Em vista desse contexto, levando em consideração absorção de pressão, entropia e pulsos de despredimento de vórtices, o trabalho de \citeonline{kam2006non} propõe uma condição de contorno explícita de absorção baseada em técnicas com esse mesmo fim, porém aplicadas na resolução das equações de Navier-Stokes usando DNS (\textit{Direct} \textit{Numerical} \textit{Simulation})\abreviatura{DNS}{\textit{Direct} \textit{Numerical} \textit{Simulation}}.

A condição de contorno de absorção ou \textit{Absorbing} \textit{Boundary} \textit{Condition} (ABC)\abreviatura{ABC}{\textit{Absorbing} \textit{Boundary} \textit{Condition}}, proposta por \citeonline{kam2006non}, consiste na adição de uma região de amortecimento para que os valores de pressão e velocidade convirjam assintoticamente a valores que caracterizam um fluido em repouso. Nesse sentido, valores alvos para um fluido em resposo de densidade ($\rho_{T}$ $=$ $\rho_{0}$) e velocidade (\textbf{$u_{T}$} $=$ $0$) são usados para calcular uma função de distribuição de amortecimento $f_{i}^{T}$\simbolo{$f_{i}^{T}$}{Função de distribuição de amortecimento}. Essa função de distribuição é definida da mesma forma que $f_{i}^{M}$, porém com os valores alvos de densidade e velocidade, ficando na forma
\begin{equation}
  f_{i}^{T} = \rho_{0}\varepsilon_{i}.
  \label{eq:f_alvo}
\end{equation}
Como essa técnica é explícita, o operador de colisão $\Omega_{i}$ é adaptado e recebe um novo termo de colisão, ficando na forma
\begin{equation}
  \Omega_{i} = -\frac{1}{\tau}(f_{i} - f_{i}^{M}) - \sigma(f_{i}^{M} - f_{i}^{T}),
  \label{eq:omega_i_abc}
\end{equation}
tal que $\sigma$ $=$ $\sigma_{T}(\delta/D)^{2}$ é o coeficiente de absorção, $\sigma_{T}$ é uma constante com valor de $0,3$, $\delta$ é a distância medida do começo da região de contorno no sentido da convergência assintótica e $D$ é o tamanho total da região de contorno no sentido da convergência assintótica.

O operador de colisão da equação \ref{eq:omega_i_abc} funciona bem para o modelo SRT, porém como nesse estudo será usado o modelo MRT algumas adaptações precisam ser realizadas, pois a operação de colisão ocorre no espaço dos momentos nesse modelo. Assim como é feito nas equações \ref{eq:MRT_2} deve-se aplicar o mesmo procedimento na função de distribuição $f_{i}^{T}$ ficando
\begin{equation}
  m_{i}^{T} = \textbf{$M$}f_{i}^{T}.
  \label{eq:abc_mrt_1}
\end{equation}
Além disso é preciso inserir esse termo no operador de colisão da equação \ref{eq:MRT_4} resultando em 

\begin{equation}
  \Omega_{i} = -\textbf{$M$}^{-1}\textbf{$S$}(m_{i} - m_{i}^{M}) - 
  \sigma\textbf{$M$}^{-1}\textbf{$S$}(m_{i}^{M} - m_{i}^{T}).
  \label{eq:abc_mrt_2}
\end{equation}

Simplificando a equação \ref{eq:abc_mrt_2} fica
\begin{equation}
  \Omega_{i} = -\textbf{$M$}^{-1}\textbf{$S$}[m_{i} - m_{i}^{M}(\sigma - 1) - m_{i}^{T}].
  \label{eq:abc_mrt_3}
\end{equation}

Adicionando esse termo na equação geral \ref{eq:f_i} do LBM o resultado é a equação
\begin{equation}
  f_{i}(\textbf{x} + c_{i}\Delta t, t + \Delta t) = f_{i}(\textbf{x}, t) -\textbf{$M$}^{-1}\textbf{$S$}[m_{i} - m_{i}^{M}(\sigma - 1) - m_{i}^{T}].
  \label{eq:abc_mrt_4}  
\end{equation}

\section{Palabos}

Aliado aos métodos númericos há várias tecnologias computacionais que são essenciais para aplicação e resolução de equações complexas. É bastante comum o uso de \textit{softwares} proprietários para desenvolvimento de modelos numéricos mas os mesmos restringem o desenvolvimento de estudos, ferramentas e produtos científicos. Dos fatores de restrição pode-se destacar:

\begin{itemize}
  \item não é possível saber com exatidão os métodos e equações que estão implementados no \textit{software};
  \item incapacidade do usuário realizar algum tipo de manutenção, correção de defeitos ou evolução no \textit{software};
  \item altos custos de licença de uso. 
\end{itemize}

Em vista das limitações dos \textit{softwares} proprietários, ou seja, projetos de código fechado há outras possibilidades para contornar essas limitações: \textit{softwares} livres. De acordo com \citeonline{stallman2002free} \textit{softwares} livres se caracterizam por cinco liberdades essenciais:

\begin{itemize}
  \item A liberdade de executar o programa como você desejar, para qualquer propósito;
  \item A liberdade de estudar como o programa funciona, e adaptá-lo às suas necessidades. Para tanto, acesso ao código-fonte é um pré-requisito;
  \item A liberdade de estudar como o programa funciona, e adaptá-lo às suas necessidades. Para tanto, acesso ao código-fonte é um pré-requisito;
  \item A liberdade de redistribuir cópias de modo que você possa ajudar ao próximo;
  \item A liberdade de distribuir cópias de suas versões modificadas a outros. Desta forma, você pode dar a toda comunidade a chance de beneficiar de suas mudanças. Para tanto, acesso ao código-fonte é um pré-requisito.
\end{itemize}

Tais liberdades favorecem o desenvolvimento de estudos, ferramentas e produtos científicos com um custo-benefício bastante apropriado para universidades. Em vista disso, projetos de \textit{softwares} livres se tornam atrativos para a implementação dos requisitos do LBM propostos nesse trabalho e, dentro das opções disponíveis no mercado atualmente, o \textit{software} \citeonline{palabos} é classificado como bastante conveniente visto que sua documentação é bastante abrangente, sua arquitetura de \textit{software} é totalmente modularizada com baixo acoplamento e alta coesão e sua tecnologia de processamento chega a ser no mínimo 20 vezes mais rápida que o MATLAB de acordo com \citeonline{numeric_palabos}.   

O \textit{software} livre Palabos é um projeto feito no paradígma computacional de orientação a objetos resultado da colaboração entre indústria e academia, focando produzir uma ferramenta de simulação computacional robusta, rápida e confiável. Todos os modelos nativos são implementados e testados como mostra os estudos de \citeonline{lattice_1}, \citeonline{lattice_2}, \citeonline{lattice_3}, \citeonline{lattice_4} e \citeonline{lattice_5}. De funcionalidades o \textit{software} possui:

\begin{itemize}
  \item dinâmicas físicas:
  \begin{itemize}
     \item equações de Navier-Stokes incompressível e fracamente compressível;
     \item escoamentos com força de corpo;
     \item escoamentos com diferenças de temperatura;
     \item fluidos multifásicos;
     \item modelo de turbulência Smagorinsky.
   \end{itemize}
  \item modelos básicos:
  \begin{itemize}
    \item BGK;
    \item MRT;
    \item entrópico.
  \end{itemize}
  \item condições de contorno:
  \begin{itemize}
    \item \textit{bounceback};
    \item Zou/He;
    \item Inamuro;
    \item Skordos;
    \item periódico;
    \item Dirichlet para velocidade ou pressão.
  \end{itemize}
  \item geração de malha:
  \begin{itemize}
    \item criação de malha a partir de arquivos CAD do tipo STL.  
  \end{itemize}
  \item grid:
  \begin{itemize}
    \item D2Q9;
    \item D3Q13;
    \item D3Q15;
    \item D3Q19;
    \item D3Q27;
    \item \textit{Multi-grid}. 
  \end{itemize}
  \item paralelismo:
  \begin{itemize}
    \item MPI em vários processadores;
    \item MPI em vários computadores em rede.
  \end{itemize}
  \item dados de saída:
  \begin{itemize}
    \item arquivos de dados em ASCII;
    \item arquivos de dados em formato binário;
    \item arquivos de dados em formato de imagem GIF;
    \item arquivos de dados em formato VTK para visualização no \textit{software} \citeonline{paraview}.
  \end{itemize}
\end{itemize}

Mesmo com várias funcionalidades nativas, o \textit{software} \citeonline{palabos} precisa ter outras outras funcionalidades implementadas para que possa atender o escopo desse trabalho. Para atender esse requisito, o projeto \citeonline{palabos_acoustic} foi criado como uma versão do \citeonline{palabos} que contém todos os modelos e implementações desenvolvidas nesse trabalho. As funcionalidades desenvolvidas nesse trabalho são:

\begin{itemize}
  
  \item condições de contorno:
  \begin{itemize}
    \item condição de contorno anecóica de \citeonline{kam2006non} para BGK D2Q9;
    \item condição de contorno anecóica de \citeonline{kam2006non} para MRT D2Q9 e D3Q19;
    \item condição de contorno para excitação do duto com \textit{sweep} ou soma de harmônicos.
  \end{itemize}

  \item geração de malha:
  \begin{itemize}
    \item criação automática de malha com vários tamanhos e espessuras de dutos.
  \end{itemize}

  \item dados de saída:
  \begin{itemize}
    \item relatórios automáticos de execução;
    \item dados e relatórios de execução organizados automaticamente por pastas com hora e data.
  \end{itemize}
\end{itemize}


Para executar o \citeonline{palabos_acoustic} é preciso dos seguintes \textit{softwares} básicos instalados como pré-requisitos:

\begin{itemize}
  \item sistema operacional linux Ubuntu 16.04 ou CentOS 7.2;
  \item compilador de C++ do tipo g++ 4.8;
  \item biblioteca de processamento paralelo Open MPI 1.10. 
\end{itemize}

Para cada novo modelo é preciso criar uma pasta com o nome do modelo contendo o arquivo de compilação \textbf{Makefile} e o código fonte do modelo numérico escrito em C++ com extenção \textbf{.cpp}. No arquivo \textbf{Makefile} é possível configurar aonde se encontra a instalação do Palabos, arquivo do modelo numérico com extenção \textbf{.cpp}, opções de depuração e opções de paralelização. No arquivo de extenção \textbf{.cpp} se encontra o código fonte do modelo numérico a ser simulado e o mesmo é composto de acordo com os procedimentos do fluxograma da Figura \ref{fig:palabos_fluxo}.


\begin{figure}[ht!]
\centering
  \includegraphics[width=.8\linewidth]{figuras/palabos_modelo_fluxo.pdf}
  \caption[Fluxograma de um modelo numérico no Palabos]{Fluxograma geral de um código fonte de um modelo numérico no Palabos.}
  \label{fig:palabos_fluxo}
\end{figure}

Como é mostrado na Figura \ref{fig:palabos_fluxo}, todo código de modelo numérico no Palabos possui os seguintes procedimentos:

\begin{itemize}
  \item importar bibliotecas: nesse procedimento são importadas as bibliotecas que contêm as funções e classes que serão usadas ao longo do processamento do modelo. Normalmente são bibliotecas do próprio Palabos ou bibliotecas com funções matemáticas;
  \item definir variáveis globais: normalmente nessa etapa são definidas valores de pré-processamento como o tamanho do domínio, valores macroscópicos do fluido como número de Reynolds, tempo total de simulação, viscosidade cinemática e o tipo de modelo LBM;
  \item definir condições iniciais e de contorno: nessa etapa a malha do domínio é consolidada, valores de densidade e velocidade são atribuídas para cada célula do domínio e condições de contorno são impostas;
  \item criar arquivos para gravação de dados: são criados ponteiros e arquivos de diversas extensões para que os dados sejam gravados;
  \item alterar condição de contorno: nessa etapa o modelo numérico entra no \textit{loop} de iterações e se necessário as condições de contorno são alteradas para, por exemplo, que um \textit{sweep} possa ser imposto;
  \item colidir: nessa etapa o operador de colisão é calculado e somado com as funções de distribuição de cada célula;
  \item propagar: os valores das funções de distribuição são propagados para células vizinhas;
  \item salvar dados: os dados normalmente de pressão e velocidades são salvos para pós-processamento. 
\end{itemize}
E assim o ciclo de procedimentos dentro do \textit{loop} é executado até que o número de iterações alcance o número máximo de tempo definido no início do programa.

Para execução é preciso efetuar os seguintes comandos no terminal linux dentro da pasta do modelo numérico:
\begin{itemize}
  \item compilação do código de extenção \textbf{.cpp} para formato binário em linguagem de máquina:
  \begin{lstlisting}[language=make, frame = single]
    $ make
  \end{lstlisting}
  \item execução do arquivo binário compilado:
  \begin{lstlisting}[language=make, frame = single]
    $ mpirun -np 
    <numero_de_processadores> 
    <nome_do_arquivo_compilado> 
    <parametros_de_entrada>
  \end{lstlisting}
  aplicando para o modelo numérico desse trabalho:
  \begin{lstlisting}[language=make, frame = single]
    $ mpirun -np 
    8
    duct_radiation_optimization
    20 0.15 1.99
  \end{lstlisting}
  tal que o raio do duto é 20 células, o mach do escoamento é 0.15 e 1.99 é a frequência de relaxação $1/\tau$. É possível também executar o Palabos com o \textit{script} \textbf{duct\_radiation\_init.m} na plataforma \citeonline{matlab} ou \citeonline{octave}. Para executar basta colocar esse \textit{script} dentro da pasta do modelo numérico e executar o seguinte comando no terminal do \citeonline{matlab} ou \citeonline{octave} dentro dessa pasta:
  \begin{lstlisting}[language=matlab, frame = single]
    >> duct_radiation_init 20 0.15 5042 8
  \end{lstlisting}
  tal que o raio do duto é 20 células, o mach do escoamento é 0.15, o número de Reynolds é 5042 e o 8 é a quantidade de processadores.
\end{itemize}


\section{Modelo Numérico}
\chapter{Resultados}

Em vista da teoria vigente na literatura e pelo que foi exposto no ponto de vista metodológico, obteve-se resultados nos seguintes contextos:
\begin{itemize}
  \item análise da condição anecóica: a partir dos históricos temporais de pressão e velocidade de partícula nas fronteiras do modelo numérico, foram feitas análises críticas de reflexão acústica;
  \item duto sem escoamento: a partir dos históricos temporais de pressão e velocidade de partícula na terminação do duto, foram feitas validações e análises críticas dos parâmetros caracterizadores da acústica interna do duto sem escoamento;
  \item duto com escoamento de exaustão: a partir dos históricos temporais de pressão e velocidade de partícula na terminação do duto, foram feitas validações e análises críticas dos parâmetros caracterizadores da acústica interna do duto com escoamento de exaustão para regimes subsônicos ($M$ $\leq$ 0,2);
  \item duto com escoamento sugado: a partir dos históricos temporais de pressão e velocidade de partícula na terminação do duto, foram feitas validações, análises críticas e investigação dos parâmetros caracterizadores da acústica interna do duto com escoamento succionado para regimes subsônicos ($M$ $\leq$ 0,2).
\end{itemize} 

\section{Análise da Condição Anecóica}

Com a finalidade mensurar e analisar o comportamento da condição de contorno anecóica por meio de métricas numéricas e objetivas, foram calculados impedâncias e coeficientes de reflexão nas fronteiras do modelo numérico, ou seja, nos pontos \textbf{A}, \textbf{B}, \textbf{C} e \textbf{D} como é mostrado na Figura \ref{fig:modelo}. As Figuras \ref{fig:resultados_A}, \ref{fig:resultados_B}, \ref{fig:resultados_C} e \ref{fig:resultados_D} mostram os resultados nos respectivos pontos citados.


% \begin{figure}
% \begin{center}
% \begin{tikzpicture}
% \begin{axis}[
%     title={},
%     width=0.9\textwidth,
%     height=0.45\textwidth,
%     xlabel={Frequência},
%     ylabel={Sensibilidade},
%     x unit={\space Hz \space},
% 	y unit={\space C/g \space},
%     ytick=data,
%     xmin=0,
%     ymin=0,
%     ymax=0.55,
%     legend pos=north west,
%     grid=minor, % Display a grid	
% 	grid style={dashed,gray!90}, % Set the style
% 	]
%      %\addplot[color=black,dashed,thick,mark=*,mark options={solid},smooth] table[x index=2,y index=3] {Data/Kollias_SBMR_095_exp.txt}; \label{Kollias_exp095}
%      %\addlegendentry{Experimental (Kollias)}
%       %\addplot[color=blue,semithick] table[x index=0,y index=1] {Data/Kollias_SBMR_095_exp.txt}; \label{Kollias_sim095}
%     %\addlegendentry{Simulação (Kollias)}
%              \addplot[color=black, thick] table[x index=0,y index=1] {dados/coeficiente_reflexao_anecoica/A_real.txt};
%              \addplot[color=black,dashed,  thick] table[x index=0,y index=1] {dados/coeficiente_reflexao_anecoica/A_imag.txt};
%               \label{Comsol_095}% \addlegendentry{Simulação (Comsol)}
% \end{axis}
% \end{tikzpicture}
%  \caption{Resposta em carga do acelerômetro para $r$ igual a 0,95.}
% \end{center}
% \end{figure}

\newcommand\scalex{1}
\newcommand\scaley{0.9}
\newcommand\scaleA{0.5}

\begin{figure}
\begin{subfigure}{\scaleA \textwidth}
  \input{figuras/impedancia_A.tex}
\end{subfigure}%
\begin{subfigure}{\scaleA \textwidth}
  \input{figuras/reflexao_A.tex}
\end{subfigure}
\caption{Resultados da impedância (\ref{fig:A_impedancia}) e coeficiente de reflexão (\ref{fig:A_reflexao}) calculados no ponto $\textbf{A}$. Na Figura (\ref{fig:A_impedancia}) a linha contínua representa a parte real e a linha tracejada representa a parte imaginária.}
\label{fig:resultados_A}
\end{figure}

\begin{figure}
\begin{subfigure}{\scaleA \textwidth}
  \input{figuras/impedancia_B.tex}
\end{subfigure}%
\begin{subfigure}{\scaleA \textwidth}
  \input{figuras/reflexao_B.tex}
\end{subfigure}
\caption{Resultados da impedância (\ref{fig:B_impedancia}) e coeficiente de reflexão (\ref{fig:B_reflexao}) calculados no ponto $\textbf{B}$. Na Figura (\ref{fig:B_impedancia}) a linha contínua representa a parte real e a linha tracejada representa a parte imaginária.}
\label{fig:resultados_B}
\end{figure}

\begin{figure}
\begin{subfigure}{\scaleA \textwidth}
  \begin{tikzpicture}
  \begin{axis}[
	width=\scalexA \textwidth,
  height=\scaleyA \textwidth,
	xmin=0,
  xmax=2.5,
    ymin=0,
    ymax=0.7,
    ytick distance=0.1,
    xtick distance=0.5,
    grid=major, % Display a grid	
 	%grid style={dashed,gray!90}, % Set the style
	xlabel = \small{$ka$},
	ylabel = \small{$Z_{\textbf{C}}$},
   y tick label style={/pgf/number format/.cd,%
          scaled y ticks = false,
          set decimal separator={,},
          fixed},
      x tick label style={/pgf/number format/.cd,%
          scaled x ticks = false,
          set decimal separator={,},
          fixed}%
  ]
 \addplot[color=black, thick] table[x index=0,y index=1] {dados/coeficiente_reflexao_anecoica/C_real.txt};
   \addplot[color=black,dashed,  thick] table[x index=0,y index=1] {dados/coeficiente_reflexao_anecoica/C_imag.txt};
   
  \end{axis}
  \end{tikzpicture}
  \caption[Impedância $Z_{\textbf{C}}$]{}
  \label{fig:C_impedancia}
  %\caption[Impedância $Z_{\textbf{C}}$]{Resultado da impedância calculada no ponto $\textbf{C}$, próximo da condição anecóica localizada nas fronteiras do modelo numérico. A linha contínua representa a parte real e a linha tracejada representa a parte imaginária.}

\end{subfigure}%
\begin{subfigure}{\scaleA \textwidth}
  \input{figuras/reflexao_C.tex}
\end{subfigure}
\caption{Resultados da impedância (\ref{fig:C_impedancia}) e coeficiente de reflexão (\ref{fig:C_reflexao}) calculados no ponto $\textbf{C}$. Na Figura (\ref{fig:C_impedancia}) a linha contínua representa a parte real e a linha tracejada representa a parte imaginária.}
\label{fig:resultados_C}
\end{figure}

\begin{figure}
\begin{subfigure}{\scaleA \textwidth}
  \input{figuras/impedancia_D.tex}
\end{subfigure}%
\begin{subfigure}{\scaleA \textwidth}
  \input{figuras/reflexao_D.tex}
\end{subfigure}
\caption{Resultados da impedância (\ref{fig:D_impedancia}) e coeficiente de reflexão (\ref{fig:D_reflexao}) calculados no ponto $\textbf{D}$. Na Figura (\ref{fig:D_impedancia}) a linha contínua representa a parte real e a linha tracejada representa a parte imaginária.}
\label{fig:resultados_D}
\end{figure}


\newpage
\section{Duto sem Escoamento}

\begin{figure}[ht!]
\centering
  \begin{tikzpicture}
  \begin{axis}[
  width=0.9\textwidth,
  height=0.5\textwidth,
  xmin=0,
  xmax=1.8,
    ymin=0.4,
    ymax=1,
    ytick distance=0.1,
    grid=major, % Display a grid  
  %grid style={dashed,gray!90}, % Set the style
  xlabel = Número de Helmholtz ($ka$),
  ylabel = Coeficiente de Reflexão ($R_{r}$),
  ]
 \addplot[color=black, thick] table[x index=0,y index=1] {dados/duto_sem_escoamento/analytical_data_abs_r.txt};
 \addplot[color=black, mark=o, only marks] table[x index=0,y index=1] {dados/duto_sem_escoamento/simulation_data_abs_r.txt};
   
  \end{axis}
  \end{tikzpicture}
  \caption[Coeficiente de Reflexão $R_{r}$ sem Escoamento]{Resultado da magnitude do coeficiente de reflexão $R_{r}$ calculado no ponto $\textbf{P}$ na terminação do duto sem escoamento. A linha contínua representa o resultado analítico do estudo de \citeonline{levine1948radiation} e os pontos circulares representam os resultados calculados pela ferramenta computacional proposta nesse estudo. A correlação entre os resultados foi de 99,95 \%.}
  \label{fig:abs_r_boca}
\end{figure}

\newpage
\begin{figure}[ht!]
\centering
  \begin{tikzpicture}
  \begin{axis}[
  width=0.9\textwidth,
  height=0.5\textwidth,
  xmin=0,
  xmax=1.8,
    ymin=0.4,
    ymax=0.65,
    ytick distance=0.05,
    grid=major, % Display a grid  
  %grid style={dashed,gray!90}, % Set the style
  xlabel = Número de Helmholtz ($ka$),
  ylabel = Correção da Terminação ($l/a$),
  ]
 \addplot[color=black, thick] table[x index=0,y index=1] {dados/duto_sem_escoamento/analytical_data_loa.txt};
 \addplot[color=black, mark=o, only marks] table[x index=0,y index=1] {dados/duto_sem_escoamento/simulation_data_loa.txt};
   
  \end{axis}
  \end{tikzpicture}
  \caption[Coeficiente de Correção da Terminação $l/a$ sem Escoamento]{Resultado do coeficiente de correção da terminação $l/a$ calculado no ponto $\textbf{P}$ na terminação do duto sem escoamento. A linha contínua representa o resultado analítico do estudo de \citeonline{levine1948radiation} e os pontos circulares representam os resultados calculados pela ferramenta computacional proposta nesse estudo. A correlação entre os resultados foi de 96,23 \%.}
  \label{fig:loa_boca}
\end{figure}

\newpage
\section{Duto com Escoamento de Exaustão}

\subsection{Mach 0,2}
\begin{figure}[ht!]
\centering
  \begin{tikzpicture}
  \begin{axis}[
  width=0.9\textwidth,
  height=0.5\textwidth,
  xmin=0,
  xmax=1.8,
    ymin=0.4,
    ymax=1.2,
    ytick distance=0.1,
    grid=major, % Display a grid  
  %grid style={dashed,gray!90}, % Set the style
  xlabel = Número de Helmholtz ($ka$),
  ylabel = Coeficiente de Reflexão ($R_{r}$),
  ]
 \addplot[color=black, thick] table[x index=0,y index=1] {dados/duto_exaustao/abs_r_020_analytical.txt};
 \addplot[color=black, mark=o, only marks] table[x index=0,y index=1] {dados/duto_exaustao/abs_r_020_simulation.txt};
   
  \end{axis}
  \end{tikzpicture}
  \caption[Coeficiente de Reflexão $R_{r}$ com Escoamento de Exaustão (M $=$ 0,2)]{Resultado da magnitude do coeficiente de reflexão $R_{r}$ calculado no ponto $\textbf{P}$ na terminação do duto com escoamento de exaustão (M $=$ 0,2 e Re $=$ 5514,82). A linha contínua representa o resultado analítico do estudo de \citeonline{munt1990acoustic} e os pontos circulares representam os resultados calculados pela ferramenta computacional proposta nesse estudo. A correlação entre os resultados foi de 98,1 \%.}
  \label{fig:abs_r_boca_020}
\end{figure}

\newpage
\begin{figure}[ht!]
\centering
  \begin{tikzpicture}
  \begin{axis}[
  width=0.9\textwidth,
  height=0.5\textwidth,
  xmin=0,
  xmax=1.8,
    ymin=0.15,
    ymax=0.55,
    ytick distance=0.05,
    grid=major, % Display a grid  
  %grid style={dashed,gray!90}, % Set the style
  xlabel = Número de Helmholtz ($ka$),
  ylabel = Correção da Terminação ($l/a$),
  ]
 \addplot[color=black, thick] table[x index=0,y index=1] {dados/duto_exaustao/loa_020_analytical.txt};
 \addplot[color=black, mark=o, only marks] table[x index=0,y index=1] {dados/duto_exaustao/loa_020_simulation.txt};
   
  \end{axis}
  \end{tikzpicture}
  \caption[Coeficiente de Correção da Terminação $l/a$ com Escoamento de Exaustão (M $=$ 0,2)]{Resultado do coeficiente de correção da terminação $l/a$ calculado no ponto $\textbf{P}$ na terminação do duto com escoamento de (M $=$ 0,2 e Re $=$ 5514,82). A linha contínua representa o resultado analítico do estudo de \citeonline{munt1990acoustic} e os pontos circulares representam os resultados calculados pela ferramenta computacional proposta nesse estudo. A correlação entre os resultados foi de 79,84 \%.}
  \label{fig:loa_boca_020}
\end{figure}

\newpage
\subsection{Mach = 0,15}

\begin{figure}[ht!]
\centering
  \begin{tikzpicture}
  \begin{axis}[
  width=0.9\textwidth,
  height=0.5\textwidth,
  xmin=0,
  xmax=1.8,
    ymin=0.4,
    ymax=1.2,
    ytick distance=0.1,
    grid=major, % Display a grid  
  %grid style={dashed,gray!90}, % Set the style
  xlabel = Número de Helmholtz ($ka$),
  ylabel = Coeficiente de Reflexão ($R_{r}$),
  ]
 \addplot[color=black, thick] table[x index=0,y index=1] {dados/duto_exaustao/abs_r_015_analytical.txt};
 \addplot[color=black, mark=o, only marks] table[x index=0,y index=1] {dados/duto_exaustao/abs_r_015_simulation.txt};
   
  \end{axis}
  \end{tikzpicture}
  \caption[Coeficiente de Reflexão $R_{r}$ com Escoamento de Exaustão (M $=$ 0,15)]{Resultado da magnitude do coeficiente de reflexão $R_{r}$ calculado no ponto $\textbf{P}$ na terminação do duto com escoamento de exaustão (M $=$ 0,15 e Re $=$ 2057,71). A linha contínua representa o resultado analítico do estudo de \citeonline{munt1990acoustic} e os pontos circulares representam os resultados calculados pela ferramenta computacional proposta nesse estudo. A correlação entre os resultados foi de 99,80 \%.}
  \label{fig:abs_r_boca_015}
\end{figure}

\newpage
\begin{figure}[ht!]
\centering
  \begin{tikzpicture}
  \begin{axis}[
  width=0.9\textwidth,
  height=0.5\textwidth,
  xmin=0,
  xmax=1.8,
    ymin=0.15,
    ymax=0.55,
    ytick distance=0.05,
    grid=major, % Display a grid  
  %grid style={dashed,gray!90}, % Set the style
  xlabel = Número de Helmholtz ($ka$),
  ylabel = Correção da Terminação ($l/a$),
  ]
 \addplot[color=black, thick] table[x index=0,y index=1] {dados/duto_exaustao/loa_015_analytical.txt};
 \addplot[color=black, mark=o, only marks] table[x index=0,y index=1] {dados/duto_exaustao/loa_015_simulation.txt};
   
  \end{axis}
  \end{tikzpicture}
  \caption[Coeficiente de Correção da Terminação $l/a$ com Escoamento de Exaustão (M $=$ 0,15)]{Resultado do coeficiente de correção da terminação $l/a$ calculado no ponto $\textbf{P}$ na terminação do duto com escoamento de (M $=$ 0,15 e Re $=$ 2057,71). A linha contínua representa o resultado analítico do estudo de \citeonline{munt1990acoustic} e os pontos circulares representam os resultados calculados pela ferramenta computacional proposta nesse estudo. A correlação entre os resultados foi de 94,28 \%.}
  \label{fig:loa_boca_015}
\end{figure}


\newpage
\section{Duto com Escoamento Sugado}

\begin{figure}[ht!]
\centering
  \begin{tikzpicture}
  \begin{axis}[
  width=0.9\textwidth,
  height=0.5\textwidth,
  x tick label style={
      /pgf/number format/.cd,
          fixed,
          fixed zerofill,
          precision=2,
      /tikz/.cd
  },
  xmin=0,
  xmax=0.2,
  ymin=0.68,
  ymax=1,
  ytick distance=0.05,
  xtick distance=0.05,
  grid=major, % Display a grid  
  %grid style={dashed,gray!90}, % Set the style
  xlabel = Número de Mach ($M$),
  ylabel = Coeficiente de Reflexão ($R_{M}$),
  ]
 \addplot[color=black, thick] table[x index=0,y index=1] {dados/duto_sugado/davis_analytical.txt};
 \addplot[color=black, mark=o, only marks] table[x index=0,y index=1] {dados/duto_sugado/davis_simulation.txt};
   
  \end{axis}
  \end{tikzpicture}
  \caption[Coeficiente de reflexão $R_{M}$ com escoamento sugado]{Resultado do coeficiente de reflexão $R_{M}$ em relação ao Mach para baixas frequências ($ka$ $<$ $0,25$) com escoamento sugado. A linha contínua apresenta o resultado do estudo de \citeonline{davies1987} e os pontos circulares representam os resultados calculados pela ferramenta computacional proposta nesse estudo. A correlação entre os resultados foi de 95,45 \%.}

  \label{fig:abs_r_boca_sugado}
\end{figure}

\newpage
\begin{figure}[ht!]
\centering
  \begin{tikzpicture}
  \begin{axis}[
  width=0.9\textwidth,
  height=0.5\textwidth,
  x tick label style={
      /pgf/number format/.cd,
          fixed,
          fixed zerofill,
          precision=2,
      /tikz/.cd
  },
  xmin=0,
  xmax=0.2,
  ymin=0.1,
  ymax=1.0,
  ytick distance=0.1,
  xtick distance=0.05,
  grid=major, % Display a grid  
  %grid style={dashed,gray!90}, % Set the style
  xlabel = Número de Mach ($M$),
  ylabel = Correção da Terminação ($l_{M}$),
  ]
 \addplot[color=black, thick] table[x index=0,y index=1] {dados/duto_sugado/davis_analytical_loa.txt};
 \addplot[color=black, mark=o, only marks] table[x index=0,y index=1] {dados/duto_sugado/davis_simulation_loa.txt};
   
  \end{axis}
  \end{tikzpicture}
  \caption[Coeficiente de correção da terminação ($l_{M}$) com escoamento sugado]{Resultado do coeficiente de correção da terminação $l_{M}$ em relação ao Mach para baixas frequências ($ka$ $<$ $0,25$) com escoamento sugado. A linha contínua apresenta o resultado do estudo de \citeonline{davies1987} e os pontos circulares representam os resultados calculados pela ferramenta computacional proposta nesse estudo. A correlação entre os resultados foi de 62,53 \%.}

  \label{fig:abs_r_boca_sugado_loa}
\end{figure}
\chapter{Conclusões}

Nesse trabalho foi desenvolvida uma ferramenta computacional para análise do coeficiente de reflexão para modos normais em dutos na presença de escoamentos de baixo número de Mach ($M \leq 0,2$). 

Foi implementado um esquema computacional para avaliação do coeficiente de reflexão em dutos a partir do método de \textit{lattice} Boltzmann. Esse esquema foi desenvolvido em C++ orientado a objetos dentro do \textit{software} Palabos e fez uso do modelo MRT, condição de contorno de paredes rígidas e condição de contorno de absorção de energia acústica adaptada ao MRT. Os resultados mostraram que o esquema computacional funciona de acordo com os resultados da literatura e que a condição de absorção de energia acústica se comporta aproximadamente como impedância do meio.

Condições de contorno necessárias foram construídas, afim de representar o problema da reflexão de onda em dutos na presença de baixos números de Mach. As condições de contorno foram aplicadas num modelo numérico tridimensional de um duto não flangeado com espessura de paredes de $10 \%$ do tamanho do raio do duto. Além disso foram adaptadas as distâncias necessárias dos limites do domínio numérico em relação ao duto para que haja conservação da massa e que a condição de contorno de absorção possa se comportar regularmente. Os resultados mostraram que o modelo numérico é estável e representa o comportamento físico esperado num regime de baixos números de Mach.

Foi implementado, validado e analisado o comportamento acústico interno de dutos não flangeados com e sem escoamento de exaustão e com ondas planas. Os coeficientes de reflexão e de correção da terminação foram extraídos do modelo numérico com rotinas de pós-processamentos. Os mesmos foram comparados e analisados e possuem uma correlação em média de 90\%  com os resultados da litetura, demonstrando boa concordância com os fenômenos físicos abordados na litetura. Houve algumas divergências no coeficiente de correção da terminação num regime de exaustão e podem ser explicadas pelo fato do método de cálculo do pós-processamento não considerar a presença de escoamentos.  

O comportamento acústico interno de dutos não flangeados com escoamento sugado e com ondas planas foi implementado, validado e analisado. Os coeficientes de reflexão e de correção da terminação foram extraídos e pós-processados do modelo numérico num regime de escoamento sugado e comparado com os dados disponíveis na litetura. Apesar do coeficiente de correção da terminação não ter tido uma boa correlação, o coeficiente de reflexão foi calculado com 98,35\% de correlação demonstrando uma boa concordância com os dados da literatura. Apesar da literatura ter somente disponíveis resultados em baixas frequências ($ka \leq 0,25$) para escoamento sugado, foram calculados e analisados coeficientes de reflexão para vários números de Mach em médias e altas frequências ($ka > 0,25$). As análises demonstraram que o coeficiente de reflexão num contexto de escoamento sugado é altamente sensível a diferentes números de Mach, havendo sobretudo amplificação acima da faixa unitária para números de Strouhal $St \sim \frac{\pi}{2}$. Esse fenômeno pode ser explicado pelo fato do campo fluido dinâmico interagir com o campo acústico através de desprendimento de vórtices. Vale ressaltar também que a variação do coeficiente de reflexão em relação a vários Machs em $St \sim \frac{\pi}{2}$, diferentemente do que ocorre em regime de escoamento de exaustão que é monotônico, varia de forma não monotônica e possui um máximo em $M \sim 0,07$. Esse fenômeno pode ser explicado pela natureza do desprendimento de vórtices numa vena contracta.  


%%%%%%%%%%%%%%%%%%%%%%%%%%%%%%%%%%%%%%%%%%%%%%%%%%%%%%%%%%%%%%%%%%%%%%%%%


\bibliographystyle{ufscThesis/ufsc-alf}
\bibliography{bibliografia}
%%%%%%%%%%%%%%%%%%%%%	REFERÊNCIAS    %%%%%%%%%%%%%%%%%%%%%%%%%%%%%%%%%%%

%--------------------------------------------------------
% Elementos pós-textuais
\apendice
\chapter{Manual de Funcionamento do Palabos Acoustic}

Para executar o \citeonline{palabos_acoustic} é preciso dos seguintes \textit{softwares} básicos instalados como pré-requisitos:

\begin{itemize}
  \item sistema operacional linux Ubuntu 16.04 ou CentOS 7.2;
  \item compilador de C++ do tipo g++ 4.8;
  \item biblioteca de processamento paralelo Open MPI 1.10. 
\end{itemize}

Para cada novo modelo é preciso criar uma pasta com o nome do modelo contendo o arquivo de compilação \textbf{Makefile} e o código fonte do modelo numérico escrito em C++ com extenção \textbf{.cpp}. No arquivo \textbf{Makefile} é possível configurar aonde se encontra a instalação do Palabos, arquivo do modelo numérico com extenção \textbf{.cpp}, opções de depuração e opções de paralelização. No arquivo de extenção \textbf{.cpp} se encontra o código fonte do modelo numérico a ser simulado e o mesmo é composto de acordo com os procedimentos do fluxograma da Figura \ref{fig:palabos_fluxo}.


\begin{figure}[ht!]
\centering
  \includegraphics[width=.8\linewidth]{figuras/palabos_modelo_fluxo.pdf}
  \caption[Fluxograma de um modelo numérico no Palabos]{Fluxograma geral de um código fonte de um modelo numérico no Palabos.}
  \label{fig:palabos_fluxo}
\end{figure}

Como é mostrado na Figura \ref{fig:palabos_fluxo}, todo código de modelo numérico no Palabos possui os seguintes procedimentos:

\begin{itemize}
  \item importar bibliotecas: nesse procedimento são importadas as bibliotecas que contêm as funções e classes que serão usadas ao longo do processamento do modelo. Normalmente são bibliotecas do próprio Palabos ou bibliotecas com funções matemáticas;
  \item definir variáveis globais: normalmente nessa etapa são definidas valores de pré-processamento como o tamanho do domínio, valores macroscópicos do fluido como número de Reynolds, tempo total de simulação, viscosidade cinemática e o tipo de modelo LBM;
  \item definir condições iniciais e de contorno: nessa etapa a malha do domínio é consolidada, valores de densidade e velocidade são atribuídas para cada célula do domínio e condições de contorno são impostas;
  \item criar arquivos para gravação de dados: são criados ponteiros e arquivos de diversas extensões para que os dados sejam gravados;
  \item alterar condição de contorno: nessa etapa o modelo numérico entra no \textit{loop} de iterações e se necessário as condições de contorno são alteradas para, por exemplo, que um \textit{sweep} possa ser imposto;
  \item colidir: nessa etapa o operador de colisão é calculado e somado com as funções de distribuição de cada célula;
  \item propagar: os valores das funções de distribuição são propagados para células vizinhas;
  \item salvar dados: os dados normalmente de pressão e velocidades são salvos para pós-processamento. 
\end{itemize}
E assim o ciclo de procedimentos dentro do \textit{loop} é executado até que o número de iterações alcance o número máximo de tempo definido no início do programa.

Para execução é preciso efetuar os seguintes comandos no terminal linux dentro da pasta do modelo numérico:
\begin{itemize}
  \item compilação do código de extenção \textbf{.cpp} para formato binário em linguagem de máquina:
  \begin{lstlisting}[language=make, frame = single]
    $ make
  \end{lstlisting}
  \item execução do arquivo binário compilado:
  \begin{lstlisting}[language=make, frame = single]
    $ mpirun -np 
    <numero_de_processadores> 
    <nome_do_arquivo_compilado> 
    <parametros_de_entrada>
  \end{lstlisting}
  aplicando para o modelo numérico desse trabalho:
  \begin{lstlisting}[language=make, frame = single]
    $ mpirun -np 
    8
    duct_radiation_optimization
    20 0.15 1.99
  \end{lstlisting}
  tal que o raio do duto é 20 células, o mach do escoamento é 0.15 e 1.99 é a frequência de relaxação $1/\tau$. É possível também executar o Palabos com o \textit{script} \textbf{duct\_radiation\_init.m} na plataforma \citeonline{matlab} ou \citeonline{octave}. Para executar basta colocar esse \textit{script} dentro da pasta do modelo numérico e executar o seguinte comando no terminal do \citeonline{matlab} ou \citeonline{octave} dentro dessa pasta:
  \begin{lstlisting}[language=matlab, frame = single]
    >> duct_radiation_init 20 0.15 5042 8
  \end{lstlisting}
  tal que o raio do duto é 20 células, o mach do escoamento é 0.15, o número de Reynolds é 5042 e o 8 é a quantidade de processadores.
\end{itemize}


%\anexo
%\chapter{Exemplificando um Anexo}
%Texto do anexo aqui.
\end{document}
